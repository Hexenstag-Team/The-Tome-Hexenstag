%%%%%%%%%%%%%%%%%%%%%%%%%%%%%%%%%%%%%%%%%%%%%%%%%%
\classentry{Monster Tamer}
%%%%%%%%%%%%%%%%%%%%%%%%%%%%%%%%%%%%%%%%%%%%%%%%%%

Most Monster Tamers dedicate their lives to taming Monsters very early in life. Monster Tamers generally come from single parent homes or are orphans. Many Monster Tamers learn their skills because they love Monsters or are simply competitive - while others see Monsters as a relatively easy path to power and dominate their Monsters in order to fuel their lusts for eternal acquisitiveness. Such Monster Tamers may turn to theft or extortion to attempt to steal the Monsters of other Monster Tamers.

\textbf{Alignment:} Usually Extreme Alignment, they tend to shy away from neutrality as their constant battles of will with Monsters generally make them quite accustomed to choosing sides.

\textbf{Starting Gold:} 4d6x10gp (140gp)

\textbf{Starting Age:} Simple

\textbf{Hit Die:} d6

\textbf{Class Skills:} The \currentclassname{}'s class skills  are \linkskill{Balance} (Dex), \linkskill{Climb} (Str), \linkskill{Craft} (Int), \linkskill{Craft} Gemcutting (Int), \linkskill{Handle Animal} (Cha), \linkskill{Heal} (Wis), \linkskill{Hide} (Dex), \linkskill{Jump} (Str), \linkskill{Knowledge} (Int), \linkskill{Knowledge} Dungeoneering (Int), \linkskill{Listen} (Wis), \linkskill{Move Silently} (Dex), \linkskill{Profession} (Wis), \linkskill{Ride} (Dex), \linkskill{Search} (Int), \linkskill{Sense Motive} (Wis), \linkskill{Speak Language} (n/a), \linkskill{Spot} (Wis), \linkskill{Survival} (Wis) \linkskill{Swim} (Str), \linkskill{Tumble} (Dex), and \linkskill{Use Rope} (Dex).

\textbf{Skills/Level:} 4 + Intelligence Bonus

\modebab{}
\poorfor{}
\goodref{}
\poorwil{}

\begin{classtable}
\levelone{Control Monster, Train Monster, Dread Lore, Wild Empathy}
\leveltwo{Craft Soul Prison, Heal Monster}
\levelthree{Subtype Specialization}
\levelfour{Increased Awareness, Double Team}
\levelfive{Speak with Monsters}
\levelsix{Craft Greater Soul Prison}
\levelseven{Type Specialization}
\leveleight{Transfer Control}
\levelnine{Advanced Monster Healing}
\levelten{Craft Leaden Seal}
\leveleleven{Store Monster, Recall Monster}
\leveltwelve{Second Subtype Specialization}
\levelthirteen{-}
\levelfourteen{Second Type Specialization}
\levelfifteen{Craft Master Prison}
\levelsixteen{-}
\levelseventeen{Fast Recall Monster}
\leveleighteen{Third Subtype Specialization}
\levelnineteen{Third Type Mastery}
\leveltwenty{Subtype Mastery}
\end{classtable}

\classfeatures

\textbf{Weapon and Armor Proficiency:} Monster Tamers are proficient with all simple weapons, sap, nets, bolas, and whips. Monster Tamers have proficiency only with light armor. Monster Tamers are considered proficient when using any bludgeoning weapon they are normally proficient with for inflicting subdual damage (thus, they do not suffer a -4 to-hit penalty when attempting to inflict subdual damage with any bludgeoning weapon they are proficient with).

\textbf{Caster Levels:} Even though Monster Tamers do not gain spells per day or have spell levels - Monster Tamers have many caster level dependant abilities. A Monster Tamer gains a Monster Tamer caster level for every Monster Tamer class level. If a Monster Tamer gains a Prestige Class which adds to Caster levels - she may choose to raise Monster Tamer caster levels instead of other caster levels.

\textbf{Monster:} A Monster is any Aberration, Animal, Dragon, Elemental, Magical Beast, Ooze, Outsider, Plant, or Vermin which advances by Hit Dice rather than By Character Class.

\textbf{Control Monster (Ex):} A Monster Tamer can have a number of Monsters in Soul Prisons equal to her Charisma Modifier. A Controlled Monster behaves like a summoned monster when released from its Soul Prisonr. A Monster Tamer cannot control a Monster whose Challenge Rating is greater than the Monster Tamer's Caster Level.
An uncontrolled Monster will act as if set free, possibly going on a rampage, running away, or simply sleeping until it is returned to its Soul Prison. 
Furthermore, Dragon type Monsters are harder to control than other Monsters, and use twice their CR (or their own CR + 4, whichever is less) to determine whether they will obey their Monster Tamer. A Controlled Monster cannot use any Summoning ability. More than one controlled Monster can be out of their balls at any one time - but only the first one released behaves like a summoned monster - any subsequent released Monster will act normally, usually standing around and watching events transpire, or sleeping (extreme events can cause them to take direct action at DM's option).

Once a Monster Tamer has reached the limit of the number of Monsters which can be controlled, the Monster Tamer cannot control any more until one or more of the controlled Monsters are released from control or killed. Releasing a Monster from control takes about 10 minutes. Control can be reasserted, but only if the Monster Tamer has the ability to control that many Monsters.

\textbf{Train Monster (Ex):} A Monster Tamer can train or evolve their Monster with their \linkskill{Handle Animal} skill. As an extraordinary ability,
a Monster Tamer need not choose specific animals as trainable and can use \linkskill{Handle Animal} on any Monster. Training a Monster
takes 8 hours and has a DC of 15 + Monster's (new) CR. The effects available from Training Monster are based on the number of
Ranks in \linkskill{Handle Animal} the Monster Tamer has:

\begin{itemize*}
\item \textbf{3 ranks - Learn Trick:}  This is just like teaching to an animal companion (see DMG page 205). Note that some Monsters are intelligent enough so that they are able to perform Tricks without being specifically taught - and all Monsters are able to learn at least 4 tricks even if their intelligence would not normally be high enough.
\item \textbf{6 ranks - Grow Monster:} This causes the Monster to advance 1 Hit Die, if it would not cause the Monster to exceed its advancement limit. This may cause the creature to grow in size category, see the monster description. This may also cause the Monster to become uncontrolled, if this raises its CR to past the maximum CR the Monster Tamer can control. You select
what skills, if any, a Monster Tamer gains for its level, and if this would cause a Monster to gain a feat you may select the feat.
\item \textbf{9 ranks - Evolve Monster:}: This causes the Monster to evolve to a more advanced form. The Monster gains a template of your choice. Note that this may cause the Monster to become uncontrolled, if this raises the CR to past the maximum CR the Monster Tamer can control. The Monster remains a Monster even if its type changes to a type which is not normally a Monster. Monsters who become Dragons in this way are not harder to control than natural dragons are. You select what skills, if any, a Monster gains with its template, and if this would cause a Monster to gain one or more feats you may select the feat(s). At the DM's option, a Monster may be evolved into a similar but more powerful form that is normally represented by a separate entry. For example: a DM might allow a Monster Tamer to evolve her Red Slaad into a Green Slaad, or a Fiendish Horse into a Nightmare. A Monster that type changes into a different type, gains a permanent one-time "Hard to Control" modifier as if its CR was 1 higher than it actually is.
\item \textbf{12 ranks - Inspire Monster:} You may be an especially kind or cruel master to your Monster, giving it a permanent +2 Sacred or Profane bonus to any statistic. You may only give this bonus once to each Monster, and you cannot give different bonuses (Sacred or Profane) to different Monsters.
\end{itemize*}

\textbf{Dread Lore (Ex):} A Monster Monster Tamer accumulates significant knowledge about the Monsters that they face. The amount of knowledge a Monster Monster Tamer has on an encountered wild monster is linked to the Monster Tamer's \linkskill{Knowledge} Dungeoneering. The abilities granted depend upon how many ranks the Monster Tamer has in the relevant skill:

\begin{itemize*}
\item \textbf{3 ranks - Identify Monster:} A Monster Tamer can automatically identify the name, type, and subtype of any Monster encountered.
\item \textbf{6 ranks - Full Monster Entry:} A Monster Tamer's player can open the Monster Manual (or other relevant source material) to the appropriate page and read the Monster's entry. If the Monster Tamer's player chooses, she may read the relatively uninformative descriptive text at the beginning of the entry to other players out loud. In addition, a Monster Tamer may note whether a Monster encounterred in the wild has extra advancement hit dice and/or caster level - though not necessarily what kind or how many.
\item \textbf{12 ranks - Fully Identify Monster:} The Monster Tamer is able to instantly identify any Monster's advancement hit dice and caster level (if any).
\end{itemize*}

\textbf{Wild Empathy (Ex):} A  Monster Tamer can improve the attitude of a Monster. This ability functions just like a \linkskill{Diplomacy} check to improve the attitude of a person. The Monster Tamer rolls 1d20 and adds her Monster Tamer level and his Charisma bonus to determine the wild empathy check result. The starting attitude of varies deoending on Monster.

To use wild empathy, the Monster Tamer and the Monster must be able to study each other, which means that they must be within 30 feet of one another under normal visibility conditions. Generally, influencing a Monster in this way takes 1 minute, but, as with influencing people, it might take more or less time.

\textbf{Craft Soul Prison (Sp):} A 2nd level Monster Tamer can craft Soul Prisons. A Soul Prison costs 25 GP to make using \linkskill{Craft} Gemcutting at DC 10.

A Soul Prison acts as a thrown weapon, which is used as a ranged touch attack with arange increment of 15'.

If a Soul Prison thrown by a Monster Tamer hits a Monster it inflicts 1 point of subdual damage per caster level, if the Monster is unconscious after being hit by the Soul Prison it is sucked into the Soul Prison and now belongs to the Monster Tamer who threw the Soul Prison, the Soul Prison is now sitting in a square formerly occupied by the captured Monster.

If a Soul Prison hits a Monster it becames attuned to that Monster and cannot be used on any other Monster.

\textbf{Heal Monster (Sp):} A Monster Tamer may attempt to accelerate the healing of a Monster in its Soul Prison. By spending a full round action, a Monster Tamer can attempt a \linkskill{Heal} Check (DC 15) to either convert all regular damage suffered by the Monster into subdual damage.

This ability may be used on each Monster 3 plus the Monster Tamer's Wisdom bonus times per day.

\textbf{Subtype Specialization (Ex):} A Monster Tamer can choose a subtype which is her specialty. A Monster Tamer gains a +1 bonus on
all Bluff, Handle Animal, Knowledge, Listen, Sense Motive, Spot, and Survival, checks when using these skills on
or about such creatures for every 3 caster levels. A Monster Tamer can choose a second Subtype to be equally proficient with at 12th level, and a third at 18th.

A Monster Tamer can Control one extra Monster which must be of a subtype that she specializes in. Subtypes include: Air, Aquatic, Chaotic, Cold, Earth, Electricity, Evil, Fire, Good, Lawful, Reptilian, and Water.

\textbf{Increased Awareness (Ex):} At 4th level and above, a Monster Tamer's Monster become more intelligent and aware. After the
Monster Tamer has owned her Monster for at least 1 week, its intelligence changes half of the Monster Tamers ranks in \linkskill{Handle Animal} if that is more than its normal intelligence.

In addition, a Monster Tamer can make her Monster gradually see things her way - a Monster's alignment shifts one degree towards the Monster Tamer's each week if she can succeed in an Wild Empathy roll at a DC of (10 + the Monster's CR). The DM decides whether it moves Law/Chaos or Good/Evil first depending upon circumstances.Monster's subtypes are unaffected, so an Evil Monster such as an Efreet would stay subtype [Evil] even if it subsequently became of Good alignment.

\textbf{Double Team:} Upon reaching 4th level, the Monster Tamer is able to control two Monsters out of their balls simultaneously, even in battle. This ability only functions so long as both Monsters are less than 2 CR of  Monster Tamer's caster level.

\textbf{Speak with Monsters (Ex):} At 5th level a Monster Tamer and her Monsters are under the effects of \linkspell{Tongues}, at all times, which onlly effects her and her Monsters.

\textbf{Craft Greater Soul Prison (Sp):} At 7th level a Monster Tamer can craft a Greater Soul Prison, which is a more powerful form of Soul Prison. A Greater Soul Prison costs 250 GP to make using \linkskill{Craft} Gemcutting at DC 15. Works just like Soul Prison but with d4 subdual damage per caster level.

\textbf{Type Specialization:} At 7th level, you can choose a single creature type to gain the same skill bonuses as your subtype specialization
with a creature type instead. You may choose a second type to Specialize in at 14th level, and a third at 19th.

You may have an additional controlled Monster, which must be of a type you are specialized in. Type and Subtype Specialization bonuses are cumulative.

\textbf{Transfer Control:} At 8th level a Monster Tamer can choose to change which Monster she controls, up to her regular limit of controlled Monsters. All newly controlled Monsters must be in Soul Prisons possessed and owned by the Monster Tamer. Transfer Control is a full-round action. Normally transferring control takes 10 minutes per Monster so transferred.

\textbf{Advanced Monster Healing (Sp):} A Monster Tamer can, at 9th level, use \linkspell{Heal} as a Spell like ability a number of times a day equal to her wisdom modifier, with a minimum of once a day. A Monster Tamer can only Heal Monsters she controls, but can heal them whether they are in their Soul Prisons or not.

\textbf{Craft Leaden Seal (Sp):} A Monster Tamer can craft a Leaden Seal. A Leaden Seal is a much more powerful form of Soul Prison. It costs 2500 GP to make using \linkskill{Craft} Gemcutting at DC 20. Works just like Soul Prison but with d8 subdual damage per caster level.

\textbf{Store Monster (Sp):} Starting at 11th level, as a move equivalent action, a Monster Tamer can send a Soul Prison with a Monster in it to a completely safe extra dimensional space. A Soul Prison must be within Close range (25 feet + 5 feet per 2 caster levels) to be stored. Store Monster cannot be combined with a normal move. Store Monster is a spell-like ability.

\textbf{Recall Monster (Sp):} Starting at 11th level, as a full round action, a Monster Tamer can transport a Stored Soul Prison from her extra dimensional space to her hand.

\textbf{Craft Master Prison (Sp):} A Master Prison is the ultimate expression of the Monster Hunte, It costs 25000 GP to make using \linkskill{Craft} Gemcutting at DC 25. and subdues the first Monster it hits, if that Monster does not have more hit dice than caster level of Monster Tamer. If a Monster is too strong to be captured automatically it may yet succumb as it still suffers d12
subdual damage per caster levels.

\textbf{Fast Recall Monster (Sp):} As Recall Monster, but Recalling Monster is a free action.

\textbf{Subtype Mastery:} The Monster Tamer chooses one subtype that she is already specialized in to Master. All her Leaden Seals function
like Master Prisons against Monsters of that subtype, and she can control one extra Monster of that subtype, in addition to her bonus controlled Monsters from type and subtype specialization.


