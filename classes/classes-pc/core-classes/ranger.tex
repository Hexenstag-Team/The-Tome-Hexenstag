%%%%%%%%%%%%%%%%%%%%%%%%%%%%%%%%%%%%%%%%%%%%%%%%%%
\classentry{Ranger}
%%%%%%%%%%%%%%%%%%%%%%%%%%%%%%%%%%%%%%%%%%%%%%%%%%

The forests are home to fierce and cunning creatures, such as bloodthirsty owlbears and malicious displacer beasts. But more cunning and powerful than these monsters is the Ranger, a skilled hunter and stalker. He knows the woods as if they were his home (as indeed they are), and he knows his prey in deadly detail.

\textbf{Alignment:} Any. Most are good, and such Rangers usually function as protectors of the wild areas. In this role, a Ranger seeks out and destroys or drives off evil creatures that threaten the wilderness. Good Rangers also protect those who travel through the wilderness, serving sometimes as guides and sometimes as unseen guardians. Most Rangers are also chaotic, preferring to follow the ebb and flow of nature or of their own hearts instead of rigid rules. Evil Rangers, though rare, are much to be feared. They revel in nature’s thoughtless cruelty and seek to emulate her most fearsome predators. They gain divine spells just as good Rangers do, for nature herself is indifferent to good and evil.

\textbf{Starting Gold:} 6d6x10 gp (210 gold).

\textbf{Starting Age:} Moderate

\textbf{Hit Die:} d8

\textbf{Class Skills:} The \currentclassname{}'s class skills (and the key ability for each skill) are \linkskill{Balance} (Dex), \linkskill{Climb} (Str), \linkskill{Concentration} (Con), \linkskill{Craft} (Int), \linkskill{Handle Animal} (Cha), \linkskill{Heal} (Wis), \linkskill{Hide} (Dex), \linkskill{Jump} (Str), \linkskill{Knowledge} (Int), \linkskill{Listen} (Wis), \linkskill{Move Silently} (Dex), \linkskill{Profession} (Wis), \linkskill{Ride} (Dex), \linkskill{Search} (Int), \linkskill{Sense Motive} (Wis), \linkskill{Speak Language} (n/a), \linkskill{Spot} (Wis), \linkskill{Survival} (Wis) \linkskill{Swim} (Str), \linkskill{Tumble} (Dex), and \linkskill{Use Rope} (Dex).

\textbf{Skills/Level:} 6 + Intelligence Bonus

\goodbab{}
\poorfor{}
\goodref{}
\poorwil{}

\begin{minorcastingclasstable}
\levelone{\multicolumn{1}{p{5cm}}{\raggedright{}1st Favored Enemy, \linkfeat{Track}, Wild Empathy}		& 2 & - & - & - & - & -& -}
\leveltwo{\multicolumn{1}{p{5cm}}{\raggedright{}Animal Companion, Skirmish (+1d6)}				& 3 & - & - & - & - & -& -}
\levelthree{\multicolumn{1}{p{5cm}}{\raggedright{}[Combat] Feat, Trackless Step}					& 3 & 0 & - & - & - & - & -}
\levelfour{\multicolumn{1}{p{5cm}}{\raggedright{}Skirmish (+1d6, +1 AC)}						& 3 & 1 & - & - & - & - & -}
\levelfive{\multicolumn{1}{p{5cm}}{\raggedright{}2nd Favored Enemy}							& 3 & 2 & 0 & - & - & - & -}
\levelsix{\multicolumn{1}{p{5cm}}{\raggedright{}[Combat] Feat, Skirmish (+2d6, +1 AC)}				& 3 & 2 & 1 & - & - & - & -}
\levelseven{\multicolumn{1}{p{5cm}}{\raggedright{}Evasion}								& 3 & 2 & 1 & - & - & - & -}
\leveleight{\multicolumn{1}{p{5cm}}{\raggedright{}Skirmish (+2d6, +2 AC)}						& 3 & 2 & 2 & 0 & - & - & -}
\levelnine{\multicolumn{1}{p{5cm}}{\raggedright{}Flawless Stride}							& 3 & 2 & 2 & 1 & - & - & -}
\levelten{\multicolumn{1}{p{5cm}}{\raggedright{}3rd Favored Enemy, [Combat] Feat, Skirmish (+3d6, +2 AC)}	& 3 & 2 & 2 & 1 & - & - & -}
\leveleleven{\multicolumn{1}{p{5cm}}{\raggedright{}Swift Tracker}							& 3 & 2 & 2 & 2 & 0 & - & -}
\leveltwelve{\multicolumn{1}{p{5cm}}{\raggedright{}Skirmish (+3d6, +3 AC)}						& 3 & 2 & 2 & 2 & 1 & - & -}
\levelthirteen{\multicolumn{1}{p{5cm}}{\raggedright{}Camouflage}							& 3 & 2 & 2 & 2 & 1 & - & -}
\levelfourteen{\multicolumn{1}{p{5cm}}{\raggedright{}[Combat] Feat, Skirmish (+4d6, +3 AC)}			& 3 & 2 & 2 & 2 & 2 & 0 & -}
\levelfifteen{\multicolumn{1}{p{5cm}}{\raggedright{}4th Favored Enemy}						& 3 & 2 & 2 & 2 & 2 & 1 & -}
\levelsixteen{\multicolumn{1}{p{5cm}}{\raggedright{}Skirmish (+4d6, +4 AC)}						& 3 & 2 & 2 & 2 & 2 & 1 & -}
\levelseventeen{\multicolumn{1}{p{5cm}}{\raggedright{}[Combat] Feat}						& 3 & 2 & 2 & 2 & 2 & 2 & 0}
\leveleighteen{\multicolumn{1}{p{5cm}}{\raggedright{}Skirmish (+5d6, +4 AC)}					& 3 & 2 & 2 & 2 & 2 & 2 & 1}
\levelnineteen{\multicolumn{1}{p{5cm}}{\raggedright{}Hide in Plain Sight}						& 3 & 2 & 2 & 2 & 2 & 2 & 2}
\leveltwenty{\multicolumn{1}{p{5cm}}{\raggedright{}5th Favored Enemy, Skirmish (+5d6, +5 AC)}			& 3 & 2 & 2 & 2 & 2 & 2 & 2}
\end{minorcastingclasstable}

\begin{wraptable}{O}{8.75cm}
\caption{\currentclassname{} Spells Known}
\rowcolors{1}{white}{offyellow}
\begin{tabular}{l*{7}{c}}
\textbf{Level} & \textbf{0th} & \textbf{1st} & \textbf{2nd} & \textbf{3rd} & \textbf{4th} & \textbf{5th} & \textbf{6th} \\
1st & 2 & - & - & - & - & - & -\\
2nd & 3 & - & - & - & - & - & -\\
3rd & 3 & 0 & - & - & - & - & -\\
4th & 3 & 1 & - & - & - & - & -\\
5th & 3 & 2 & 0 & - & - & - & -\\
6th & 3 & 2 & 1 & - & - & - & -\\
7th & 3 & 2 & 1 & - & - & - & -\\
8th & 3 & 2 & 2 & 0 & - & - & -\\
9th & 3 & 2 & 2 & 1 & - & - & -\\
10th & 3 & 2 & 2 & 1 & - & - & -\\
11th & 3 & 2 & 2 & 2 & 0 & - & -\\
12th & 3 & 2 & 2 & 2 & 1 & - & -\\
13th & 3 & 2 & 2 & 2 & 1 & - & -\\
14th & 3 & 2 & 2 & 2 & 2 & 0 & -\\
15th & 3 & 2 & 2 & 2 & 2 & 1 & -\\
16th & 3 & 2 & 2 & 2 & 2 & 1 & -\\
17th & 3 & 2 & 2 & 2 & 2 & 2 & 0\\
18th & 3 & 2 & 2 & 2 & 2 & 2 & 1\\
19th & 3 & 2 & 2 & 2 & 2 & 2 & 2\\
20th & 3 & 2 & 2 & 2 & 2 & 2 & 2\\
\end{tabular}
\end{wraptable}

\classfeatures

\textbf{Weapon and Armor Proficiency:} A Ranger is proficient with all simple and martial weapons, and with light armor and shields (except tower shields).

\textbf{Spells:} A Ranger gains the ability to cast a small number of divine spells, which are drawn from the Ranger spell list. A Ranger must choose and prepare his spells in advance (see below).

To prepare or cast a spell, a Ranger must have a Wisdom score equal to at least 10 + the spell level. The Difficulty Class for a saving throw against a Ranger's spell is 10 + the spell level + the Ranger's Wisdom modifier.

Like other spellcasters, a Ranger can cast only a certain number of spells of each spell level per day. His base daily spell allotment is given on Table: The Ranger. In addition, he receives bonus spells per day if he has a high Wisdom score.

If a Ranger multiclass with a Druid they'll use the Druid Spells Per Day with Ranger being half Druid level for spells; Caster level is both Druid and Ranger level. 

\begin{wraptable}{O}{8.75cm}
\caption{\currentclassname{} Favored Enemies}
\rowcolors{1}{white}{offyellow}
\begin{tabular}{l*{7}{c}}
\textbf{Type (Subtype)} & \textbf{Type (Subtype)}\\
Abberation & Humanoid (Reptilian)\\
Animal & Magical Beast\\
Construct & Monstrous Humanoid\\
Dragon & Ooze\\
Elemental & Outsider (Air)\\
Fey & Outsider (Chaotic)\\
Giant & Outisder (Earth)\\
Humanoid (Aquatic) & Outsider (Evil)\\
Humanoid (Dwarf) & Outsider (Fire)\\
Humanoid (Elf) & Outsider (Good)\\
Humanoid (Goblinoid) & Outsider (Lawful)\\
Humanoid (Gnoll) & Outsider (Native)\\
Humanoid (Gnome) & Outsider (Water)\\
Humanoid (Halfling) & Plant\\
Humanoid (Human) & Undead\\
Humanoid (Orc) & Vermin\\
\end{tabular}
\end{wraptable}

\textbf{Deity, Domain, and Domain Spells:} A Ranger's deity influences his alignment, what magic he can perform, his values, and how others see him. A Ranger chooses a domain from among those belonging to his deity. A Ranger can select an alignment domain (Chaos, Evil, Good, or Law) only if his alignment matches that domain.

Domain gives the Ranger access to a domain spell at each spell level he can cast, from 1st on up, as well as a granted power. In addition to the stated number of spells per day for 1st through 6th level spells, a Ranger gets a domain spell slot for each spell level, starting at 3rd. and gaining  new one at the 0 on Spells Per Day Table If a domain spell is not on the Rangers spell list, a Ranger can prepare it only in his domain spell slot.

\textbf{Favored Enemy (Ex):} At 1st level, a Ranger may select a type of creature from among those given on Table: Ranger Favored Enemies. The Ranger gains a +2 bonus on \linkskill{Bluff}, \linkskill{Listen}, \linkskill{Sense Motive}, \linkskill{Spot}, and \linkskill{Survival} checks when using these skills against creatures of this type. Likewise, he gets a +2 bonus on weapon damage rolls against such creatures.

At 5th level and every five levels thereafter (10th, 15th, and 20th level), the Ranger may select an additional favored enemy from those given on the table. In addition, at each such interval, the bonus against any one favored enemy (including the one just selected, if so desired) increases by 2. 

If the Ranger chooses humanoids or outsiders as a favored enemy, they must also choose an associated subtype, as indicated on the table. If a specific creature falls into more than one category of favored enemy, the Ranger's bonuses do not stack; they simply uses whichever bonus is higher.

\textbf{Track:} A Ranger gains \linkfeat{Track} as a bonus feat.

\textbf{Wild Empathy (Ex):} A Ranger can improve the attitude of an animal. This ability functions just like a \linkskill{Diplomacy} check to improve the attitude of a person. The Ranger rolls 1d20 and adds his Ranger level and his Charisma bonus to determine the wild empathy check result. The typical domestic animal has a starting attitude of indifferent, while wild animals are usually unfriendly.

To use wild empathy, the Ranger and the animal must be able to study each other, which means that they must be within 30 feet of one another under normal visibility conditions. Generally, influencing an animal in this way takes 1 minute, but, as with influencing people, it might take more or less time.

The Ranger can also use this ability to influence a magical beast with an Intelligence score of 1 or 2, but he takes a -4 penalty on the check.

\textbf{Skirmish (Ex):} A Ranger relies on mobility to deal extra damage and improve her defense. They deals an extra 1d6 points of damage on all attacks their makes during any round in which they moved at least 10 feet. The extra damage applies only to attacks taken during the Ranger’s turn. This extra damage increases by 1d6 for every four levels gained above 1st (2d6 at 5th, 3d6 at 9th, 4d6 at 13th, and 5d6 at 17th level). The extra damage only applies against living creatures that have a discernible anatomy. Undead, constructs, oozes, plants, incorporeal creatures, and creatures immune to extra damage from critical hits are not vulnerable to this additional damage. The Ranger must be able to see the target well eough to pick out a vital spot and must be able to reach such a spot. Rangers can apply this extra damage to ranged attacks made while skirmishing, but only if the target is within 30 feet. At 3rd level, a Ranger gains a +1 competence bonus to Armor Class during any round in which she moves at least 10 feet. The bonus applies as soon as the Ranger has moved 10 feet, and lasts until the start of her next turn.
This bonus improves by 1 for every four levels gained above 3rd (+2 at 7th, +3 at 11th, +4 at 15th, and +5 at 19th level).

A Ranger loses this ability when wearing medium or heavy armor or when carrying a medium or heavy load. If they gains the skirmish ability from another class, the bonuses stack.

\textbf{Animal Companion (Ex):} At 2nd level, a Ranger gains an Animal from the table below. This animal is a loyal companion that accompanies the Ranger on her adventures as appropriate for its kind.

A 2nd-level Ranger’s companion is completely typical for its kind except as noted on the Ranger Animal Companion table below. As a Ranger advances in level, the animal’s power increases,

If a Ranger releases her companion from service, she may gain a new one by performing a ceremony requiring 24 uninterrupted hours of prayer. This ceremony can also replace an animal companion that has perished.

A Ranger of 8th level or higher may select from alternative lists of animals (see the below). Should she select an animal companion from one of these alternative lists, the creature gains abilities as if the character’s Ranger level were lower than it actually is. Subtract the value indicated in the appropriate list header from the character’s Ranger level and compare the result with the Ranger level entry on the table in the sidebar to determine the animal companion’s powers. (If this adjustment would reduce the Ranger’s effective level to 0 or lower, she can’t have that animal as a companion.) 

\textbf{Trackless Step:} At level 3 a Ranger leaves no trail in natural surroundings and cannot be tracked. She may choose to leave a trail if so desired.

\textbf{Evasion:} At 7th level, a Ranger can avoid even magical and unusual attacks with great agility. If he makes a successful Reflex saving throw against an attack that normally deals half damage on a successful save (such as a red dragon’s fiery breath or a fireball), he instead takes no damage. Evasion can be used only if the Ranger is wearing light armor or no armor. A helpless Ranger (such as one who is unconscious or paralysed) does not gain the benefit of evasion.

\textbf{Flawless Stride:} At level 9 a Ranger can move through any sort of terrain that slows movement (such as undergrowth, rubble, and similar terrain) at her normal speed and without taking damage or suffering any other impairment.
This ability does not let her move more quickly through terrain that requires a Climb or Swim check to navigate, nor can she move more quickly through terrain or undergrowth that has been magically manipulated to impede motion.
A Ranger loses this benefit when wearing medium or heavy armor or when carrying a medium or heavy load.

\textbf{Swift Tracker:} At level 11 a Ranger can move at his normal speed while following tracks without taking the normal –5 penalty. He takes only a –10 penalty (instead of the normal –20) when moving at up to twice normal speed while tracking.

\textbf{Camouflage:}At level 13, a Ranger while in any sort of natural terrain they can hide in Bright Light with only a -5

\textbf{Hide in Plain Sight:} At level 19, a Ranger while in any sort of natural terrain they can hide in Bright Light without the -10.

\textbf{Ex-Rangers:} A Ranger who grossly violates the code of conduct required by their god loses all their spells, and must atones or seek a new god to regain their spells (see the \linkspell{Atonement} spell description).

%%%%%%%%%%%%%%%%%%%%%%%%%
\subsubsection{Ranger's Animal Companions}
%%%%%%%%%%%%%%%%%%%%%%%%%

\begin{smallbasictable}{Ranger Animal Companions}{l l l}
\textbf{2nd Level} & \textbf{7th Level (Level-6)} & \textbf{13th Level (Level-12)}\\
Badger&Ape&Ankylosaurus, Cave\\
Bat&Axebeak&Ape, Dire\\
Brixashulty\textsuperscript{1}&Badger, Dire&Bear, Brown\\
Camel&Bat, Dire&Boar, Dire\\
Caribou&Bear, Black&Crocodile, Giant\\
Chordevoc\textsuperscript{1}&Bison&Deinonychus\\
Chuckwalla (treat as Lizard)&Boar&Eagle, Dire\\
Climbdog&Branta&Hawk, Dire\textsuperscript{2}\\
Coyote (treat as Dog)&Brixashulty\textsuperscript{1}&Lion\\
Dog&Cheetah&Megaloceros\\
Dog, Riding&Chordevoc\textsuperscript{1}&Peccary, Dire (treat as Boar, Dire)\\
Donkey&Crocodile&Protoceratops\\
Eagle&Fleshraker&Rhinoceros\\
Gyrfalcon (treat as Hawk)&Hawk, Dire\textsuperscript{2}&Snake, Huge viper\\
Hawk&Jackal, Dire&Terror Bird\\
Horned Lizard&Leopard&Tiger\\
Horse, Heavy&Lizard, Monitor&Wolf, Dire\\
Horse, Light&Peccary (treat as Boar)&Wolverine, Dire\\
Hyena&Puma (treat as Leopard)&\cellcolor{white}\textsuperscript{2}Non-Air Subtype Rangers\\
Jackal&Snake, Constrictor&\cellcolor{white}\\
Moon Owl (treat as Owl)&Snake, Large viper&\cellcolor{white}\\
Owl&Snow Leopard (treat as Leopard)&\cellcolor{white}\\
Pony&Toad, Dire&\cellcolor{white}\\
Rat, Dire&Weasel, Dire&\cellcolor{white}\\
Raven&Wolverine&\cellcolor{white}\\
Serval&\cellcolor{white}\textsuperscript{1}Non-Small Rangers&\cellcolor{white}\\
Snake, Medium Viper&\cellcolor{white}\textsuperscript{2}Air Subtype Rangers Only&\cellcolor{white}\\
Snake, Small Viper&\cellcolor{white}\cellcolor{white} &\cellcolor{white}\\
Snowy Owl (treat as Owl)&\cellcolor{white}&\cellcolor{white}\\
Swindlespitter&\cellcolor{white}&\cellcolor{white}\\
Tressym&\cellcolor{white}&\cellcolor{white}\\
Vulture&\cellcolor{white}&\cellcolor{white}\\
Wolf&\cellcolor{white}&\cellcolor{white}\\
\textsuperscript{1}Small Rangers Only&\cellcolor{white}&\cellcolor{white}\\
\end{smallbasictable}

\pagebreak

\begin{smallbasictable}{Ranger's Animal Companion}{l l l l l l}
\multicolumn{1}{p{2cm}}{\raggedright{} \textbf{Ranger's Caster Level}} & \multicolumn{1}{p{2cm}}{\raggedright{} \textbf{Bonus HD}} & \multicolumn{1}{p{2cm}}{\raggedright{} \textbf{Natural Armor Adj.}} &  \multicolumn{1}{p{2cm}}{\raggedright{}\textbf{Str/Dex Adj.}} & \multicolumn{1}{p{2cm}}{\raggedright{} \textbf{Bonus Tricks}} & \textbf{Special}\\
2-4 & +0 & +0 & +0 & 1 & Link, Share Spells\\
5-10 & +2 & +2 & +1 & 2 & Evasion\\
9-16 & +4 & +4 & +2 & 3 & Devotion\\
17-20 & +6 & +6 & +3 & 4 & Multiattack\\
\end{smallbasictable}

\textbf{Animal Companion Basics:} Use the base statistics for a creature of the companion’s kind, as given in the Monster Manual, but make the following changes.

\textit{Ranger's Caster Level:} The Ranger's Caster Level is used for determining the companion’s abilities and the alternative lists available to the character; If multiclassed with Druid use Druid's Animal Companions.

\textit{Bonus HD:} Extra eight-sided (d8) Hit Dice, each of which gains a Constitution modifier, as normal. Remember that extra Hit Dice improve the animal companion’s base attack and base save bonuses. An animal companion’s base attack bonus is the same as that of a Ranger of a level equal to the animal’s HD. An animal companion has good Fortitude and Reflex saves (treat it as a character whose level equals the animal’s HD). An animal companion gains additional skill points and feats for bonus HD as normal for advancing a monster’s Hit Dice (see the Monster Manual).

\textit{Natural Armor Adj.:} The number noted here is an improvement to the animal companion’s existing natural armor bonus.

\textit{Str/Dex Adj.:} Add this value to the animal companion’s Strength and Dexterity scores.

\textit{Bonus Tricks:} The value given in this column is the total number of “bonus” tricks that the animal knows in addition to any that the Ranger might choose to teach it (see the \linkskill{Handle Animal}). These bonus tricks don’t require any training time or \linkskill{Handle Animal} checks, and they don’t count against the normal limit of tricks known by the animal. The Ranger selects these bonus tricks, and once selected, they can’t be changed.

\textit{Link (Ex):} A Ranger can handle her animal companion as a free action, or push it as a move action, even if she doesn’t have any ranks in the Handle Animal skill. The Ranger gains a +4 circumstance bonus on all wild empathy checks and \linkskill{Handle Animal} checks made regarding an animal companion.

\textit{Share Spells (Ex):} At the Ranger’s option, she may have any spell (but not any spell-like ability) she casts upon herself also affect her animal companion. The animal companion must be within 5 feet of her at the time of casting to receive the benefit. If the spell or effect has a duration other than instantaneous, it stops affecting the animal companion if the companion moves farther than 5 feet away and will not affect the animal again, even if it returns to the Ranger before the duration expires. Additionally, the Ranger may cast a spell with a target of “You” on her animal companion (as a touch range spell) instead of on herself. A Ranger and her animal companion can share spells even if the spells normally do not affect creatures of the companion’s type (animal).

\textit{Evasion (Ex):} If an animal companion is subjected to an attack that normally allows a Reflex saving throw for half damage, it takes no damage if it makes a successful saving throw.

\textit{Devotion (Ex):} An animal companion’s devotion to its master is so complete that it gains a +4 morale bonus on Will saves against enchantment spells and effects.

\textit{Multiattack:} An animal companion gains Multiattack as a bonus feat if it has three or more natural attacks (see the Monster Manual for details on this feat) and does not already have that feat. If it does not have the requisite three or more natural attacks, the animal companion instead gains a second attack with its primary natural weapon, albeit at a –5 penalty.

\pagebreak