%%%%%%%%%%%%%%%%%%%%%%%%%%%%%%%%%%%%%%%%%%%%%%%%%%
\classentry{Paladin}
%%%%%%%%%%%%%%%%%%%%%%%%%%%%%%%%%%%%%%%%%%%%%%%%%%

A Paladin believes in a set of duties based on a universal moral law. Whether Paladins are right about this is a matter of dispute among philosophers to this day, but wise men don't raise the topic to a Paladin's face. Despite this ideological inflexibility, most Paladins are pretty nice people: they keep their promises, help the needy, and work to make the world a better place to live.

\textbf{Alignment:} Paladins must be Good; Many Paladins are also Lawful because of the strictness of their philosophy, but there are Chaotic Paladins who also uphold the Code of Conduct.

\textbf{Starting Gold:} 6d6x10 gp (210 gold).

\textbf{Starting Age:} Moderate

\textbf{Hit Die:} d10

\textbf{Class Skills:} The \currentclassname{}'s class skills (and the key ability for each skill) are \linkskill{Climb} (Str), \linkskill{Concentration} (Con), \linkskill{Craft} (Int), \linkskill{Diplomacy} (Cha), \linkskill{Handle Animal} (Cha), \linkskill{Heal} (Cha), \linkskill{Intimidate} (Cha), \linkskill{Jump} (Str), \linkskill{Knowledge} (Int), \linkskill{Listen} (Wis), \linkskill{Profession} (Wis), \linkskill{Ride} (Dex), \linkskill{Sense Motive} (Wis), \linkskill{Spellcraft} (Int), \linkskill{Spot} (Wis), and \linkskill{Swim} (Str).

\textbf{Skills/Level:} 4 + Intelligence Bonus

\goodbab{}
\goodfor{}
\poorref{}
\goodwil{}

\begin{minorcastingclasstable}
\levelone{\multicolumn{1}{p{5cm}}{\raggedright{}Aura of good, Code of Conduct, \linkfeat{Mounted Combat}, Turn Undead}		& 2 & - & - & - & - & -& -}
\leveltwo{\multicolumn{1}{p{5cm}}{\raggedright{}Divine grace, Smite Evil +1d6}								& 3 & - & - & - & - & -& -}
\levelthree{\multicolumn{1}{p{5cm}}{\raggedright{}Aura of courage, Divine Health, Divine Inspiration}					& 3 & 0 & - & - & - & - & -}
\levelfour{\multicolumn{1}{p{5cm}}{\raggedright{}Smite Evil +2d6}										& 3 & 1 & - & - & - & - & -}
\levelfive{\multicolumn{1}{p{5cm}}{\raggedright{}Aura of Defiance, Special Mount}								& 3 & 2 & 0 & - & - & - & -}
\levelsix{\multicolumn{1}{p{5cm}}{\raggedright{}Divine Inspiration, Smite Evil +3d6}								& 3 & 2 & 1 & - & - & - & -}
\levelseven{\multicolumn{1}{p{5cm}}{\raggedright{}Aura of Resolve}										& 3 & 2 & 1 & - & - & - & -}
\leveleight{\multicolumn{1}{p{5cm}}{\raggedright{}Smite Evil +4d6, Mettle}									& 3 & 2 & 2 & 0 & - & - & -}
\levelnine{\multicolumn{1}{p{5cm}}{\raggedright{}Aura of Health, Divine Inspiration, Fast Casting (0 and 1st)}				& 3 & 2 & 2 & 1 & - & - & -}
\levelten{\multicolumn{1}{p{5cm}}{\raggedright{}Smite Evil +5d6}										& 3 & 2 & 2 & 1 & - & - & -}
\leveleleven{\multicolumn{1}{p{5cm}}{\raggedright{}Aura of Life, Fast Casting (2nd)}								& 3 & 2 & 2 & 2 & 0 & - & -}
\leveltwelve{\multicolumn{1}{p{5cm}}{\raggedright{}Divine Inspiration, Smite Evil +6d6}							& 3 & 2 & 2 & 2 & 1 & - & -}
\levelthirteen{\multicolumn{1}{p{5cm}}{\raggedright{}Fast Casting (3rd)}									& 3 & 2 & 2 & 2 & 1 & - & -}
\levelfourteen{\multicolumn{1}{p{5cm}}{\raggedright{}Aura of Freedom, Smite Evil +7d6}							& 3 & 2 & 2 & 2 & 2 & 0 & -}
\levelfifteen{\multicolumn{1}{p{5cm}}{\raggedright{}Divine Inspiration, Fast Casting (4th)}							& 3 & 2 & 2 & 2 & 2 & 1 & -}
\levelsixteen{\multicolumn{1}{p{5cm}}{\raggedright{}Smite Evil +8d6}										& 3 & 2 & 2 & 2 & 2 & 1 & -}
\levelseventeen{\multicolumn{1}{p{5cm}}{\raggedright{}Aura of Truth, Fast Casting (5th)}							& 3 & 2 & 2 & 2 & 2 & 2 & 0}
\leveleighteen{\multicolumn{1}{p{5cm}}{\raggedright{}Divine Inspiration, Smite Evil +9d6}							& 3 & 2 & 2 & 2 & 2 & 2 & 1}
\levelnineteen{\multicolumn{1}{p{5cm}}{\raggedright{}Aura of Forbiddance, Fast Casting (6th)}						& 3 & 2 & 2 & 2 & 2 & 2 & 2}
\leveltwenty{\multicolumn{1}{p{5cm}}{\raggedright{}Miracle, Smite Evil +10d6}								& 3 & 2 & 2 & 2 & 2 & 2 & 2}
\end{minorcastingclasstable}

\classfeatures

\textbf{Weapon and Armor Proficiency:} Paladins are proficient with all simple and martial weapons, with light, medium, and heavy armor, and with shields and great shields.

\textbf{Aura of Good (Ex):} The power of a paladin's aura of good is equal to their paladin level.

\textbf{Code of Conduct:} Paladins follow a strict moral philosophy with two key features: only intentions, not consequences, matter; and any moral obligation applies to all rational creatures, at all times. Paladins derive their duties from considering if they would want everyone to act as they do all the time. If they would, that's a universal duty; if not, that action is forbidden. Paladins respect the autonomy and free wills of rational creatures, and treat all rational creatures, even evil ones, as ends in themselves, rather than as means to an end. Paladins imagine a hypothetical kingdom where all rational creatures both follow and create the laws, and choose laws based on what would make this kingdom a better place to live. Of course, since they believe that everyone should act this way too, they aren't much fun at cocktail parties.

While adventuring, paladins encounter a lot of creatures with alien mentalities and biologies. Most paladins consider monsters
with incomprehensible or inflexible viewpoints outside the community of rational creatures, even if they're otherwise intelligent, so feel no obligation to treat them any differently from animals. Similarly, most paladins feel that creatures obligated to kill other rational creatures for sustenance, such as vampires and illithids, should voluntarily commit suicide, and that if they fail in this duty, other rational creatures may "assist" them in its performance.

The list that follows is some examples of how paladins interpret their code in practice. However, it's not exhaustive.
\begin{itemize*}
\item Paladins may not lie or deceive. If no one told the truth, language would be useless.
\item Paladins must fulfill their promises. If everyone broke promises, there would be no point in making them.
\item Paladins may not use mind-affecting effects. Mind-affecting abilities destroy the autonomy of rational creatures.
\item Paladins must slay evil when possible. Remember that not only does an afterlife exist, but one can go there and visit it to watch Evil souls receive rewards from Evil deities. Since killing Evil people makes both them and everyone else better off (because they're no longer around in life to do Evil), it is in fact a moral duty to send successful Evil to the afterlife expediently.
\item Paladins may not steal. If everyone stole, the concept of property would be meaningless. This doesn't, however, prevent paladins from killing evil creatures, then taking their stuff.
\item Paladins must offer reasonable aid to those needier than themselves. If no one gave appropriate charity, than they could not expect any assistance when they needed it.
\item Paladins must seek to develop their talents; thus, most paladins are adventurers. If no one cultivated their abilities, the world would be a poorer place.
\item Paladins must not treat animals and similar creatures with cruelty. Deliberate cruelty deadens the feeling of compassion that promotes moral behavior towards rational creatures.
\end{itemize*}

\textbf{Mounted Combat:} A paladin gains \linkfeat{Mounted Combat} as a bonus feat at first level. If they already has \linkfeat{Mounted Combat}, they may gain any [Combat] feat they meets the prerequisites for instead.

\textbf{Spells} A paladin casts divine spells which are drawn from the paladin spell list. When a paladin gain access to a new level of spells, they automatically knows all the spells for that level on the paladin spell list; essentially, her spell list is the same as her spells known list. In addition, her divine inspiration class feature (see below) allows her to add a small number of spells to her spell list. they can cast any spell they knows without preparing it ahead of time.

To cast a paladin spell, a paladin must have a Charisma score of 10 + the spell's level. The Difficulty Class of the saving throw against a paladin's spells are 10 + the spell's level + her Cha modifier. A paladin has the same number of spells per day as a bard, except that they never gains more than three spell slots per level. they gains bonus spells for a high Charisma.

A paladin's caster level is equal to her class level.

If a Paladin multiclass with a Cleric they'll use the Cleric Spells Per Day with Paladin being half Cleric level for spells; Caster level is both Paladin and Cleric level. 

\textbf{Deity, Domain, and Domain Spells:} A Paladin's deity influences his alignment, what magic he can perform, his values, and how others see him. A Paladin chooses a domain from among those belonging to his deity. A Paladin can select an alignment domain (Chaos, Good, or Law) only if his alignment matches that domain.

Domain gives the Paladin access to a domain spell at each spell level he can cast, from 1st on up, as well as a granted power. In addition to the stated number of spells per day for 1st through 6th level spells, a Paladin gets a domain spell slot for each spell level, starting at 3rd. and gaining  new one at the 0 on Spells Per Day Table If a domain spell is not on the Paladin's spell list, a Paladin can prepare it only in his domain spell slot.

\setlength{\intextsep}{-5pt}
\begin{wraptable}{O}{8.75cm}
\caption{\currentclassname{} Spells Known}
\rowcolors{1}{white}{offyellow}
\begin{tabular}{l*{7}{c}}
\textbf{Level} & \textbf{0th} & \textbf{1st} & \textbf{2nd} & \textbf{3rd} & \textbf{4th} & \textbf{5th} & \textbf{6th} \\
1st & 2 & - & - & - & - & - & -\\
2nd & 3 & - & - & - & - & - & -\\
3rd & 3 & 0 & - & - & - & - & -\\
4th & 3 & 1 & - & - & - & - & -\\
5th & 3 & 2 & 0 & - & - & - & -\\
6th & 3 & 2 & 1 & - & - & - & -\\
7th & 3 & 2 & 1 & - & - & - & -\\
8th & 3 & 2 & 2 & 0 & - & - & -\\
9th & 3 & 2 & 2 & 1 & - & - & -\\
10th & 3 & 2 & 2 & 1 & - & - & -\\
11th & 3 & 2 & 2 & 2 & 0 & - & -\\
12th & 3 & 2 & 2 & 2 & 1 & - & -\\
13th & 3 & 2 & 2 & 2 & 1 & - & -\\
14th & 3 & 2 & 2 & 2 & 2 & 0 & -\\
15th & 3 & 2 & 2 & 2 & 2 & 1 & -\\
16th & 3 & 2 & 2 & 2 & 2 & 1 & -\\
17th & 3 & 2 & 2 & 2 & 2 & 2 & 0\\
18th & 3 & 2 & 2 & 2 & 2 & 2 & 1\\
19th & 3 & 2 & 2 & 2 & 2 & 2 & 2\\
20th & 3 & 2 & 2 & 2 & 2 & 2 & 2\\
\end{tabular}
\end{wraptable}

\textbf{Turn Undead (Su):} A paladin can turn undead as a cleric. They may use this ability a number of times per day equal to 3 + her Charisma modifier.

\textbf{Divine Grace (Su):} Beginning at 2nd level, a paladin gains a sacred bonus to her saving throws equal to her Charisma bonus (if any).

\textbf{Smite Evil (Su):} Starting at 2nd level, a paladin deals 1d6 extra damage to any evil creatures they attacks in melee. These dice are not multiplied by damage multipliers, and are not applied to any bonus attacks beyond those granted by Base Attack Bonus. This damage increases by +1d6 every two levels thereafter.

\textbf{Aura of Courage (Su):} Beginning at 3rd level, a paladin is immune to fear (magical or otherwise). Each ally within 60 feet of her gains a +4 morale bonus on saving throws against fear effects.

This ability functions while the paladin is conscious, but not if they is unconscious or dead.

\textbf{Divine Health (Su):} Starting at 3rd level, a paladin gains immunity to all diseases, including supernatural and magical diseases.

\textbf{Divine Inspiration:} At 3rd level, a paladin may add a single cleric spell from the abjuration, conjuration (healing), evocation, or transmutation schools, or a single spell with the [Good] descriptor, to her spell list. This spell must be of a level equal to or lower than the highest they can cast, and must not have the death, evil, or mind-affecting descriptors. they may add another spell to her spell list, subject to the same restrictions, at 6th, 9th, 12th, 15th, and 18th levels.

\textbf{Aura of Defiance (Su):} Beginning at 5th level, a paladin gains a second aura, which grants her immunity to charm effects. Each ally within 60 feet of her gains a +4 morale bonus on saving throws against charm effects.

While her aura of good is always active, they may only have one other aura active at any time. They may change which aura they has active as a swift action. All auras (except her aura of good) function only when they is conscious

\textbf{Special Mount:} Starting at 5th level, a paladin gains the services of a level-appropriate cohort whom they rides into battle. This cohort does not gain XP and is always two levels lower than the paladin's character level. Typical choices include good-aligned dragons and magical beasts or quadruped outsiders with the [Good] subtype, but almost anything works so long as it's rational and willing to follow the paladin's code, and they can ride it.

The paladin and her mount share an empathic link out to a distance of up to 1 mile. Because of this empathic link, the paladin has the same connection to an item or place that her mount does, just as with a master and his familiar.

At the paladin's option, they may have any spell (but not any spell-like ability) they casts on herself also affect her mount. The mount must be within 5 feet at the time of casting to receive the benefit. If the spell or effect has a duration other than instantaneous, it stops affecting the mount if it moves farther than 5 feet away and will not affect the mount again even if it returns to the paladin before the duration expires. Additionally, the paladin may cast a spell with a target of "You" on her mount (as a touch range spell) instead of on herself.

\textbf{Aura of Resolve (Su):} Beginning at 7th level, a paladin gains the aura of resolve, which grants her immunity to compulsion effects. Each ally within 60 feet of her gains a +4 morale bonus on saving throws against compulsion effects.

\textbf{Mettle (Ex):} Starting at 8th level, a paladin who succeeds on a Fortitude partial or Willpower partial save takes no effect, as if she had immunity.

\textbf{Aura of Health (Su):} Beginning at 9th level, a paladin gains the aura of health, which grants her immunity to all poisons. Each ally within 60 feet of her gains a +4 morale bonus on saving throws against poison.

\textbf{Fast Casting:} Starting at 9th level, a paladin can cast any paladin orison or 1st-level paladin spell as a swift action, as if she had applied the Quicken metamagic feat to the spell; if she wishes to add additional metamagic, it doesn't increase the casting time beyond a swift action. Every two levels thereafter, the level of spells she can cast as a swift action increases by one.

\textbf{Aura of Life (Su):} Beginning at 11th level, a paladin gains the aura of life, which grants her immunity to death effects and negative energy. Each ally within 60 feet of her gains a +4 morale bonus on saving throws against death effects and negative energy effects.

\textbf{Aura of Freedom (Su):} Starting at 14th level, a paladin gains the aura of freedom, which grants her the effects of freedom of movement continuously. Each ally within 60 feet of her gains a +4 morale bonus on saving throws against effects that freedom of movement would prevent.

\textbf{Aura of Truth (Su):} Beginning at 17th level, a paladin gains the aura of truth, which grants her the effects of true seeing continuously. Each ally within 60 feet of her gains a +4 morale bonus on saving throws against illusions

\textbf{Aura of Forbiddance (Su):}  Starting at 19th level, a paladin gains the aura of forbiddance, which allows her to prevent any planar travel effect with 60 feet of her from occurring, at her option. No save is offered against this effect.

\textbf{Miracle (Sp):} Beginning at 20th level, a paladin may use miracle as spell-like ability once per day.

\textbf{Ex-Paladins:} A paladin who ceases to be good, who willfully commits an evil act, or who grossly violates the code of conduct loses all paladin spells and abilities (including the service of the paladin's mount, but not weapon, armor, and shield proficiencies). She may not progress any farther in levels as a paladin. She regains her abilities and advancement potential if she atones for her violations (see the \linkspell{atonement} spell description), as appropriate.