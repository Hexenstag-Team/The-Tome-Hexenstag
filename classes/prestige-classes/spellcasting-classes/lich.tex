%%%%%%%%%%%%%%%%%%%%%%%%%%%%%%%%%%%%%%%%%%%%%%%%%%
\classentry{Lich}
%%%%%%%%%%%%%%%%%%%%%%%%%%%%%%%%%%%%%%%%%%%%%%%%%%

Feared by mortal beings for their malign magic, their intelligence, and their willingness to embrace Undeath for a chance to live forever, Liches are evil beings of great power.

A Lich is an undead Spellcaster, usually a Wizard or Sorcerer but sometimes a Cleric or other Spellcaster, who has used its magical powers to unnaturally extend its life.

A Lich is a gaunt and skeletal humanoid with withered flesh stretched tight across horribly visible bones. Its eyes have long ago been lost to decay, but a bright pinpoints of light burn in the empty eye sockets with the colour of their primary School of Magic, often green for Necromancy (List below).

\textbf{Prerequisites:} 
\begin{description*}
\item[\hspace*{1.5cm}Spells:] Must be able to cast 5th level spells including \linkspell{Magic Jar}.
\item[\hspace*{1.5cm}Skills:] \linkskill{Knowledge} (Arcana) 10 ranks, \linkskill{Knowledge} (Undead) 10 ranks
\end{description*}

\textbf{Alignment:} Usually Evil.

\textbf{Hit Die:} d12

\textbf{Class Skills:} The \currentclassname{}'s class skills (and the key ability for each skill) are \linkskill{Hide} (Dex), \linkskill{Knowledge} (Int), \linkskill{Listen} (Wis), \linkskill{Move Silently} (Dex), \linkskill{Search} (Int), \linkskill{Sense Motive} (Wis), and \linkskill{Spellcraft} (Int).

\textbf{Skills/Level:} 2 + Intelligence Bonus

\poorbab{}
\poorfor{}
\poorref{}
\goodwil{}

\begin{extraclasstable}{\textbf{Spellcasting}}
\levelone{\multicolumn{1}{p{6.5cm}}{\raggedright Int +2, Natural Armor +2, Damaging Touch 1d6+5, Paralyzing Touch (1d4 rounds), Phylactery (1st Stage), Resistances (Cold 5, Electricity 5)} & +1 Spellcaster level}
\leveltwo{\multicolumn{1}{p{6.5cm}}{\raggedright Natural Armor +3, Damage Reduction 5/bludgeoning and magic, Damaging Touch (1d8+5), Fear Aura (10-ft. radius), Fortification (Light), Paralyzing Touch (1d4 minutes), Phylactery (2nd stage), Resistances (Cold 10, Electricity 10)} & +1 Spellcaster level}
\levelthree{\multicolumn{1}{p{6.5cm}}{\raggedright Wis +2, Natural Armor +4, Damage Reduction 10/bludgeoning and magic, Fear Aura (30-ft. radius), Fortification (Moderate), Paralyzing Touch (1d4 hours), Phylactery (3rd stage), Resistances (Cold 20, Electricity 20)} & +1 Spellcaster level}
\levelfour{\multicolumn{1}{p{6.5cm}}{\raggedright Cha +2, Natural Armor +5, Damage Reduction 15/bludgeoning and magic, Fear Aura (60-ft. radius), Immunities (Cold, Electricity, Polymorph), Paralyzing Touch (Permanent), Phylactery (4th stage), Turn Resistance +4, Undeath} & +1 Spellcaster level}
\end{extraclasstable}

\classfeatures

\textbf{Weapon and Armor Proficiency:} The Lich gains no proficiency with armor or weapons.

\textbf{Spellcasting:} Every level, the Lich casts spells (including gaining any new spell slots and spell knowledge) as if she had also gained a level in any spellcasting class she had previous to gaining that level.

\textbf{Ability Score Changes:} The indicated ability score increases or decreases by the amount noted. These changes are cumulative. At first, a would-be Lich gains an intellectual understanding of the process necessary to become a Lich (represented by the Intelligence increase at 1st level). Then she gains the intuitive understanding of what the process entails (represented by the Wisdom increase at 3rd level). Finally, she gains a surge of confidence as the process is completed (represented by the Charisma increase at 4th level).

\textbf{Natural Armor Replacement:} At each level of Lich, the character uses either her existing Natural Armor bonus or the Lich's value given above, whichever is greater.

\textbf{Damaging Touch (Su):} At 1st level, a Lich's touch deals 1d6+5 points of negative energy damage to any living target. A Will save (DC 10 + 1/2 Lich's HD + Lich's Cha modifier) halves this damage. A Lich who attacks with a natural weapon may deal this damage in addition to the normal damage for that attack. When the Lich reaches 2nd level, the damage for her touch increases to 1d8+5 points.

\textbf{Paralyzing Touch (Su):} Any living creature hit by a Lich's touch attack must succeed on a Fortitude save (DC 10 + 1/2 Lich's HD + Lich's Cha modifier) or be paralyzed for 1d4 rounds. A Lich who attacks with a natural weapon may deal this damage in addition to the normal damage for that attack. Remove paralysis or any effect that can remove a curse frees the victim, but the effect cannot be dispelled. Anyone paralyzed by a Lich seems dead, though a successful DC 20 Spot check or DC 15 Heal check reveals that the victim is still alive.

At 2nd level, the duration of this effect increases to 1d4 minutes. At 3rd level, the duration increases to 1d4 hours. At 4th level, the effect is permanent until removed.

\textbf{Phylactery:} To complete her transformation to a Lich, the character must create a Phylactery using \linkspell{Magic Jar} on a gem or crystal and putting it in a object. The Phylactery is crafted in four stages, and the Lich transfers a bit more of her life force to it at each stage. It does not, however, grant her any of the normal benefits of a Phylactery until it is fully completed.

Paying the cost of each stage of its construction evey level. Taking levels of Lich requires her to pay 1/4 of her ''XP to level'' into her Phylactery evey level (Rounding up); This will never reduce your level.

For the purpose of determining item saving throws, the Phylactery has a caster level equal to that of the Lich at the time she completed the most recent stage of work.

The most common physical form for a Phylactery is a sealed metal box containing strips of parchment on which magical phrases have been transcribed. The box is a Tiny object with 40 hit points, hardness 20, and a break DC of 40. Other kinds of phylacteries can also exist, such as rings, amulets, or similar items.

Once the Phylactery has been completed, the Lich can avoid permanent destruction as long as her Phylactery survives. If she dies or is destroyed, she reappears 1d10 days after her old body's death. She gains her new physical form by grafting her undead spirit to a humanoid corpse, mindless undead, or some weak-minded creature within 4 miles of her Phylactery. The new body has all the abilities and powers of her old one, though any items she used to carry are lost (probably taken by those who slew her old body). Likewise, any spells or effects bound to her old body with permanency do not spontaneously appear on her new one. Most Liches who recover from death spend a year or more tracking down their items and learning more about their attackers, and it is not unusual for a Lich to wait decades before exacting her revenge.

If no body is available she's just a Shadow in both the Material Plane and the Shadow Planes at the same time, but is unable to interact with the Material Plane and does not inherit any abilities as a Shadow other than the spirit grafting Will Save (DC 10 + Lich's Wis Modifier), and must stay within 4 miles of her Phylactery.

If the Phylactery is destroyed while the Lich is still active in a body, her undead life force automatically joins that body. She takes no penalties of any kind for that joining, but without a Phylactery, she cannot recover if her body is subsequently destroyed. She may create a new Phylactery to replace a lost one if she has the time and resources to do so (1 Level of XP cost to make new one).

\textbf{Resistances (Ex):} At 1st level, a Lich gains resistance to Cold 5 and Electricity 5. Each of these Resistances increases to 10 at 2nd level and to 20 at 3rd level.

\textbf{Damage Reduction (Su):} At 2nd level, a Lich gains Damage Reduction 5/bludgeoning and magic. Her natural weapons are treated as magic weapons for the purpose of overcoming Damage Reduction. Her Damage Reduction increases to 10/bludgeoning and magic at 3rd level and to 15/bludgeoning and magic at 4th level.

\textbf{Fear Aura (Su):} Beginning at 2nd level, a Lich is shrouded in a dreadful aura of death and evil. Any creature in a 10-foot radius that looks at her must make a Will save if it is within 10 feet of her and has fewer than 5 HD. Failure means the creature is affected as though by a fear spell (caster level equals Lich's character level). A creature that successfully saves cannot be affected again by the same Lich's aura for 24 hours. The Lich can negate the effect of her aura by concealing all of her withered flesh and hiding her glowing eyes, usually by means of clothing that covers her entire body coupled with some sort of deep cowled cloak or visored helmet. This effect happens to all Liches not just Evil ones.

The radius of this effect expands to 30 feet at 3rd level and to 60 feet at 4th level.

\textbf{Fortification (Ex):} When a Lich attains 2nd level, her internal organs begin to shut down as she continues her metamorphosis into an undead creature and starts to transfer her life energy into her Phylactery. Thus, she is treated as if she had the light Fortification armor property (25\% chance for any critical hit or sneak attack against her to become a normal attack). If she has any version of the Fortification special ability from another source (such as a spell or magic item), use the better value. At 3rd level, she is treated as if she had moderate Fortification (50\% chance for any critical hit or sneak attack against her to become a normal attack).

\textbf{Turn Resistance (Ex):} At 4th level, a Lich gains +4 Turn Resistance. She is treated as an undead with 4 more Hit Dice than she actually has for the purpose of turn, rebuke, command, or bolster attempts.

\textbf{Immunities (Ex):} A 4th-level Lich is immune to Cold, Electricity, and Polymorph, though she can still use Polymorph effects on herself.

\textbf{Undeath:} Upon reaching 4th level, the Lich becomes fully undead. Her type changes to undead and gains the Dark Minded (subtype) and keeps all subtype of her living self.

A Lich Cleric becomes an ex-Cleric at this time if her deity does not allow undead/Lich Clerics. However, she can remedy this situation either by offering her allegiance to a god that does accept undead/Lich Clerics. A Lich Cleric who could previously turn undead loses that ability but gains the ability to rebuke undead. Likewise, a Lich Cleric who could previously spontaneously cast cure spells can now do so with inflict spells (Even it she a Good Cleric).

Upon becoming undead, a Lich Wizard is shunned by her familiar unless it is a bat or rat, but she can acquire a bat or rat familiar to replace her previous one in the usual way.				

\textbf{Colour of Magic}
\begin{itemize*}
	\item Abjuration: Blue
	\item Conjuration: Yellow
	\item Divination: Gray
	\item Enchantment: Pink
	\item Evocation: Red
	\item Illusion: Purple
	\item Necromancy: Green
	\item Transmutation: Orange
\end{itemize*}