%%%%%%%%%%%%%%%%%%%%%%%%%%%%%%%%%%%%%%%%%%%%%%%%%%
\classentry{Master Specialist}
%%%%%%%%%%%%%%%%%%%%%%%%%%%%%%%%%%%%%%%%%%%%%%%%%%

The path of the master specialist requires the kind of dedicated and studious mind that only a wizard can provide—other arcane casters simply don't have the capability to meet this class's needs. The best way to become a master specialist is to be a specialist wizard and take 1st level in the prestige class as your 4th character level. Then you can advance through all ten levels of the class and, after com- pleting it at 13th level, either return to wizard or move on to another prestige class such as archmage.

\textbf{Prerequisites:} 
\begin{description*}
\item[\hspace*{1.5cm}Skills:] \linkskill{Knowledge} (Arcana) 5 ranks, \linkskill{Spellcraft} 5 ranks.
\item[\hspace*{1.5cm}Feat:] \linkfeat{Spell Focus} (school of specialization).
\item[\hspace*{1.5cm}Spellcasting:] Must be able to cast 2nd-level arcane spells.
\end{description*}

\textbf{Alignment:} Any.

\textbf{Hit Die:} d4

\textbf{Class Skills:} The \currentclassname{}'s class skills are \linkskill{Concentration} (Con), \linkskill{Craft} (Int), \linkskill{Decipher Script} (Int), \linkskill{Knowledge} (Int), \linkskill{Profession} (Wis), \linkskill{Spellcraft} (Int)

\textbf{Skills/Level:} 2 + Intelligence Bonus

\poorbab{}
\poorfor{}
\poorref{}
\goodwil{}

\begin{extraclasstable}{\multicolumn{1}{p{4cm}}{\textbf{Spellcasting}}}
\levelone{\multicolumn{1}{p{5cm}}{\raggedright{} - } & \multicolumn{1}{p{4cm}}{\raggedright{} +1 level of wizard spellcasting ability}}
\leveltwo{\multicolumn{1}{p{5cm}}{\raggedright{} Expanded Spellbook} & \multicolumn{1}{p{4cm}}{\raggedright{} +1 level of wizard spellcasting ability}}
\levelthree{\multicolumn{1}{p{5cm}}{\raggedright{} \linkfeat{Magical Aptitude} } & \multicolumn{1}{p{4cm}}{\raggedright{} +1 level of wizard spellcasting ability}}
\levelfour{\multicolumn{1}{p{5cm}}{\raggedright{} Minor School Esoterica} & \multicolumn{1}{p{4cm}}{\raggedright{} +1 level of wizard spellcasting ability}}
\levelfive{\multicolumn{1}{p{5cm}}{\raggedright{} Expanded Spellbook} & \multicolumn{1}{p{4cm}}{\raggedright{} +1 level of wizard spellcasting ability}}
\levelsix{\multicolumn{1}{p{5cm}}{\raggedright{} Caster Level Increase +1} & \multicolumn{1}{p{4cm}}{\raggedright{} +1 level of wizard spellcasting ability}}
\levelseven{\multicolumn{1}{p{5cm}}{\raggedright{} Moderate School Esoterica} & \multicolumn{1}{p{4cm}}{\raggedright{} +1 level of wizard spellcasting ability}}
\leveleight{\multicolumn{1}{p{5cm}}{\raggedright{} Expanded Spellbook} & \multicolumn{1}{p{4cm}}{\raggedright{} +1 level of wizard spellcasting ability}}
\levelnine{\multicolumn{1}{p{5cm}}{\raggedright{} Caster Level Increase +2} & \multicolumn{1}{p{4cm}}{\raggedright{} +1 level of wizard spellcasting ability}}
\levelten{\multicolumn{1}{p{5cm}}{\raggedright{} Major School Esoterica} & \multicolumn{1}{p{4cm}}{\raggedright{} +1 level of wizard spellcasting ability}}
\end{extraclasstable}

\classfeatures

\textbf{Weapon and Armor Proficiency:} The Master Specialist gains no Proficiency with any arms or armours.

\textbf{Spellcasting:} At each level, you gain new spells per day and an increase in caster level (and spells known, if applicable) as if you had also gained a level in the wizard class. You do not, however, gain any other benefit a character of that class would have gained.

\textbf{Magical Aptitude:} At 3rd level, you gain \linkfeat{Magical Aptitude} as a bonus feat.

\textbf{Expanded Spellbook:} When you reach 2nd level, you can add one spell of your chosen school to your spellbook. The spell can be of any level that you can cast, and it is in addition to the normal spells gained when increasing your level. You can add another spell of your chosen school to your spellbook at 5th and at 8th level.

\textbf{Minor School Esoterica (Ex):} At 4th level, your unflagging focus on your chosen school opens your mind to new possibilities and grants you the first taste of the unique skills of a master specialist. You gain an ability from those below based on your chosen school.

\begin{itemize*}
\item \textbf{Abjuration:} You gain a competence bonus on dispel checks equal to 1/2 your master specialist level. Conjuration: Any creature you summon or call appears with extra hit points equal to your caster level.
\item \textbf{Conjuration:} Any creature you summon or call appears with extra hit points equal to your caster level.
\item \textbf{Divination:} Divination spells you cast that have a duration of concentration remain in effect for a number of extra rounds equal to 1/2 your master specialist level after you cease
concentrating. You can cast other spells and otherwise act normally during this duration.
\item \textbf{Enchantment:} Targets of your charm spells do not gain a bonus on their saves due to being currently threatened or attacked by you or your allies. In addition, subjects of your compulsion spells do not get a bonus on saves due to being forced to take an action against their natures.
\item \textbf{Evocation:} Targets of your charm spells do not gain a bonus on their saves due to being currently threatened or attacked by you or your allies. In addition, subjects of your compulsion spells do not get a bonus on saves due to being forced to take an action against their natures.
\item \textbf{Illusion:} The save DCs of your illusion spells that have a saving throw entry of “Will disbelief” increase by 2.
\item \textbf{Necromancy:} When you cast a necromancy spell, undead allies within 60 feet gain turn resistance and a bonus on saves equal to your master specialist level for a number of rounds equal to your master specialist level.
\item \textbf{Transmutation:} When a transmutation spell you have cast is successfully dispelled, it remains in effect for 1 round and then ends as normal for dispelling. If a creature is respon- sible for the dispelling effect, it knows that the spell has been dispelled but is functioning for another round.
\end{itemize*}

\textbf{Caster Level Increase (Ex):} Upon reaching 6th level, add 1 to your caster level whenever you cast a spell of your chosen school. At 9th level, you instead add 2 to your caster level.

\textbf{Moderate School Esoterica (Ex):} At 7th level, your long study of your chosen school leads to a breakthrough. You gain an ability from those below based on your chosen school. Each ability is triggered automatically when you cast a spell from your chosen school and lasts for a number of rounds equal to the spell's level.

\begin{itemize*}
\item \textbf{Abjuration:} If you are subject to a spell that has a partial or half effect on a successful save, you suffer no adverse effect if you successfully save.
\item \textbf{Conjuration:} Dispel checks made against your conjuration spells treat your caster level as if it were 5 higher than normal.
\item \textbf{Divination:} You gain uncanny dodge (PH 50) for the duration of the spell.
\item \textbf{Enchantment:} You can immediately reroll any failed Will save against an enchantment or mind-affecting spell or ability; you must accept the result of the second roll.
\item \textbf{Evocation:} You gain resistance 20 to any one energy type that matches a descriptor used by the spell you just cast.
\item \textbf{Illusion:} You gain concealment.
\item \textbf{Necromancy:} You are immune to ability damage, ability drain, energy drain, and negative levels.
\item \textbf{Transmutation:} You can immediately reroll any failed Fortitude save against a transmutation spell or ability; you must accept the result of the second roll.
\end{itemize*}

\textbf{Major School Esoterica (Ex):} At 10th level, your knowledge of your chosen school reaches its peak. You gain an ability from those below based on your chosen school; each one can be used three times per day.

\begin{itemize*}
\item \textbf{Abjuration:} When casting an abjuration spell that normally has a range of personal, you can instead choose to cast it as a touch spell that affects a single creature. When casting an abjuration spell that is an emanation centered on you, you can instead choose to cast it as a touch spell that emanates from the touched creature.
\item \textbf{Conjuration:} You can cast a conjuration spell with a casting time of 1 standard action as a swift action.
\item \textbf{Divination:} When you cast a divination spell, you also gain true seeing (as the spell) for 5 rounds.
\item \textbf{Enchantment:} Any creature that successfully saves against one of your enchantment spells must save again 1 round later (as if you had cast the spell again) with a +5 bonus on the save.
\item \textbf{Evocation:} Any creature that fails its save against one of your evocation spells takes damage again 1 round later equal to half the damage it took when you cast the spell.
\item \textbf{Illusion:} You can cast any illusion spell as a stilled and silent spell and eschew the materials (per the Still Spell, Silent Spell, and Eschew Materials feats) without an increase in caster level or casting time.
\item \textbf{Necromancy:} When you cast a necromancy spell, undead allies within 60 feet gain fast healing 10 for 5 rounds.
\item \textbf{Transmutation:} When a creature successfully saves against a transmutation spell you cast, it takes damage equal to the level of the spell.
\end{itemize*}
