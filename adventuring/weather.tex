%%%%%%%%%%%%%%%%%%%%%%%%%%%%%%%%%%%%%%%%%%%%%%%%%%
\section{Weather}
%%%%%%%%%%%%%%%%%%%%%%%%%%%%%%%%%%%%%%%%%%%%%%%%%%

Sometimes weather can play an important role in an adventure.

Table: Random Weather is an appropriate weather table for general use, and can 
be used as a basis for a local weather tables. Terms on that table are defined 
as follows.

\textbf{Calm:} Wind speeds are light (0 to 10 mph).

\textbf{Cold:} Between 0\textdegree{} and 40\textdegree{} Fahrenheit during the day, 10 to 20 degrees 
colder at night.

\textbf{Cold Snap:} Lowers temperature by -10\textdegree{} F.

\textbf{Downpour:} Treat as rain (see Precipitation, below), but conceals as fog. 
Can create floods (see above). A downpour lasts for 2d4 hours.

\textbf{Heat Wave:} Raises temperature by +10\textdegree{} F.

\textbf{Hot:} Between 85\textdegree{} and 110\textdegree{} Fahrenheit during the day, 10 to 20 degrees 
colder at night.

\textbf{Moderate:} Between 40\textdegree{} and 60\textdegree{} Fahrenheit during the day, 10 to 20 degrees 
colder at night.

\textbf{Powerful Storm (Windstorm/Blizzard/Hurricane/Tornado):}
Wind speeds are over 50 mph (see Table: Wind Effects). In addition, blizzards 
are accompanied by heavy snow (1d3 feet), and hurricanes are accompanied by downpours 
(see above). Windstorms last for 1d6 hours. Blizzards last for 1d3 days. Hurricanes 
can last for up to a week, but their major impact on characters will come in a 
24-to-48-hour period when the center of the storm moves through their area. Tornadoes 
are very short-lived (1d6x10 minutes), typically forming as part 
of a thunderstorm system. 

\textbf{Precipitation:} Roll d\% to determine whether the precipitation is fog 
(01-30), rain/snow (31-90), or sleet/hail (91-00). Snow and sleet occur only when 
the temperature is 30\textdegree{} Fahrenheit or below. Most precipitation lasts for 2d4 hours. 
By contrast, hail lasts for only 1d20 minutes but usually accompanies 1d4 hours 
of rain.

\textbf{Storm (Duststorm/Snowstorm/Thunderstorm):} Wind 
speeds are severe (30 to 50 mph) and visibility is cut by three-quarters. Storms 
last for 2d4-1 hours. See Storms, below, for more details. 

\textbf{Warm:} Between 60\textdegree{} and 85\textdegree{} Fahrenheit during the day, 10 to 20 degrees 
colder at night.

\textbf{Windy:} Wind speeds are moderate to strong (10 to 30 mph); see Table: Wind 
Effects on the following page.

\begin{table}[htb]
\rowcolors{1}{white}{offyellow}
\caption{Random Weather}
\centering
\begin{tabular}{c l l l l}
\textbf{d\%} & \textbf{Weather} & \textbf{Cold Climate} & \textbf{Temperate Climate\textsuperscript{1}} & \textbf{Desert}\\
01-70 & Normal weather & Cold, calm & Normal for season\textsuperscript{2} & Hot, calm\\
71-80 & Abnormal weather & \shortstack{Heat wave (01-30) or\\ cold snap (31-100)} & \shortstack{Heat wave (01-50) or\\ cold snap (51-100)} & Hot, windy\\
81-90 & Inclement weather & Precipitation (snow) & Precipitation (normal for season) & Hot, windy\\
91-99 & Storm & Snowstorm & Thunderstorm, snowstorm\textsuperscript{3} & Duststorm\\
100 & Powerful storm & Blizzard & Windstorm, blizzard\textsuperscript{4}, hurricane, tornado & Downpour\\
\multicolumn{5}{l}{\textsuperscript{1} Temperate includes forest, hills, marsh, mountains, plains, and warm aquatic.}\\
\multicolumn{5}{l}{\textsuperscript{2} Winter is cold, summer is warm, spring and autumn are temperate. Marsh regions are slightly warmer in winter.}\\
%superscript 3 and superscript 4 aren't provided in the actual SRD, there's nothing to complete this table with.
\end{tabular}
\end{table}

%%%%%%%%%%%%%%%%%%%%%%%%%
\subsection{Rain, Snow, Sleet, and Hail}
%%%%%%%%%%%%%%%%%%%%%%%%%

Bad weather frequently slows or halts travel and makes it virtually impossible 
to navigate from one spot to another. Torrential downpours and blizzards obscure 
vision as effectively as a dense fog.

Most precipitation is rain, but in cold conditions it can manifest as snow, sleet, 
or hail. Precipitation of any kind followed by a cold snap in which the temperature 
dips from above freezing to 30\textdegree{} F or below may produce ice.

\textit{Rain:} Rain 
reduces visibility ranges by half, resulting in a -4 penalty on \linkskill{Spot} and \linkskill{Search} 
checks. It has the same effect on flames, ranged weapon attacks, and Listen checks 
as severe wind.

\textit{Snow:} Falling snow has the same effects on visibility, ranged weapon attacks, 
and skill checks as rain, and it costs 2 squares of movement to enter a snow-covered 
square. A day of snowfall leaves 1d6 inches of snow on the ground.

\textit{Heavy Snow:} Heavy snow has the same effects as normal snowfall, but also 
restricts visibility as fog does (see Fog, below). A day of heavy snow leaves 1d4 
feet of snow on the ground, and it costs 4 squares of movement to enter a square 
covered with heavy snow. Heavy snow accompanied by strong or severe winds may result 
in snowdrifts 1d4x5 feet deep, especially in and around objects 
big enough to deflect the wind -- a cabin or a large tent, for instance. There is 
a 10\% chance that a heavy snowfall is accompanied by lightning (see Thunderstorm, 
below). Snow has the same effect on flames as moderate wind.

\textit{Sleet:} Essentially frozen rain, sleet has the same effect as rain while 
falling (except that its chance to extinguish protected flames is 75\%) and the 
same effect as snow once on the ground. 

\textit{Hail:} Hail does not reduce visibility, but the sound of falling hail makes 
Listen checks more difficult (-4 penalty). Sometimes (5\% chance) hail can become 
large enough to deal 1 point of lethal damage (per storm) to anything in the open. 
Once on the ground, hail has the same effect on movement as snow.

%%%%%%%%%%%%%%%%%%%%%%%%%
\subsection{Storms}\index{Storms}
%%%%%%%%%%%%%%%%%%%%%%%%%

The combined effects of precipitation (or dust) and wind that accompany all storms 
reduce visibility ranges by three quarters, imposing a -8 penalty on \linkskill{Spot}, \linkskill{Search}, 
and \linkskill{Listen} checks. Storms make ranged weapon attacks impossible, except for those 
using siege weapons, which have a -4 penalty on attack rolls. They automatically 
extinguish candles, torches, and similar unprotected flames. They cause protected 
flames, such as those of lanterns, to dance wildly and have a 50\% chance to extinguish 
these lights. See Table: Wind Effects for possible consequences to creatures caught 
outside without shelter during such a storm. Storms are divided into the following 
three types. 

\textit{Duststorm (CR 3):} These desert storms differ from other storms in that 
they have no precipitation. Instead, a duststorm blows fine grains of sand that 
obscure vision, smother unprotected flames, and can even choke protected flames 
(50\% chance). Most duststorms are accompanied by severe winds and leave behind 
a deposit of 1d6 inches of sand. However, there is a 10\% chance for a greater 
duststorm to be accompanied by windstorm-magnitude winds (see Table: Wind Effects). 
These greater duststorms deal 1d3 points of nonlethal damage each round to anyone 
caught out in the open without shelter and also pose a choking hazard (see Drowning -- except 
that a character with a scarf or similar protection across her mouth and nose does 
not begin to choke until after a number of rounds equal to 10 x 
her Constitution score). Greater duststorms leave 2d3-1 feet of fine sand in their 
wake.

\textit{Snowstorm:} In addition to the wind and precipitation common to other storms, 
snowstorms leave 1d6 inches of snow on the ground afterward. 

\textit{Thunderstorm:} In addition to wind and precipitation (usually rain, but 
sometimes also hail), thunderstorms are accompanied by lightning that can pose 
a hazard to characters without proper shelter (especially those in metal armor). 
As a rule of thumb, assume one bolt per minute for a 1-hour period at the center 
of the storm. Each bolt causes electricity damage equal to 1d10 eight-sided dice. 
One in ten thunderstorms is accompanied by a tornado (see below). 

\textbf{Powerful Storms:} Very high winds and torrential precipitation reduce visibility 
to zero, making Spot, Search, and Listen checks and all ranged weapon attacks impossible. 
Unprotected flames are automatically extinguished, and protected flames have a 
75\% chance of being doused. Creatures caught in the area must make a DC 20 Fortitude 
save or face the effects based on the size of the creature (see Table: Wind Effects). 
Powerful storms are divided into the following four types.

\textit{Windstorm:} While accompanied by little or no precipitation, windstorms 
can cause considerable damage simply through the force of their wind.

\textit{Blizzard:} The combination of high winds, heavy snow (typically 1d3 feet), 
and bitter cold make blizzards deadly for all who are unprepared for them.

\textit{Hurricane:} In addition to very high winds and heavy rain, hurricanes are 
accompanied by floods. Most adventuring activity is impossible under such conditions.

\textit{Tornado:} One in ten thunderstorms is accompanied by a tornado.

%%%%%%%%%%%%%%%%%%%%%%%%%
\subsection{Fog}\index{Fog}
%%%%%%%%%%%%%%%%%%%%%%%%%

Whether in the form of a low-lying cloud or a mist rising from the ground, fog 
obscures all sight, including darkvision, beyond 5 feet. Creatures 5 feet away 
have concealment (attacks by or against them have a 20\% miss chance).

%%%%%%%%%%%%%%%%%%%%%%%%%
\subsection{Winds}\index{Wind}
%%%%%%%%%%%%%%%%%%%%%%%%%

The wind can create a stinging spray of sand or dust, fan a large fire, heel over 
a small boat, and blow gases or vapors away. If powerful enough, it can even knock 
characters down (see Table: Wind Effects), interfere with ranged attacks, or impose 
penalties on some skill checks.

\textit{Light Wind:} A gentle breeze, having little or no game effect.

\textit{Moderate Wind:} A steady wind with a 50\% chance of extinguishing small, 
unprotected flames, such as candles.

\textit{Strong Wind:} Gusts that automatically extinguish unprotected flames (candles, 
torches, and the like). Such gusts impose a -2 penalty on ranged attack rolls and 
on \linkskill{Listen} checks.

\textit{Severe Wind:} In addition to automatically extinguishing any unprotected 
flames, winds of this magnitude cause protected flames (such as those of lanterns) 
to dance wildly and have a 50\% chance of extinguishing these lights. Ranged weapon 
attacks and Listen checks are at a -4 penalty. This is the velocity of wind produced 
by a \textit{gust of wind }spell.

\textit{Windstorm:} Powerful enough to bring down branches if not whole trees, 
windstorms automatically extinguish unprotected flames and have a 75\% chance of 
blowing out protected flames, such as those of lanterns. Ranged weapon attacks 
are impossible, and even siege weapons have a -4 penalty on attack rolls. Listen 
checks are at a -8 penalty due to the howling of the wind. 

\textit{Hurricane-Force Wind:} All flames are extinguished. Ranged attacks are 
impossible (except with siege weapons, which have a -8 penalty on attack rolls). 
Listen checks are impossible: All characters can hear is the roaring of the wind. 
Hurricane-force winds often fell trees.

\textit{Tornado (CR 10):} All flames are extinguished. All ranged attacks are impossible 
(even with siege weapons), as are Listen checks. Instead of being blown away (see 
Table: Wind Effects), characters in close proximity to a tornado who fail their 
Fortitude saves are sucked toward the tornado. Those who come in contact with the 
actual funnel cloud are picked up and whirled around for 1d10 rounds, taking 6d6 
points of damage per round, before being violently expelled (falling damage may 
apply). While a tornado's rotational speed can be as great as 300 mph, the funnel 
itself moves forward at an average of 30 mph (roughly 250 feet per round). A tornado 
uproots trees, destroys buildings, and causes other similar forms of major destruction.

\begin{table}[htb]
\rowcolors{1}{white}{offyellow}
\caption{Wind Effects}
\centering
\begin{tabular}{l l l l l c}
\textbf{Wind Force} & \textbf{Wind Speed} & \multicolumn{1}{p{2.5cm}}{\textbf{Ranged Attacks Normal/Siege Weapons\textsuperscript{1}}} & \textbf{Creature Size\textsuperscript{2}} & \multicolumn{1}{p{2cm}}{\textbf{Wind Effect on Creatures}} & \textbf{Fort Save DC}\\
Light & 0-10 mph & -/- & Any & None & -\\
Moderate & 11-20 mph & -/- & Any & None & -\\
Strong & 21-30 mph & -2/- & Tiny or smaller & Knocked down & 10\\
& & & Small or larger & None & \\
Severe & 31-50 mph & -4/- & Tiny & Blown away & 15\\
& & & Small & Knocked down& \\
& & & Medium & Checked& \\
& & & Large or larger & None& \\
Windstorm & 51-74 mph & Impossible/-4 & Small or smaller & Blown away & 18\\
& & & Medium & Knocked down& \\
& & & Large or Huge & Checked& \\
& & & Gargantuan or Colossal & None& \\
Hurricane & 75-174 mph & Impossible/-8 & Medium or smaller & Blown away & 20\\
& & & Large & Knocked down& \\
& & & Huge & Checked& \\
& & & Gargantuan or Colossal & None& \\
Tornado & 175-300 mph & Impossible/impossible & Large or smaller & Blown away & 30\\
& & & Huge & Knocked down& \\
& & & Gargantuan or Colossal & Checked& \\
\multicolumn{6}{p{16cm}}{\textsuperscript{1} The siege weapon category includes ballista and catapult attacks as well as boulders tossed by giants.}\\
\multicolumn{6}{p{16cm}}{\textsuperscript{2} Flying or airborne creatures are treated as one size category smaller than their actual size, so an airborne Gargantuan dragon is treated as Huge for purposes of wind effects.}\\
\multicolumn{6}{p{16cm}}{\textit{Checked:} Creatures are unable to move forward against the force of the wind. Flying creatures are blown back 1d6x5 feet.}\\
\multicolumn{6}{p{16cm}}{\textit{Knocked Down:}\index{Knocked Down} Creatures are knocked prone by the force of the wind. Flying creatures are instead blown back 1d6x10 feet.}\\
\multicolumn{6}{p{16cm}}{\textit{Blown Away:} Creatures on the ground are knocked prone and rolled 1d4x10 feet, taking 1d4 points of nonlethal damage per 10 feet. Flying creatures are blown back 2d6×10 feet and take 2d6 points of nonlethal damage due to battering and buffeting.}\\
\end{tabular}
\end{table}

%%%%%%%%%%%%%%%%%%%%%%%%%%%%%%%%%%%%%%%%%%%%%%%%%%
\section{The Environment}
%%%%%%%%%%%%%%%%%%%%%%%%%%%%%%%%%%%%%%%%%%%%%%%%%%

Environmental hazards specific to one kind of terrain (such as an avalanche, which 
occurs in the mountains) are described in Wilderness, above. Environmental hazards 
common to more than one setting are detailed below.

%%%%%%%%%%%%%%%%%%%%%%%%%
\subsection{Acid Effects}
%%%%%%%%%%%%%%%%%%%%%%%%%

Corrosive acids deals 1d6 points of damage per round of exposure except in the 
case of total immersion (such as into a vat of acid), which deals 10d6 points of 
damage per round. An attack with acid, such as from a hurled vial or a monster's 
spittle, counts as a round of exposure.

The fumes from most acids are inhaled poisons. Those who come close enough to a 
body of acid large enough to dunk a creature in it must make a DC 13 Fortitude save or 
take 1 point of Constitution damage. All such characters must make a second save 
1 minute later or take another 1d4 points of Constitution damage.

Creatures immune to acid's caustic properties might still drown in it if they are 
totally immersed (see Drowning).

%%%%%%%%%%%%%%%%%%%%%%%%%
\subsection{Cold Dangers}\index{Cold Weather Effects}
%%%%%%%%%%%%%%%%%%%%%%%%%

Cold and exposure deal nonlethal damage to the victim. This nonlethal damage cannot 
be recovered until the character gets out of the cold and warms up again. Once 
a character is rendered unconscious through the accumulation of nonlethal damage, 
the cold and exposure begins to deal lethal damage at the same rate.

An unprotected character in cold weather (below 40\textdegree{} F) must make a Fortitude save 
each hour (DC 15, + 1 per previous check) or take 1d6 points of nonlethal damage. 
A character who has the \linkskill{Survival} skill may receive a bonus on this saving throw 
and may be able to apply this bonus to other characters as well (see the skill 
Description).

In conditions of severe cold or exposure (below 0\textdegree{} F), an unprotected character 
must make a Fortitude save once every 10 minutes (DC 15, +1 per previous check), 
taking 1d6 points of nonlethal damage on each failed save. A character who has 
the Survival skill may receive a bonus on this saving throw and may be able to 
apply this bonus to other characters as well (see the skill description). Characters 
wearing winter clothing only need check once per hour for cold and exposure damage.

A character who takes any nonlethal damage from cold or exposure is beset by frostbite 
or hypothermia (treat her as fatigued). These penalties end when the character 
recovers the nonlethal damage she took from the cold and exposure.

Extreme cold (below -20\textdegree{} F) deals 1d6 points of lethal damage per minute (no save). 
In addition, a character must make a Fortitude save (DC 15, +1 per previous check) 
or take 1d4 points of nonlethal damage. Those wearing metal armor or coming into 
contact with very cold metal are affected as if by a \textit{chill metal }spell.

%%%
\subsubsection{Ice Effects}
%%%

Characters walking on ice must spend 2 squares of movement to enter a square covered 
by ice, and the DC for \linkskill{Balance} and \linkskill{Tumble} checks increases by +5. Characters in 
prolonged contact with ice may run the risk of taking damage from severe cold (see 
above).

%%%%%%%%%%%%%%%%%%%%%%%%%
\subsection{Darkness}\index{Darkness}
%%%%%%%%%%%%%%%%%%%%%%%%%

Darkvision allows many characters and monsters to see perfectly well without any 
light at all, but characters with normal vision (or low-light vision, for that 
matter) can be rendered completely blind by putting out the lights. Torches or 
lanterns can be blown out by sudden gusts of subterranean wind, magical light sources 
can be dispelled or countered, or magical traps might create fields of impenetrable 
darkness.

In many cases, some characters or monsters might be able to see, while others are 
blinded. For purposes of the following points, a blinded creature is one who simply 
can't see through the surrounding darkness.

\begin{itemize}
\item Creatures blinded by darkness lose the ability to deal extra damage due to precision (for example, a sneak attack).
\item Blinded creatures are hampered in their movement, and pay 2 squares of movement 
per square moved into (double normal cost). Blinded creatures can't run or charge.
\item All opponents have total concealment from a blinded creature, so the blinded creature 
has a 50\% miss chance in combat. A blinded creature must first pinpoint the location 
of an opponent in order to attack the right square; if the blinded creature launches 
an attack without pinpointing its foe, it attacks a random square within its reach. 
For ranged attacks or spells against a foe whose location is not pinpointed, roll 
to determine which adjacent square the blinded creature is facing; its attack is 
directed at the closest target that lies in that direction.
\item A blinded creature loses its Dexterity adjustment to AC and takes a -2 penalty to AC.
\item A blinded creature takes a -4 penalty on \linkskill{Search} checks and most Strength- and Dexterity-based 
skill checks, including any with an armor check penalty. A creature blinded by 
darkness automatically fails any skill check relying on vision.
\item Creatures blinded by darkness cannot use gaze attacks and are immune to gaze attacks.
\item A creature blinded by darkness can make a \linkskill{Listen} check as a free action each round 
in order to locate foes (DC equal to opponents' Move Silently checks). A successful 
check lets a blinded character hear an unseen creature "over there somewhere".
It's almost impossible to pinpoint the location of an unseen creature. A Listen 
check that beats the DC by 20 reveals the unseen creature's square (but the unseen 
creature still has total concealment from the blinded creature).
\item A blinded creature can grope about to find unseen creatures. A character can make 
a touch attack with his hands or a weapon into two adjacent squares using a standard 
action. If an unseen target is in the designated square, there is a 50\% miss chance 
on the touch attack. If successful, the groping character deals no damage but has 
pinpointed the unseen creature's current location. (If the unseen creature moves, 
its location is once again unknown.)
\item If a blinded creature is struck by an unseen foe, the blinded character pinpoints 
the location of the creature that struck him (until the unseen creature moves, 
of course). The only exception is if the unseen creature has a reach greater than 
5 feet (in which case the blinded character knows the location of the unseen opponent, 
but has not pinpointed him) or uses a ranged attack (in which case, the blinded 
character knows the general direction of the foe, but not his location).
\item A creature with the scent ability automatically pinpoints unseen creatures within 5 feet of its location.
\end{itemize}

%%%%%%%%%%%%%%%%%%%%%%%%%
\subsection{Falling}\index{Falling}
%%%%%%%%%%%%%%%%%%%%%%%%%

\textbf{Falling Damage:} The basic rule is simple: 1d6 points of damage per 10 
feet fallen, to a maximum of 20d6.

If a character deliberately jumps instead of merely slipping or falling, the damage 
is the same but the first 1d6 is nonlethal damage. A DC 15 \linkskill{Jump} check or DC 15 
\linkskill{Tumble} check allows the character to avoid any damage from the first 10 feet fallen 
and converts any damage from the second 10 feet to nonlethal damage. Thus, a character 
who slips from a ledge 30 feet up takes 3d6 damage. If the same character deliberately 
jumped, he takes 1d6 points of nonlethal damage and 2d6 points of lethal damage. 
And if the character leaps down with a successful Jump or Tumble check, he takes 
only 1d6 points of nonlethal damage and 1d6 points of lethal damage from the plunge.

Falls onto yielding surfaces (soft ground, mud) also convert the first 1d6 of damage 
to nonlethal damage. This reduction is cumulative with reduced damage due to deliberate 
jumps and the Jump skill.

\textbf{Falling into Water:} Falls into water are handled somewhat differently. 
If the water is at least 10 feet deep, the first 20 feet of falling do no damage. 
The next 20 feet do nonlethal damage (1d3 per 10-foot increment). Beyond that, 
falling damage is lethal damage (1d6 per additional 10-foot increment).

Characters who deliberately dive into water take no damage on a successful DC 15 
Swim check or DC 15 Tumble check, so long as the water is at least 10 feet deep 
for every 30 feet fallen. However, the DC of the check increases by 5 for every 
50 feet of the dive. 

%%%%%%%%%%%%%%%%%%%%%%%%%
\subsection{Falling Objects}
%%%%%%%%%%%%%%%%%%%%%%%%%

Just as characters take damage when they fall more than 10 feet, so too do they 
take damage when they are hit by falling objects.

Objects that fall upon characters deal damage based on their weight and the distance 
they have fallen.

For each 200 pounds of an object's weight, the object deals 1d6 points of damage, 
provided it falls at least 10 feet. Distance also comes into play, adding an additional 
1d6 points of damage for every 10-foot increment it falls beyond the first (to 
a maximum of 20d6 points of damage).

Objects smaller than 200 pounds also deal damage when dropped, but they must fall 
farther to deal the same damage. Use Table: Damage from Falling Objects to see 
how far an object of a given weight must drop to deal 1d6 points of damage.

\begin{table}[htb]
\rowcolors{1}{white}{offyellow}
\caption{Damage from Falling Objects}
\centering
\begin{tabular}{cc}
\textbf{Object Weight} & \textbf{Falling Distance}\\
200-101lb & 20ft\\
100-51lb & 30ft\\
50-31lb & 40ft\\
30-11lb & 50ft\\
10-6lb & 60ft\\
5-1lb & 70ft\\
\end{tabular}
\end{table}

For each additional increment an object falls, it deals an additional 1d6 points 
of damage.

Objects weighing less than 1 pound do not deal damage to those they land upon, 
no matter how far they have fallen.

%%%%%%%%%%%%%%%%%%%%%%%%%
\subsection{Heat Dangers}\index{Heat Dangers}
%%%%%%%%%%%%%%%%%%%%%%%%%

Heat deals nonlethal damage that cannot be recovered until the character gets cooled 
off (reaches shade, survives until nightfall, gets doused in water, is targeted 
by \textit{endure elements}, and so forth). Once rendered unconscious through the 
accumulation of nonlethal damage, the character begins to take lethal damage at 
the same rate.

A character in very hot conditions (above 90\textdegree{} F) must make a Fortitude saving 
throw each hour (DC 15, +1 for each previous check) or take 1d4 points of nonlethal 
damage. Characters wearing heavy clothing or armor of any sort take a -4 penalty 
on their saves. A character with the \linkskill{Survival} skill may receive a bonus on this 
saving throw and may be able to apply this bonus to other characters as well (see 
the skill description). Characters reduced to unconsciousness begin taking lethal 
damage (1d4 points per hour).

In severe heat (above 110\textdegree{} F), a character must make a Fortitude save once every 
10 minutes (DC 15, +1 for each previous check) or take 1d4 points of nonlethal 
damage. Characters wearing heavy clothing or armor of any sort take a -4 penalty 
on their saves. A character with the Survival skill may receive a bonus on this 
saving throw and may be able to apply this bonus to other characters as well. Characters 
reduced to unconsciousness begin taking lethal damage (1d4 points per each 10-minute 
period).

A character who takes any nonlethal damage from heat exposure now suffers from 
heatstroke and is fatigued.

These penalties end when the character recovers the nonlethal damage she took from 
the heat.

Extreme heat (air temperature over 140\textdegree{} F, fire, boiling water, lava) deals lethal 
damage. Breathing air in these temperatures deals 1d6 points of damage per minute 
(no save). In addition, a character must make a Fortitude save every 5 minutes 
(DC 15, +1 per previous check) or take 1d4 points of nonlethal damage. Those wearing 
heavy clothing or any sort of armor take a -4 penalty on their saves. In addition, 
those wearing metal armor or coming into contact with very hot metal are affected 
as if by a \textit{heat metal }spell.

Boiling water deals 1d6 points of scalding damage, unless the character is fully 
immersed, in which case it deals 10d6 points of damage per round of exposure.

%%%
\subsubsection{Catching on Fire}
%%%

Characters exposed to burning oil, bonfires, and noninstantaneous magic fires 
might find their clothes, hair, or equipment on fire. Spells with an instantaneous 
duration don't normally set a character on fire, since the heat and flame 
from these come and go in a flash.

Characters at risk of catching fire are allowed a DC 15 Reflex save to avoid this 
fate. If a character's clothes or hair catch fire, he takes 1d6 points of damage 
immediately. In each subsequent round, the burning character must make another 
Reflex saving throw. Failure means he takes another 1d6 points of damage that round. 
Success means that the fire has gone out. (That is, once he succeeds on his saving 
throw, he's no longer on fire.)

A character on fire may automatically extinguish the flames by jumping into enough 
water to douse himself. If no body of water is at hand, rolling on the ground or 
smothering the fire with cloaks or the like permits the character another save 
with a +4 bonus.

Those unlucky enough to have their clothes or equipment catch fire must make DC 
15 Reflex saves for each item. Flammable items that fail take the same amount of 
damage as the character.

%%%
\subsubsection{Lava Effects}
%%%

Lava or magma deals 2d6 points of damage per round of exposure, except in the case 
of total immersion (such as when a character falls into the crater of an active 
volcano), which deals 20d6 points of damage per round.

Damage from magma continues for 1d3 rounds after exposure ceases, but this additional 
damage is only half of that dealt during actual contact (that is, 1d6 or 10d6 points 
per round).

An immunity or resistance to fire serves as an immunity to lava or magma. However, 
a creature immune to fire might still drown if completely immersed in lava (see 
Drowning, below).

%%%%%%%%%%%%%%%%%%%%%%%%%
\subsection{Smoke Effects}\index{Smoke}
%%%%%%%%%%%%%%%%%%%%%%%%%

A character who breathes heavy smoke must make a Fortitude save each round (DC 
15, +1 per previous check) or spend that round choking and coughing. A character 
who chokes for 2 consecutive rounds takes 1d6 points of nonlethal damage.

Smoke obscures vision, giving concealment (20\% miss chance) to characters within 
it.

%%%%%%%%%%%%%%%%%%%%%%%%%
\subsection{Starvation and Thirst}\index{Starvation}\index{Thirst}
%%%%%%%%%%%%%%%%%%%%%%%%%

Characters might find themselves without food or water and with no means to obtain 
them. In normal climates, Medium characters need at least a gallon of fluids and 
about a pound of decent food per day to avoid starvation. (Small characters need 
half as much.) In very hot climates, characters need two or three times as much 
water to avoid dehydration.

A character can go without water for 1 day plus a number of hours equal to his 
Constitution score. After this time, the character must make a Constitution check 
each hour (DC 10, +1 for each previous check) or take 1d6 points of nonlethal damage.

A character can go without food for 3 days, in growing discomfort. After this time, 
the character must make a Constitution check each day (DC 10, +1 for each previous 
check) or take 1d6 points of nonlethal damage.

Characters who have taken nonlethal damage from lack of food or water are fatigued. 
Nonlethal damage from thirst or starvation cannot be recovered until the character 
gets food or water, as needed -- not even magic that restores hit points heals this 
damage.

%%%%%%%%%%%%%%%%%%%%%%%%%
\subsection{Suffocation}\index{Suffocation}
%%%%%%%%%%%%%%%%%%%%%%%%%

A character who has no air to breathe can hold her breath for 2 rounds per point 
of Constitution. After this period of time, the character must make a DC 10 Constitution 
check in order to continue holding her breath. The save must be repeated each round, 
with the DC increasing by +1 for each previous success.

When the character fails one of these Constitution checks, she begins to suffocate. 
In the first round, she falls unconscious (0 hit points). In the following round, 
she drops to -1 hit points and is dying. In the third round, she suffocates.

\textbf{Slow Suffocation:} A Medium character can breathe easily for 6 hours in 
a sealed chamber measuring 10 feet on a side. After that time, the character takes 
1d6 points of nonlethal damage every 15 minutes. Each additional Medium character 
or significant fire source (a torch, for example) proportionally reduces the time 
the air will last.

Small characters consume half as much air as Medium characters. A larger volume 
of air, of course, lasts for a longer time. 

%%%%%%%%%%%%%%%%%%%%%%%%%
\subsection{Water Dangers}\index{Water Dangers}
%%%%%%%%%%%%%%%%%%%%%%%%%

Any character can wade in relatively calm water that isn't over his head, no check 
required. Similarly, swimming in calm water only requires skill checks with a DC 
of 10. Trained swimmers can just take 10. (Remember, however, that armor or heavy 
gear makes any attempt at swimming much more difficult. See the \linkskill{Swim} skill description.)

By contrast, fast-moving water is much more dangerous. On a successful DC 15 Swim 
check or a DC 15 Strength check, it deals 1d3 points of nonlethal damage per round 
(1d6 points of lethal damage if flowing over rocks and cascades). On a failed check, 
the character must make another check that round to avoid going under.

Very deep water is not only generally pitch black, posing a navigational hazard, 
but worse, it deals water pressure damage of 1d6 points per minute for every 100 
feet the character is below the surface. A successful Fortitude save (DC 15, +1 
for each previous check) means the diver takes no damage in that minute. Very cold 
water deals 1d6 points of nonlethal damage from hypothermia per minute of exposure.

%%%
\subsubsection{Drowning}\index{Drowning}
%%%

Any character can hold her breath for a number of rounds equal to twice her Constitution 
score. After this period of time, the character must make a DC 10 Constitution 
check every round in order to continue holding her breath. Each round, the DC increases 
by 1. 

When the character finally fails her Constitution check, she begins to drown. In 
the first round, she falls unconscious (0 hp). In the following round, she drops 
to -1 hit points and is dying. In the third round, she drowns.

It is possible to drown in substances other than water, such as sand, quicksand, 
fine dust, and silos full of grain.
