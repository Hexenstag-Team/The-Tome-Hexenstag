%%%%%%%%%%%%%%%%%%%%%%%%%
\skillentry{Knowledge}{(Int; Trained Only)}
%%%%%%%%%%%%%%%%%%%%%%%%%

Like the \linkskill{Craft} and \linkskill{Profession} skills, Knowledge actually encompasses a number of unrelated skills. Knowledge represents a study of some body of lore, possibly an academic or even scientific discipline.

Below are listed typical fields of study.

\begin{itemize*}
	\item \textbf{Arcana:} You are knowledgeable about the general principles and theories of magic, how it works, and what can and cannot be done with it. This is theoretical knowledge, which cannot be used to either cast spells or identify spells being cast. If you have 5 or more ranks of Knowledge (Arcana), it will provide a synergy bonus of +2 to any Spellcraft check made to determine the success of spell research.
	\item \textbf{Architecture:} You are knowledgeable about historical and current trends and styles in the design of buildings, aqueducts, bridges and fortifications. You can identify the particular style a building was built with, and you know what influenced that style. You know how to design buildings in a given style and draw clear and detailed plans for them in such a way that engineers and craftsmen can build based on them.
	\item \textbf{Dungeoneering:} You are knowledgeable about Aberrations, Caverns, Oozes, and Spelunking.
	\item \textbf{Code of Martial Honor:} You know the principles of a code of honor by heart, can recite them by memory, and can apply them to any and all situations in your life.
	\item \textbf{Geography (Region):} You are familiar with the lands, terrain, climate, people and customs of a specific region.
	\item \textbf{Hearth Wisdom:} You know a wide variety of common tales and wisdom also known as "old wives' tales." You know of among other things folklore, myths, origins of place names, and folk rememdies for common ailments.
	\item \textbf{Heraldry (Region):} You are familiar with the heradric crests and blazons of a given region. If the region has any regulated system for the symbology of heraldry, you are familiar with it. If you see a blazon, you have a better chance of identifying the noble house it signifies.
	\item \textbf{History (Region):} You know about the events that have shaped a particular region. You can describe what happened, when it happened, and what historical figures were involved. You understand the context of this region and how it became what it is today.
	\item \textbf{Illithid Lore:} You are familiar with the capabilities, habits, strengths and weaknesses of illithids. You know how to detect their presence, how to track them, and how to defeat them.
	\item \textbf{Law (Region):} You are familiar with the laws of this region. You know what acts are legal and illegal, and what the punishments are for crimes. You also know of any precedents that affect current laws or their interpretation.
	\item \textbf{Literature:} You are well read in the classics and know many stories, plays, ballads, epic poems and legends.
	\item \textbf{Local (Region):} You have knowledge of a specific local area. You know your way around the streets in a city, or trails and paths in the wilderness. You know what the best inns and restaraunts are, and you also know of any cultural peculularities and of any notable or scenic spots. You also know of the major personalities in the area.
	\item \textbf{Mathematics:} You can solve a broad variety of problems in basic math, algebra and geometry. You have knowledge of the latest discoveries, proofs and theorems of the various mathematical specialties.
	\item \textbf{Nature:} You are familiar with the common types of normal plants and animals and know how they fit into the overall ecology. You know the seasons and various other rhythms and cycles that govern the natural world, and you have a sense of the flow and patterns of weather.
	\item \textbf{Nobility \& Royalty (Region):} You know the names and rankings of noble and royal persons in a given region and may recognize them on sight. You know how the various families are interrelated, have some knowledge of their family trees, and know the relative power and status of the noble houses. You can identify a house's heraldric crest by sight, but your knowledge of heraldric symbology does not extend past sight recognition. You are familiar with the proper forms of address for nobles and royals, and know how to properly act in their presence.
	\item \textbf{Planes:} You are knowledgeable of other planes of existence. You know what conditions exist on the various planes, and what powers live on these planes.
	\item \textbf{Politics (Region):} You know the inner workings of governments and their bureaucracies. You know how to petition officials and work the system, including the offering of bribes and the use of other subterfuge.
	\item \textbf{Psionics:} You are knowledgeable about the general principles and theories of psionics, how it works, and what can and cannot be done with it. This is only theoretical knowledge, which cannot be used to either manifest psionic powers or identify such powers as they are manifested by others. The DC for answering really easy questions about psionics is 10, for basic questions is 15, and for really hard questions is 30. If you have 5 or more ranks of Autohypnosis, you get a +2 synergy bonus on Knowledge (psionics) checks.
	\item \textbf{Religion:} You know of the various gods and goddesses in the pantheons of the world and know the portfolios and domains for them, and can identify the holy symbols of the gods. You know the basic mythic history of the pantheon. You also know about lost or dead gods.
	\item \textbf{Streetwise:} You know how to get by on the street as a commoner. You know how to conduct yourself to blend in on the street and in common taprooms. You are familiar with the social aspects of drinking and flirting, know the common games of chance, and can tail people on busy streets without being noticed.
	\item \textbf{Undead:} You know about the various common types of undead and about the various mundane ways of dealing with them. You know what tactics they use against the living, and the best tactics to use against them.
	\item \textbf{War:} You are knowledgeable about military theory. You know the history of the famous and pivotal battles, and know the tactics used. You know the principles governing military conduct, and the theory behind such things as siege engines, sapping, and siege tactics and strategy.
	\item \textbf{Weaponry:} You have a knowledge of the history and development of weaponry. You know what weapons were designed as counters to what other weapons, what weapons are best against specific types of armor, what tactics a specific weapon is good for, and what tactics are good against a specific weapon
\end{itemize*}

\textbf{Check:} Answering a question within your field of study has a DC of 10 (for really easy questions), 15 (for basic questions), or 20 to 30 (for really tough questions).

In many cases, you can use this skill to identify monsters and their special powers or vulnerabilities. In general, the DC of such a check equals 10 + the monster's CR. A successful check allows you to remember a bit of useful information about that monster.

For every 5 points by which your check result exceeds the DC, you recall another piece of useful information.

\textbf{Action:} Usually none. In most cases, making a Knowledge check doesn't take an action -- you simply know the answer or you don't.

\textbf{Try Again:} No. The check represents what you know, and thinking about a topic a second time doesn't let you know something that you never learned in the first place.

\textbf{Untrained:} An untrained Knowledge check is simply an Intelligence check. Without actual training, you know only common knowledge (DC 10 or lower).
