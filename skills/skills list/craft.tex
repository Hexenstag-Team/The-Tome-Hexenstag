%%%%%%%%%%%%%%%%%%%%%%%%%
\skillentry{Craft}{(Int)}
%%%%%%%%%%%%%%%%%%%%%%%%%

Like \linkskill{Knowledge}, \linkskill{Perform}, and \linkskill{Profession}, Craft is actually a number of separate skills. You could have several Craft skills, each with its own ranks, each purchased as a separate skill.

A Craft skill is specifically focused on creating something. If nothing is created by the endeavor, it probably falls under the heading of a \linkskill{Profession} skill.

\begin{itemize*}
	\item \textbf{Armorsmithing:} You are capable of making armors or shields of any type that is predominantly metal. This skill does not apply to the making of fur or hide, padded, leather, or studded leather armor, nor does it apply to wooden shields
	\item \textbf{Basketweaving:} You are capable of weaving baskets from straw, reeds, bamboo or other suitable material.
	\item \textbf{Bookbinding:} You can take pages of manuscript and make a book out of them by sewing (or otherwise fastening) the pages together and binding them within book covers. Book covers are typically made of leather, but could be made of other materials.
	\item \textbf{Bowmaking:} You are capable of make bows and arrows, and fletching the arrows.
	\item \textbf{Blacksmithing:} You can work with iron and make useful items out of it, e.g. nails, horseshoes, and various tools. If you have 5 or more ranks of Craft (metalworking), you get a +2 synergy bonus on any Craft (blacksmithing) check.
	\item \textbf{Brickmaking:} You are capable of making bricks from mud or clay, and know how to cure them either in the sun or by baking them in a kiln. You can also maintain and repair the kiln used.
	\item \textbf{Calligraphy:} You are skilled with quill and ink to produce beautiful, flowing and neat writing. You can also draw the decorative patterns that often adorn manuscripts or that embellish capital letters or signatures.
	\item \textbf{Candlemaking:} You are skilled at making candles from animal fat, tallow, or wax.
	\item \textbf{Carpentry:} You are skilled at constructing sturdy buildings and objects such as furniture and cabinetry using milled lumber. Specialties within carpentry would include cartwrights, coopers, wainwrights and wheelwrights. This skill would also be used to make wooden shields.
	\item \textbf{Cobbling:} You can make and repair shoes and boots, typically from leather or wood. Other materials could be used to make footwear such as sandals.
	\item \textbf{Cooking:} You can produce prepared food dishes of some sophistication. You are familiar with various methods of food preparation (e.g. frying, baking, broiling, steaming) and the general types of food items (sauces, soups, stews, etc.), the uses and effects of various ingredients and spices, and can easily follow or modify a recipe, or even create new recipes.
	\item \textbf{Gemcutting:} You can cut, finish and polish rough gems, thus making them more valuable and more suitable for use in jewelry. You can also identify the type of gem it is, and estimate its value.
	\item \textbf{Glassblowing:} You can make glass from sands or other suitable materials, and create bottles and various other items by blowing molten glass. You know what additives will produce colors, and can identify different types of glass. You are also skilled at cutting glass, and otherwise working with it.
	\item \textbf{Gunsmithing:} You can construct various smokepowder weapons, from arquebuses to matchlock and flintlock muskets and pistols. If you have 5 or more ranks of Craft (machining), you get a +2 synergy bonus on any Craft (gunsmithing) check. If you have 5 or more ranks of Craft (metalworking), you can also make cannon. (smokepowder weapons not in yet)
	\item \textbf{Leatherworking:} You can create various leather items from leather, such as predominately leather armor, saddles, scabbards, backpacks, bags, pouches and other such items.
	\item \textbf{Locksmithing:} You can make mechanical locks, and are familiar with the operating principles of locks. If you have 5 or more ranks of this craft, you get a +2 synergy bonus on \linkskill{Disable Device} skill check to open locks. If you have 5 or more ranks of Craft (machining), you get a +2 synergy bonus on any Craft (locksmithing) check.
	\item \textbf{Machining:} You can craft gears and pulleys and assemble them into small or intricate mechanical devices using precision hand tools. Clockmaking would be a classic use for this skill. You can also make mechanical traps as you might find built into locks and chests.
	\item \textbf{Metalworking:} You can make general items out of metal, either useful or decorative. Tinkers and cutlers would used this skill. This skill also subsumes specializing in metals, such as goldsmith, silversmith, coppersmith, etc. This skill also encompasses the craft of making and minting coins. If you have 5 or more ranks of Craft (blacksmithing), you get a +2 synergy bonus on any Craft (metalworking) check.
	\item \textbf{Needleworking:} You are capable of various forms of needlework - knitting, crocheting, quiltmaking, tatting, laceworking - to make such things as garments, embroidery, quilts, blankets, doilies, and so on.
	\item \textbf{Painting:} You can use inks and paints to paint pictures of people and scenery. You have good eye for colours and a good sense of visual perspective.
	\item \textbf{Papermaking:} You know the procedure for making sheets of paper, parchment, vellum, or other such materials, that are suitable for writing on and using as documents.
	\item \textbf{Pottery:} You are skilled at making pots, jugs and other containers using a variety of clays. You can also operate and maintain a firing kiln and know how to glaze items to produce finished items. You are familiar with the types of clays used to make items and know where to find them and blend them as necessary.
	\item \textbf{Ropemaking:} You know how to make strong rope from various fibers such as hemp and silk.
	\item \textbf{Sculpture:} You can accurately sculpt finely detailed depictions of people, animals, objects or decorative designs into stone. The sculptures you create are works of art. If you have 5 or more ranks of Craft (stonecarving), you get a +2 synergy bonus on any Craft (sculpture) check.
	\item \textbf{Sewing:} You can cut and sew together cloth of various type to make garments. Seamstresses, dressmakers, tailors, hatmakers, and glovers would use this skill to ply their trade. If you have 5 or more ranks of Craft(leatherworking), you get a +2 synergy bonus on any Craft(sewing) check involved with making garments out of leather.
	\item \textbf{Shipmaking:} The shipwright uses this skill to build sea-worthy vessels fit to ship people and cargo across the seas. He knows the proper techniques for making keels, planks, bulkheads, rudders and masts from lumber provided. If you have 5 or more ranks of Craft (carpentry), you get a +2 synergy bonus on any Craft (Shipmaking) check.
	\item \textbf{Stonecarving:} You can use chisels and awls to carve general utilitarian shapes from stone. Such shapes include blades or other weapons, blocks of specific shapes, or tools. If you are literate, you can also carve letters or simple diagrams into stone. You are familiar with the properties of the various types of stone suitable for carving, and know where such types of stone may be found. A stonecarver may also work in a quarry to produce the blocks that will be used by stonemasons and sculptors. You cannot use this skill to carve artwork
	\item \textbf{Stonemasonry:} You are capable of constructing buildings and other objects from quarried blocks of stone. You know how to fit the blocks together and support what you build so that it can bear loads. If you have 5 or more ranks of Craft (stonecarving), you get a +2 synergy bonus on any Craft (stonemasonry) check.
	\item \textbf{Thatching:} You can take reeds, straw, fronds or other such materials and make them into a roof so as to keep the elements out of the structure.
	\item \textbf{Trapmaking:} This craft allows you to make traps and snares like you would use to capture or kill animals. The types of traps and snares you can make range from simple spring-loaded paw-traps to cages that allow entrance but not exit. In wilderness settings, this would allow you to make traps like branches that are pulled back with ropes and have spikes attached to them or like net traps. This does not include the complex mechanical devices built into objects (like locks and chests) and designed to cause damage to unauthorized accessors. If you have 5 or more ranks of Craft (machining), you get a +2 synergy bonus on any Craft (trapmaking) check. If you have 5 or more ranks in Knowledge (architecture and engineering) you can build entire rooms that are trapped and use counterweights to balance or set them.
	\item \textbf{Weaponsmithing:} You can use this skill to create metal weapons such as (but not limited to) swords and maces.
	\item \textbf{Weaving:} You can take various fibers (e.g. silk, wool, cotton, hair) and wave them together to make cloth, and are familiar with the use and maintenance of a loom. You can also make rugs and tapestries.
	\item \textbf{Woodcarving:} You can use hand tools such as hand axes and whittling knives to make useful or artistic items out of raw wood. You are familiar with the properties of various types of wood, and can make such items as staves, shields, wands, tools, utensils, masks, combs, furniture, holy symbols, and figureheads from wood.
\end{itemize*}

\textbf{Check:} You can practice your trade and make a decent living, earning about half your check result in gold pieces per week of dedicated work. You know how to use the tools of your trade, how to perform the craft's daily tasks, how to supervise untrained helpers, and how to handle common problems. (Untrained laborers and assistants earn an average of 1 silver piece per day.)

The basic function of the Craft skill, however, is to allow you to make an item of the appropriate type. The DC depends on the complexity of the item to be created. The DC, your check results, and the price of the item determine how long it takes to make a particular item. The item's finished price also determines the cost of raw materials.

In some cases, the \linkspell{Fabricate} spell can be used to achieve the results of a Craft check with no actual check involved. However, you must make an appropriate Craft check when using the spell to make articles requiring a high degree of craftsmanship.

A successful Craft check related to woodworking in conjunction with the casting of the \linkspell{Ironwood} spell enables you to make wooden items that have the strength of steel.

When casting the spell \linkspell{Minor Creation}, you must succeed on an appropriate Craft check to make a complex item.

All crafts require artisan's tools to give the best chance of success. If improvised tools are used, the check is made with a -2 circumstance penalty. On the other hand, masterwork artisan's tools provide a +2 circumstance bonus on the check.

To determine how much time and money it takes to make an item, follow these steps.

\begin{enumerate}
	\item Find the item's price. Put the price in silver pieces (1 gp = 10 sp).
	\item Find the DC from the table below.
	\item Pay one-third of the item's price for the cost of raw materials.
	\item Make an appropriate Craft check representing one week's work. If the check succeeds, multiply your check result by the DC. If the result x the DC equals the price of the item in sp, then you have completed the item. (If the result x the DC equals double or triple the price of the item in silver pieces, then you've completed the task in one-half or one-third of the time. Other multiples of the DC reduce the time in the same manner.) If the result x the DC doesn't equal the price, then it represents the progress you've made this week. Record the result and make a new Craft check for the next week. Each week, you make more progress until your total reaches the price of the item in silver pieces.
\end{enumerate}

If you fail a check by 4 or less, you make no progress this week.

If you fail by 5 or more, you ruin half the raw materials and have to pay half the original raw material cost again.

\textbf{Progress by the Day:} You can make checks by the day instead of by the week. In this case your progress (check result x DC) is in copper pieces instead of silver pieces.

\textbf{Creating Masterwork Items:} You can make a masterwork item---a weapon, suit of armor, shield, or tool that conveys a bonus on its use through its exceptional craftsmanship, not through being magical. To create a masterwork item, you create the masterwork component as if it were a separate item in addition to the standard item. The masterwork component has its own price (300 gp for a weapon or 150 gp for a suit of armor or a shield) and a Craft DC of 20. Once both the standard component and the masterwork component are completed, the masterwork item is finished. \textit{Note:} The cost you pay for the masterwork component is one-third of the given amount, just as it is for the cost in raw materials.

\textbf{Repairing Items:} Generally, you can repair an item by making checks against the same DC that it took to make the item in the first place. The cost of repairing an item is one-fifth of the item's price. 

When you use the Craft skill to make a particular sort of item, the DC for checks involving the creation of that item are typically as given on the following table.

\end{multicols}

\begin{multicolsbasictable}{l l c}

\textbf{Item} & \textbf{Craft Skill} & \textbf{Craft DC}\\
Acid & Alchemy\textsuperscript{1} & 15	\\
Alchemist's fire, smokestick, or tindertwig & Alchemy\textsuperscript{1} & 20\\
Antitoxin, sunrod, tanglefoot bag, or thunderstone & Alchemy\textsuperscript{1} & 25\\
Armor or shield  & Armorsmithing & 10 + AC bonus\\
Longbow or shortbow & Bowmaking & 12\\
Composite longbow or composite shortbow & Bowmaking & 15\\
\multicolumn{1}{p{10cm}}{\raggedright{}Composite longbow or composite shortbow with high strength rating} & Bowmaking & 15 + (2 x rating)\\
Crossbow & Weaponsmithing & 15\\
Simple melee or thrown weapon & Weaponsmithing & 12\\
Martial melee or thrown weapon & Weaponsmithing & 15\\
Exotic melee or thrown weapon & Weaponsmithing & 18\\
Mechanical trap & 	Trapmaking & Varies\textsuperscript{2}\\
Very simple item (wooden spoon) & Varies & 5\\
Typical item (iron pot) & Varies & 10\\
High-quality item (bell) & Varies & 15\\
Complex or superior item (lock) & Varies & 20\\
\multicolumn{3}{l}{\cellcolor{white}\textsuperscript{1} You must be a spellcaster to craft any of these items.}\\
\multicolumn{3}{l}{\textsuperscript{2} Traps have their own rules for construction.}\\
\end{multicolsbasictable}

\begin{multicols}{2}

\textbf{Action:} Does not apply. Craft checks are made by the day or week (see above).

\textbf{Try Again:} Yes, but each time you miss by 5 or more, you ruin half the raw materials and have to pay half the original raw material cost again.

You may voluntarily add +10 to the indicated DC to craft an item. This allows you to create the item more quickly (since you'll be multiplying this higher DC by your Craft check result to determine progress). You must decide whether to increase the DC before you make each weekly or daily check.

To make an item using Craft (alchemy), you must have alchemical equipment and be a spellcaster. If you are working in a city, you can buy what you need as part of the raw materials cost to make the item, but alchemical equipment is difficult or impossible to come by in some places. Purchasing and maintaining an alchemist's lab grants a +2 circumstance bonus on Craft (alchemy) checks because you have the perfect tools for the job, but it does not affect the cost of any items made using the skill.

\textbf{Synergy:} If you have 5 ranks in a Craft skill, you get a +2 bonus on \linkskill{Appraise} checks related to items made with that Craft skill.
