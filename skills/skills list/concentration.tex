\end{multicols}
%%%%%%%%%%%%%%%%%%%%%%%%%
\skillentry{Concentration}{(Con)}
%%%%%%%%%%%%%%%%%%%%%%%%%

\textbf{Check:} You must make a Concentration check whenever you might potentially be distracted (by taking damage, by harsh weather, and so on) while engaged in some action that requires your full attention. Such actions include casting a spell, concentrating on an active spell, directing a spell, using a spell-like ability, or using a skill that would provoke an attack of opportunity. In general, if an action wouldn't normally provoke an attack of opportunity, you need not make a Concentration check to avoid being distracted.

If the Concentration check succeeds, you may continue with the action as normal. If the check fails, the action automatically fails and is wasted. If you were in the process of casting a spell, the spell is lost. If you were concentrating on an active spell, the spell ends as if you had ceased concentrating on it. If you were directing a spell, the direction fails but the spell remains active. If you were using a spell-like ability, that use of the ability is lost. A skill use also fails, and in some cases a failed skill check may have other ramifications as well.

The table below summarizes various types of distractions that cause you to make a Concentration check. If the distraction occurs while you are trying to cast a spell, you must add the level of the spell you are trying to cast to the appropriate Concentration DC. If more than one type of distraction is present, make a check for each one; any failed Concentration check indicates that the task is not completed.

\begin{multicolsbasictable}{l p{12cm}}

\textbf{Concentration DC} & \textbf{Distraction}\\
10 + damage dealt & Damaged during the action.\textsuperscript{2}\\
10 + half of continuous & Taking continuous damage during the damage last dealt action.\textsuperscript{3}\\
Distracting spell's save DC & Distracted by nondamaging spell.\textsuperscript{4}\\
10 & Vigorous motion (on a moving mount, taking a bouncy wagon ride, in a small boat in rough water, belowdecks in a stormtossed ship).\\
15 & Violent motion (on a galloping horse, taking a very rough wagon ride, in a small boat in rapids, on the deck of a storm-tossed ship).\\
20 & Extraordinarily violent motion (earthquake).\\
15 & Entangled.\\
20 & Grappling or pinned. (You can cast only spells without somatic components for which you have any required material component in hand.)\\
5 & Weather is a high wind carrying blinding rain or sleet.\\
10 & Weather is wind-driven hail, dust, or debris.\\
Distracting spell's save DC & Weather caused by a spell, such as \linkspell{Storm of Vengeance}.\textsuperscript{4}\\
\multicolumn{2}{p{\linewidth}}{\cellcolor{white}\textsuperscript{1} If you are trying to cast, concentrate on, or direct a spell when the distraction occurs, add the level of the spell to the indicated DC.}\\
\multicolumn{2}{p{\linewidth}}{\cellcolor{white}\textsuperscript{2} Such as during the casting of a spell with a casting time of 1 round or more, or the execution of an activity that takes more than a single full-round action (such as Disable Device). Also, damage stemming from an attack of opportunity or readied attack made in response to the spell being cast (for spells with a casting time of 1 action) or the action being taken (for activities requiring no more than a full-round action).}\\
\multicolumn{2}{p{\linewidth}}{\cellcolor{white}\textsuperscript{3} Such as from \linkspell{Acid Arrow}.}\\
\multicolumn{2}{p{\linewidth}}{\cellcolor{white}\textsuperscript{4} If the spell allows no save, use the save DC it would have if it did allow a save.}\\
\end{multicolsbasictable}

\pagebreak

\begin{multicols}{2}

\textbf{Action:} None. Making a Concentration check doesn't take an action; it is either a free action (when attempted reactively) or part of another action (when attempted actively).

\textbf{Try Again:} Yes, though a success doesn't cancel the effect of a previous failure, such as the loss of a spell you were casting or the disruption of a spell you were concentrating on.

\textbf{Special:} You can use Concentration to cast a spell, use a spell-like ability, or use a skill defensively, so as to avoid attacks of opportunity altogether. This doesn't apply to other actions that might provoke attacks of opportunity.

The DC of the check is 15 (plus the spell's level, if casting a spell or using a spell-like ability defensively). If the Concentration check succeeds, you may attempt the action normally without provoking any attacks of opportunity. A successful Concentration check still doesn't allow you to take 10 on another check if you are in a stressful situation; you must make the check normally. If the Concentration check fails, the related action also automatically fails (with any appropriate ramifications), and the action is wasted, just as if your concentration had been disrupted by a distraction. 
