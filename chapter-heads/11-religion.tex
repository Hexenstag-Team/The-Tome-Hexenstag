%%%%%%%%%%%%%%%%%%%%%%%%%%%%%%%%%%%%%%%%%%%%%%%%%%
%%%%%%%%%%%%%%%%%%%%%%%%%%%%%%%%%%%%%%%%%%%%%%%%%%
\chapter{Religion}
%%%%%%%%%%%%%%%%%%%%%%%%%%%%%%%%%%%%%%%%%%%%%%%%%%
%%%%%%%%%%%%%%%%%%%%%%%%%%%%%%%%%%%%%%%%%%%%%%%%%%

The gods are many. A few, such as Pelor (god of the sun), have grand temples that sponsor mighty processions through the streets on high holy days. Others, such as Erythnul (god of slaughter), have temples only in hidden places or evil lands. While the gods most strongly make their presence felt through their clerics, they also have lay followers who more or less attempt to live up to their deities’ standards. The typical person has a deity whom he considers to be his patron. Still, it is only prudent to be respectful toward and even pray to other deities when the time is right. Before setting out on a journey, a follower of Pelor might leave a small sacrifice at a wayside shrine to Fharlanghn (god of roads) to improve his chances of having a safe journey. As long as one’s own deity is not at odds with the others in such an act of piety, such simple practices are common. In times of tribulation, however, some people recite dark prayers to evil deities. Such prayers are best muttered under one’s breath, lest others overhear

Deities rule the various aspects of human existence: good and evil, law and chaos, life and death, knowledge and nature. In addition, various nonhuman races have racial deities of their own. A character may not be a cleric of a racial deity unless he is of the right race, but he may worship such a deity and live according to that deity’s guidance. For a deity who is not tied to a particular race (such as Pelor), a cleric’s race is not an issue. Deities of certain monster types are identified in the Monster Manual. Many more deities than those described here or mentioned in the Monster Manual also exist.

%%%%%%%%%%%%%%%%%%%%%%%%%%%%%%%%%%%%%%%%%%%%%%%%%%
\section{Deities}
%%%%%%%%%%%%%%%%%%%%%%%%%%%%%%%%%%%%%%%%%%%%%%%%%%

For Full List see Deities \& Demigods

%%%
\subsubsection{Deities by Class}
%%%

\begin{smallbasictable}{Bard Gods}{p{2cm} p{2cm} p{5cm} p{5cm}}
\textbf{Name} & \textbf{Alignment} & \textbf{Domains} & \textbf{Portfolio}\\
\multicolumn{4}{c}{\textbf{The Faerûnian Pantheon}}\\
Corellon Larethian & Chaotic Good & \linkdomain{Chaos}, \linkdomain{Good}, \linkdomain{Protection}, \linkdomain{War} & Elves, Magic, Arts and Crafts, Music, War\\
Fharlanghn & True Neutral & \linkdomain{Luck}, \linkdomain{Protection}, \linkdomain{Travel} & Horizons, Distance, Travel, Roads\\
Olidammara & Chaotic Neutral & \linkdomain{Chaos}, \linkdomain{Luck}, \linkdomain{Trickery} & Rogues, Music, Revelry, Wine, Humor, Tricks\\
Pelor & Neutral Good & \linkdomain{Good}, \linkdomain{Healing}, \linkdomain{Strength}, \linkdomain{Sun} & Sun, Light, Strength, Healing\\
\multicolumn{4}{c}{\textbf{The Asgardian Pantheon}}\\
Balder & Neutral Good & \linkdomain{Good}, \linkdomain{Healing}, \linkdomain{Knowledge} & Beauty, Light, Music, Poetry, Rebirth\\
Freya & Neutral Good & \linkdomain{Air}, \linkdomain{Charm}, \linkdomain{Good}, \linkdomain{Magic} & Fertility, Love, Magic, Vanity\\
Hermod & Chaotic Neutral & \linkdomain{Chaos}, \linkdomain{Luck}, \linkdomain{Travel} & Luck, Communication, Freedom\\
Odin & Neutral Good & \linkdomain{Air}, \linkdomain{Knowledge}, \linkdomain{Magic}, \linkdomain{Travel}, \linkdomain{Trickery}, \linkdomain{War} & Knowledge, Magic, Supreme, War\\
\multicolumn{4}{c}{\textbf{The Olympian Pantheon}}\\
Aphrodite & Chaotic Good & \linkdomain{Chaos}, \linkdomain{Charm}, \linkdomain{Good} & Love, Beauty\\
Apollo & Chaotic Good & \linkdomain{Good}, \linkdomain{Healing}, \linkdomain{Knowledge}, \linkdomain{Magic}, \linkdomain{Sun} & Light, Prophecy, Music, Healing\\
Dionysus & Chaotic Neutral & \linkdomain{Chaos}, \linkdomain{Destruction}, \linkdomain{Madness} & Mirth, Madness, Wine, Fertility, Theater\\
\multicolumn{4}{c}{\textbf{The Pharaonic Pantheon}}\\
Bes & Chaotic Neutral & \linkdomain{Luck}, \linkdomain{Protection}, \linkdomain{Trickery} & Luck, Music, Protection\\
Hathor & Neutral Good & \linkdomain{Community}, \linkdomain{Good}, \linkdomain{Luck} & Love, Music, Dance, Moon, Fate, Motherhood\\
Isis & Neutral Good & \linkdomain{Good}, \linkdomain{Magic}, \linkdomain{Protection}, \linkdomain{Water} & Fertility, Magic, Marriage\\
Thoth & True Neutral & \linkdomain{Knowledge}, \linkdomain{Magic}, \linkdomain{Rune} & Knowledge, Wisdom, Learning\\
\end{smallbasictable}

\begin{smallbasictable}{Druid Gods}{p{2cm} p{2cm} p{5cm} p{5cm}}
\textbf{Name} & \textbf{Alignment} & \textbf{Domains} & \textbf{Portfolio}\\
\multicolumn{4}{c}{\textbf{The Faerûnian Pantheon}}\\
Ehlonna & Neutral Good & \linkdomain{Animal}, \linkdomain{Good}, \linkdomain{Plant}, \linkdomain{Sun} & Forests, Woodlands, Flora and Fauna, Fertility\\
Obad-Hai & True Neutral & \linkdomain{Air}, \linkdomain{Animal}, \linkdomain{Earth}, \linkdomain{Fire}, \linkdomain{Plant}, \linkdomain{Water} & Nature, Woodlands, Freedom, Hunting, Beasts\\
Pelor & Neutral Good & \linkdomain{Good}, \linkdomain{Healing}, \linkdomain{Strength}, \linkdomain{Sun} & Sun, Light, Strength, Healing\\
\multicolumn{4}{c}{\textbf{The Asgardian Pantheon}}\\
Frey & Neutral Good & \linkdomain{Air}, \linkdomain{Good}, \linkdomain{Plant}, \linkdomain{Sun} & Agriculture, Fertility, Harvest, Sun\\
Frigga & True Neutral & \linkdomain{Air}, \linkdomain{Animal}, \linkdomain{Community}, \linkdomain{Knowledge} & Birth, Fertility, Love\\
Odur & Chaotic Good & \linkdomain{Chaos}, \linkdomain{Fire}, \linkdomain{Sun} & Light, Sun, Travel\\
Skadi & True Neutral & \linkdomain{Destruction}, \linkdomain{Earth}, \linkdomain{Strength} & Earth, Mountains\\
Uller & Chaotic Neutral & \linkdomain{Chaos}, \linkdomain{Protection}, \linkdomain{Travel} & Archers, Hunting, Winter\\
\multicolumn{4}{c}{\textbf{The Olympian Pantheon}}\\
Artemis & Neutral Good & \linkdomain{Animal}, \linkdomain{Good}, \linkdomain{Plant}, \linkdomain{Sun} & Hunting, Wild Beasts, Childbirth, Dance\\
Demeter & True Neutral & \linkdomain{Earth}, \linkdomain{Plant}, \linkdomain{Protection} & Agriculture\\
Pan & Chaotic Neutral & \linkdomain{Animal}, \linkdomain{Chaos}, \linkdomain{Plant} & Nature, Passion, Shepherds, Mountains\\
Poseidon & Chaotic Neutral & \linkdomain{Chaos}, \linkdomain{Earth}, \linkdomain{Water} & Sea, Rivers, Earthquakes\\
\multicolumn{4}{c}{\textbf{The Pharaonic Pantheon}}\\
Apep & Neutral Evil & \linkdomain{Evil}, \linkdomain{Fire}, \linkdomain{Scalykind} & Evil, Fire, Serpents\\
Isis & Neutral Good & \linkdomain{Good}, \linkdomain{Magic}, \linkdomain{Protection}, \linkdomain{Water} & Fertility, Magic, Marriage\\
Ptah & Lawful Neutral & \linkdomain{Creation}, \linkdomain{Knowledge}, \linkdomain{Law}, \linkdomain{Travel} & Crafts, Knowledge, Secrets, Travel\\
Sobek & Lawful Evil & \linkdomain{Animal}, \linkdomain{Evil}, \linkdomain{Water} & Water, River Hazards, Crocodiles, Wetlands\\
\end{smallbasictable}

\begin{smallbasictable}{Monk Gods}{p{2cm} p{2cm} p{5cm} p{5cm}}
\textbf{Name} & \textbf{Alignment} & \textbf{Domains} & \textbf{Portfolio}\\
\multicolumn{4}{c}{\textbf{The Faerûnian Pantheon}}\\
Heironeous & Lawful Good & \linkdomain{Good}, \linkdomain{Law}, \linkdomain{War} & Valor, Chivalry, Justice, Honor, War, Daring\\
Hextor & Lawful Evil & \linkdomain{Destruction}, \linkdomain{Evil}, \linkdomain{Law}, \linkdomain{War} & Tyranny War, Discord, Massacres, Conflict, Fittness\\
St. Cuthbert & Lawful Neutral & \linkdomain{Destruction}, \linkdomain{Law}, \linkdomain{Protection}, \linkdomain{Strength} & Retribution, Common Sense, Wisdom, Zeal, Honesty, Truth, Discipline\\
\multicolumn{4}{c}{\textbf{The Asgardian Pantheon}}\\
Tyr & Lawful Neutral & \linkdomain{Law}, \linkdomain{Protection}, \linkdomain{War} & Courage, Trust, Strategy, Tactics, Writing\\
\multicolumn{4}{c}{\textbf{The Olympian Pantheon}}\\
Athena & Lawful Good & \linkdomain{Artifice}, \linkdomain{Community}, \linkdomain{Good}, \linkdomain{Knowledge}, \linkdomain{Law}, \linkdomain{War} & Wisdom, Crafts, Civilization, War\\
Nike & Lawful Neutral & \linkdomain{Law}, \linkdomain{Nobility}, \linkdomain{War} & Victory\\
\multicolumn{4}{c}{\textbf{The Pharaonic Pantheon}}\\
Anubis & Lawful Neutral & \linkdomain{Law}, \linkdomain{Magic}, \linkdomain{Repose} & Judgment, Death\\
Osiris & Lawful Good & \linkdomain{Air}, \linkdomain{Earth}, \linkdomain{Good}, \linkdomain{Law}, \linkdomain{Plant}, \linkdomain{Repose} & Harvest, Nature, Underworld\\
Ptah & Lawful Neutral & \linkdomain{Creation}, \linkdomain{Knowledge}, \linkdomain{Law}, \linkdomain{Travel} & Crafts, Knowledge, Secrets, Travel\\
Re-Horakhty & Lawful Good & \linkdomain{Glory}, \linkdomain{Good}, \linkdomain{Law}, \linkdomain{Nobility}, \linkdomain{Sun}, \linkdomain{War} & Nobility, Sun, Supreme, Vengeance\\
Sobek & Lawful Evil & \linkdomain{Animal}, \linkdomain{Evil}, \linkdomain{Water} & Water, River Hazards, Crocodiles, Wetlands\\
\end{smallbasictable}

\begin{smallbasictable}{Paladin Gods}{p{2cm} p{2cm} p{5cm} p{5cm}}
\textbf{Name} & \textbf{Alignment} & \textbf{Domains} & \textbf{Portfolio}\\
\multicolumn{4}{c}{\textbf{The Faerûnian Pantheon}}\\
Heironeous & Lawful Good & \linkdomain{Good}, \linkdomain{Law}, \linkdomain{War} & Valor, Chivalry, Justice, Honor, War, Daring\\
\multicolumn{4}{c}{\textbf{The Asgardian Pantheon}}\\
Heimdall & Lawful Good & \linkdomain{Good}, \linkdomain{Law}, \linkdomain{War} & Watchfulness, Sight, Hearing, Loyalty\\
\multicolumn{4}{c}{\textbf{The Olympian Pantheon}}\\
Athena & Lawful Good & \linkdomain{Artifice}, \linkdomain{Community}, \linkdomain{Good}, \linkdomain{Knowledge}, \linkdomain{Law}, \linkdomain{War} & Wisdom, Crafts, Civilization, War\\
\multicolumn{4}{c}{\textbf{The Pharaonic Pantheon}}\\
Osiris & Lawful Good & \linkdomain{Air}, \linkdomain{Earth}, \linkdomain{Good}, \linkdomain{Law}, \linkdomain{Plant}, \linkdomain{Repose} & Harvest, Nature, Underworld\\
Re-Horakhty & Lawful Good & \linkdomain{Glory}, \linkdomain{Good}, \linkdomain{Law}, \linkdomain{Nobility}, \linkdomain{Sun}, \linkdomain{War} & Nobility, Sun, Supreme, Vengeance\\
\end{smallbasictable}

\begin{smallbasictable}{Ranger Gods}{p{2cm} p{2cm} p{5cm} p{5cm}}
\textbf{Name} & \textbf{Alignment} & \textbf{Domains} & \textbf{Portfolio}\\
\multicolumn{4}{c}{\textbf{The Faerûnian Pantheon}}\\
Ehlonna & Neutral Good & \linkdomain{Animal}, \linkdomain{Good}, \linkdomain{Plant}, \linkdomain{Sun} & Forests, Woodlands, Flora and Fauna, Fertility\\
Obad-Hai & True Neutral & \linkdomain{Air}, \linkdomain{Animal}, \linkdomain{Earth}, \linkdomain{Fire}, \linkdomain{Plant}, \linkdomain{Water} & Nature, Woodlands, Freedom, Hunting, Beasts\\
Pelor & Neutral Good & \linkdomain{Good}, \linkdomain{Healing}, \linkdomain{Strength}, \linkdomain{Sun} & Sun, Light, Strength, Healing\\
\multicolumn{4}{c}{\textbf{The Asgardian Pantheon}}\\
Frey & Neutral Good & \linkdomain{Air}, \linkdomain{Good}, \linkdomain{Plant}, \linkdomain{Sun} & Agriculture, Fertility, Harvest, Sun\\
Heimdall & Lawful Good & \linkdomain{Good}, \linkdomain{Law}, \linkdomain{War} & Watchfulness, Sight, Hearing, Loyalty\\
Odur & Chaotic Good & \linkdomain{Chaos}, \linkdomain{Fire}, \linkdomain{Sun} & Light, Sun, Travel\\
Sif & Chaotic Good & \linkdomain{Chaos}, \linkdomain{Good}, \linkdomain{War} & War, Dueling\\
Skadi & True Neutral & \linkdomain{Destruction}, \linkdomain{Earth}, \linkdomain{Strength} & Earth, Mountains\\
Thor & Chaotic Good & \linkdomain{Chaos}, \linkdomain{Good}, \linkdomain{Protection}, \linkdomain{Strength}, \linkdomain{War}, \linkdomain{Weather} & Storms, Thunder, War\\
Tyr & Lawful Neutral & \linkdomain{Law}, \linkdomain{Protection}, \linkdomain{War} & Courage, Trust, Strategy, Tactics, Writing\\
Uller & Chaotic Neutral & \linkdomain{Chaos}, \linkdomain{Protection}, \linkdomain{Travel} & Archers, Hunting, Winter\\
\multicolumn{4}{c}{\textbf{The Olympian Pantheon}}\\
Apollo & Chaotic Good & \linkdomain{Good}, \linkdomain{Healing}, \linkdomain{Knowledge}, \linkdomain{Magic}, \linkdomain{Sun} & Light, Prophecy, Music, Healing\\
Artemis & Neutral Good & \linkdomain{Animal}, \linkdomain{Good}, \linkdomain{Plant}, \linkdomain{Sun} & Hunting, Wild Beasts, Childbirth, Dance\\
Athena & Lawful Good & \linkdomain{Artifice}, \linkdomain{Community}, \linkdomain{Good}, \linkdomain{Knowledge}, \linkdomain{Law}, \linkdomain{War} & Wisdom, Crafts, Civilization, War\\
Zeus & Chaotic Good & \linkdomain{Air}, \linkdomain{Chaos}, \linkdomain{Good}, \linkdomain{Nobility}, \linkdomain{Strength}, \linkdomain{Weather} & Sky, Air, Storms, Fate, Nobility\\
\multicolumn{4}{c}{\textbf{The Pharaonic Pantheon}}\\
Apep & Neutral Evil & \linkdomain{Evil}, \linkdomain{Fire}, \linkdomain{Scalykind} & Evil, Fire, Serpents\\
Bast & Chaotic Good & \linkdomain{Chaos}, \linkdomain{Destruction}, \linkdomain{Protection}, \linkdomain{Strength}, \linkdomain{War} & Cats, Vengeance, Protection, Punishment\\
Isis & Neutral Good & \linkdomain{Good}, \linkdomain{Magic}, \linkdomain{Protection}, \linkdomain{Water} & Fertility, Magic, Marriage\\
Nephthys & Chaotic Good & \linkdomain{Chaos}, \linkdomain{Good}, \linkdomain{Protection}, \linkdomain{Repose} & Death, Grief\\
Osiris & Lawful Good & \linkdomain{Air}, \linkdomain{Earth}, \linkdomain{Good}, \linkdomain{Law}, \linkdomain{Plant}, \linkdomain{Repose} & Harvest, Nature, Underworld\\
Sobek & Lawful Evil & \linkdomain{Animal}, \linkdomain{Evil}, \linkdomain{Water} & Water, River Hazards, Crocodiles, Wetlands\\
\end{smallbasictable}


\pagebreak
\pagebreak

\pagebreak