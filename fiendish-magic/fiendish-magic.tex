%%%%%%%%%%%%%%%%%%%%%%%%%%%%%%%%%%%%%%%%%%%%%%%%%%
\section{Fiendish Taint}
%%%%%%%%%%%%%%%%%%%%%%%%%%%%%%%%%%%%%%%%%%%%%%%%%%

Fiendish Powers are a taint upon the realms, and corrupts any who wish to use their power. As a fiend allows more of the energies of the Lower Planes to enter his body, transformations beyond his control alter his true form, forcing him to become an image of evil. The forms of the fiendish subraces are well known; balors and lemures are easily recognizable in order to better signify their roles in fiendish society, but those who embrace the quintessential nature of fiendish existence can exhibit a wild and wide variety of forms. Some are merely permutations on a theme, such as succubus who has embraced the Sphere of Violation might merely be an emotionally distant ice princess, while a bone devil who has wallowed in his Brutish nature may exhibit the addition of anorexic wings to accompany an acquired ability to fly and colorful markings along his claws that denote their mutation into poisonous implements. Other fiends offer no such frame of reference, such the totally unique Demon Princes, symbols of fiendish power and might.

Some mortals gain fiendish power by bloodline or transformation, and these beings find that they pay the same price as a \linkclass{True Fiend}. Having dabbled in fiendish power, most believed that they could embrace true evil without consequences; the truth is that fiendish power marks its bearers, displaying their nature and allegiance for all to see.

%%%
\subsubsection{Effects:}
%%%

Fiendish creature don't need to look like anything, A gelugon could look like a planetar and aside from a little player confusion at the beginning, then any combat or non-combat situation would play out the same. This is a purely flavor model, and it reinforces the choice a player has made to become a Fiend. In this way, he's getting the feeling of becoming a more powerful fiend, and the class and feat options aren't just another power option. Once he's leaving a trail of ooze or he's hiding his scales in the local pub, he's certain to feel like a fiend.

The system for these changes is simple: for every feat with the Fiend subtype you acquire, you choose one physical trait over and beyond any transformations these feats might grant or inflict, and for every Sphere you acquire you choose one mental trait. Feel free to create any new traits with your DM's permission, but make sure that they are both obvious and disturbing. Feel free to disallow the choice of a trait if it overlaps another trait (like fur covering a change in color on a set of limbs).

Physical traits can be hidden with the \linkskill{Disguise} skill or shape-changing magic or effects, and mental traits can be played off by using \linkskill{Bluff}. Once someone has noticed a trait, they may make a \linkskill{Knowledge} (Planes) check to identify that a person is a Fiend. How they react at that point is up to them.

\pagebreak

\begin{multicols}{2}

\begin{description*}
\item \textbf{Physical Traits}
\begin{itemize*}
\item Drips slime or excessive dust wherever he goes.
\item A pair of appendages are of different color from the rest of body.
\item Scales, fur, or loose/cracking skin cover arms, legs, chest, back, or face (no game effect).
\item Has a stylized wound that never heals.
\item Gains non-functional horns, spurs, or bone ridges.
\item Small hump on back.
\item Body is feverishly hot or cold.
\item Drools incessantly.
\item Breathe steams in any ambient temperature.
\item Extra joint in fingers and toes.
\item Sharpened teeth (enlarged canines, shark teeth, or some other style).
\item Alien eyes(change in color, shape, or composition).
\item Extra non-functional eye or eyes on face.
\item Altered body proportions(longer or shorts arms, legs torso, etc) or size(large head, fist, etc).
\item Apparently dying of a disease(choose a disease, but it has no game effect)
\item Limb is rotted and skeletal (no game effect).
\item Limbs have extra tendons
\item Gain non-functional wings.
\item Hands resembles spiders.
\item Smells of ash or brimstone.
\item Hole all the way through body in the center of torso.
\item Stand unnaturally still when not paying attention.
\item Cloven feet.
\item Tentacles for fingers.
\item Appears as if worms or bugs are crawling under skin.
\item Plants droop when held.
\item Extra ribs extend to pelvis.
\item Skin wrinkles when struck, and must be smoothed down.
\item When sleeping, runes and glyphs press up from under skin.
\item Holy symbols and holy water cause very mild burns(no game effect).
\end{itemize*}
\end{description*}

\begin{description*}
\item \textbf{Mental Traits}
\begin{itemize*}
\item Gaze lingers hungrily at acts associated with one's Spheres.
\item Cruel smile or laugh.
\item Distaste for the company of good-aligned people
\item Fondness for disturbing imagery.
\item Seems pained or ill in good-aligned temples.
\item Must taste the blood on his weapons after every kill.
\item Displays no emotion at atrocity or other's pain.
\item Obviously enjoys the taste of bloody meat.
\item Eat bugs when you think no one is watching.
\item Only laughs when seeing others in pain.
\item Must keep an unholy symbol in your possession, if possible.
\item Appears extremely avaricious in the presence of exposed wealth.
\item Becomes visibly excited when pain is inflicted on others.
\item Seems in awe of more powerful demons.
\item Must lie about one's past, but only about unimportant details.
\item Seems fearful and guilty in the presence of celestials.
\item Bathes as little as possible.
\item Must keep lair and possessions as clean as possible.
\item Seems angry and resentful when given a command.
\item Seems gloating and sarcastic when giving orders.
\item Seems bored when morality is discussed.
\item Seems happy during funerals and executions.
\item Leers at any woman of child-bearing age or older.
\item When confront by authority figures, always seems bitter.
\item Seems angry when charity or help is offered.
\item Seems to not understand respect for the dead.
\item Is rude when asked for help or charity.
\item Unable to show love.
\item Unable to express appreciation.
\end{itemize*}
\end{description*}

\end{multicols}

\pagebreak

%%%%%%%%%%%%%%%%%%%%%%%%%%%%%%%%%%%%%%%%%%%%%%%%%%
\section{Fiendish Spheres}
%%%%%%%%%%%%%%%%%%%%%%%%%%%%%%%%%%%%%%%%%%%%%%%%%%

Fiends (and some of their minions and associates) cast magic primarily through spell-like abilities. While many signature fiends have arbitrary lists of spell-like abilities, the Tome of Fiends offers a method to advance Fiends into thematically appropriate spell-like abilities when they advance. When a fiend has access to a sphere, she is able to use all of the abilities within that sphere up to her character level. If she gains more levels, more powers of the sphere become available. In this way the spell-like abilities of fiends created with the rules in this tome should always be æsthetically and level appropriate.

\textbf{Basic Sphere Access:} When a creature has basic access to a sphere, she can use any of the spells listed in the sphere once per day (each) as spell-like abilities, provided that their listed level is equal or lower to the creature's character level.

\textbf{Advanced Sphere Access:} When a creature has advanced access to a sphere, she can use any of the spells listed in the sphere 3 times per day (each) as spell-like abilities, provided that their listed level is equal or lower to the creature's character level.

\textbf{Expert Sphere Access:} When a creature has expert access to a sphere, any spells listed in the sphere may be used at will as spell-like abilities, provided that their listed level is equal or lower to the creature's character level.

\textbf{Creating new Spheres:} The following list of spheres isn't intended to be comprehensive, and we fully expect that some players and DMs will want many more spheres than we have scribed. All new spheres must be approved of by the DM, and should represent some actual (indifferent or evil) trait like "intoxication" or "badgers" rather than a game mechanical notion like "kicking ass and being totally sweet" or something praiseworthy like "generosity." A good place to start is actually Domains, as these are already a source by which a character gain a spell at every odd-numbered level.

\textbf{Spheres and Spell Levels:} Spell-like abilities used out of spheres are considered to be cast as a spell level equal to half the minimum needed character level to use the ability (rounded up). The save DC of a spell-like ability granted through Sphere access is Charisma-based. Thus, the save DC for a spell-like ability which becomes available at character level 5 is 13 + Charisma bonus.

\begin{multicols}{2}

\begin{description*}
\item \sphere{Bone}
\item \textbf{Special:} Any creature of 10 HD or less killed by one of your spelllike abilities rises as a zombie under your control, with no control limits.
	\item[1:] \linkspell{Command Undead}
	\item[3:] \linkspell{Desecrate}
	\item[5:] \linkspell{Animate Dead}
	\item[7:] \linkspell{Black Sand}
	\item[9:] \linkspell{Summon Undead V}
	\item[11:] \linkspell{Awaken Dread Warrior}
	\item[13:] \linkspell{Revive Undead}
	\item[15:] \linkspell{Awaken Undead}
	\item[17:] \linkspell{General of Undeath}
	\item[19:] \linkspell{Plague of Undead}
\end{description*}

\begin{description*}
\item \sphere{Bubbles}
\item \textbf{Special:} Three times per day, you may use the \linkfeat{Sculpt Spell} metamagic on any spell-like ability you can use, but only if you do not use the cone option of this metamagic feat.
	\item[1:] \linkspell{Flaming Sphere}
	\item[3:] \linkspell{Water Breathing}
	\item[5:] \linkspell{Magic Circle Against Good}
	\item[7:] \linkspell{Resilient Sphere}
	\item[9:] \linkspell{Binding}
	\item[11:] \linkspell{Telekinetic Sphere}
	\item[13:] \linkspell{Forcecage}
	\item[15:] \linkspell{Prismatic Sphere}
	\item[17:] \linkspell{Temporal Stasis}
	\item[19:] \linkspell{Time Stop}
\end{description*}

\begin{description*}
\item \sphere{Carnage}
\item \textbf{Special:} All of your damaging spell-like abilities do vile damage.
	\item[1:] \linkspell{Seething Eyebane}
	\item[3:] \linkspell{Blade of Fear and Pain}
	\item[5:] \linkspell{Lahm's Finger Darts}
	\item[7:] \linkspell{Blade Barrier}
	\item[9:] \linkspell{Fleshshiver}
	\item[11:] \linkspell{Disintigrate}
	\item[13:] \linkspell{Flensing}
	\item[15:] \linkspell{Horrid Wilting}
	\item[17:] \linkspell{Harm, Mass}
	\item[19:] \linkspell{Implosion}
\end{description*}

\begin{description*}
\item \sphere{Cold}
\item \textbf{Special:} You gain the [Cold] Subtype.
	\item[1:] \linkspell{Cone of Cold}
	\item[3:] \linkspell{Creeping Cold}
	\item[5:] \linkspell{Ice Storm}
	\item[7:] \linkspell{Wall of Ice}
	\item[9:] \linkspell{Freezing Sphere}
	\item[11:] \linkspell{Control Weather}
	\item[13:] \linkspell{Heat Drain}
	\item[15:] \linkspell{Freezing Fog}
	\item[17:] \linkspell{Fimbul Winter}
	\item[19:] \linkspell{Frostfell}
\end{description*}

\begin{description*}
\item \sphere{Death}
\item \textbf{Special:} You are immune to any magical death effect. You are also healed by negative energy like an undead creature (this does not interfere with any existing ability to be healed by positive energy).
	\item[1:] \linkspell{Death Knell}
	\item[3:] \linkspell{Ghoul Gauntlet}
	\item[5:] \linkspell{Vampire Touch}
	\item[7:] \linkspell{Enervation}
	\item[9:] \linkspell{Raise Dead}
	\item[11:] \linkspell{Symbol of Death}
	\item[13:] \linkspell{Finger of Death}
	\item[15:] \linkspell{Death Pact}
	\item[17:] \linkspell{Wail of the Banshee}
	\item[19:] \linkspell{Implosion}
\end{description*}

\begin{description*}
\item \sphere{Dominion}
\item \textbf{Special:} Once per month, you may make one of these effects permanent.
	\item[1:] \linkspell{Obscuring Mist}
	\item[3:] \linkspell{Web}
	\item[5:] \linkspell{Wall of Fire}
	\item[7:] \linkspell{Solid Fog}
	\item[9:] \linkspell{Black Tentacles}
	\item[11:] \linkspell{Programmed Illusion}
	\item[13:] \linkspell{Wall of Force}
	\item[15:] \linkspell{Teleport Circle}
	\item[17:] \linkspell{Gate} (Travel Version Only)
	\item[19:] \linkspell{Storm of Vengeance}
\end{description*}

\begin{description*}
\item \sphere{Fire}
\item \textbf{Special:} You gain the [Fire] subtype.
	\item[1:] \linkspell{fireball}
	\item[3:] \linkspell{scorching ray}
	\item[5:] \linkspell{firetrap}
	\item[7:] \linkspell{wall of fire}
	\item[9:] \linkspell{fireshield}
	\item[11:] \linkspell{incendiary cloud}
	\item[13:] \linkspell{blackfire}
	\item[15:] \linkspell{fire seeds}
	\item[17:] \linkspell{meteor swarm}
	\item[19:] \linkspell{flame strike} (this ability can be used as an immediate action).
\end{description*}

\begin{description*}
\item \sphere{Frostbite}
\item \textbf{Special:} Any creature taking cold damage from your spell-like abilities is frostbitten (treat as fatigued, which is removed when the damage is healed). If you have Frostburn, you may use those frostbite rules if you want.
	\item[1:] \linkspell{shivering touch}
	\item[3:] \linkspell{freezing ray} (as scorching ray, but cold subtyped and doing cold damage).
	\item[5:] \linkspell{shivering touch, greater}
	\item[7:] \linkspell{cone of cold}
	\item[9:] \linkspell{flesh to ice}
	\item[11:] \linkspell{entomb}
	\item[13:] \linkspell{flesh to ice, chained}
	\item[15:] \linkspell{frostfell}
	\item[17:] \linkspell{iceburg}
	\item[19:] \linkspell{soul bind} (uses a piece of ice to hold the soul)
\end{description*}

\begin{description*}
\item \sphere{Bone}
\item \textbf{Special:} Any
	\item[1:] \linkspell{One}
	\item[3:] \linkspell{One}
	\item[5:] \linkspell{One}
	\item[7:] \linkspell{One}
	\item[9:] \linkspell{One}
	\item[11:] \linkspell{One}
	\item[13:] \linkspell{One}
	\item[15:] \linkspell{One}
	\item[17:] \linkspell{One}
	\item[19:] \linkspell{One}
\end{description*}

\begin{description*}
\item \sphere{Bone}
\item \textbf{Special:} Any
	\item[1:] \linkspell{One}
	\item[3:] \linkspell{One}
	\item[5:] \linkspell{One}
	\item[7:] \linkspell{One}
	\item[9:] \linkspell{One}
	\item[11:] \linkspell{One}
	\item[13:] \linkspell{One}
	\item[15:] \linkspell{One}
	\item[17:] \linkspell{One}
	\item[19:] \linkspell{One}
\end{description*}

\begin{description*}
\item \sphere{Bone}
\item \textbf{Special:} Any
	\item[1:] \linkspell{One}
	\item[3:] \linkspell{One}
	\item[5:] \linkspell{One}
	\item[7:] \linkspell{One}
	\item[9:] \linkspell{One}
	\item[11:] \linkspell{One}
	\item[13:] \linkspell{One}
	\item[15:] \linkspell{One}
	\item[17:] \linkspell{One}
	\item[19:] \linkspell{One}
\end{description*}

\begin{description*}
\item \sphere{Bone}
\item \textbf{Special:} Any
	\item[1:] \linkspell{One}
	\item[3:] \linkspell{One}
	\item[5:] \linkspell{One}
	\item[7:] \linkspell{One}
	\item[9:] \linkspell{One}
	\item[11:] \linkspell{One}
	\item[13:] \linkspell{One}
	\item[15:] \linkspell{One}
	\item[17:] \linkspell{One}
	\item[19:] \linkspell{One}
\end{description*}

\begin{description*}
\item \sphere{Bone}
\item \textbf{Special:} Any
	\item[1:] \linkspell{One}
	\item[3:] \linkspell{One}
	\item[5:] \linkspell{One}
	\item[7:] \linkspell{One}
	\item[9:] \linkspell{One}
	\item[11:] \linkspell{One}
	\item[13:] \linkspell{One}
	\item[15:] \linkspell{One}
	\item[17:] \linkspell{One}
	\item[19:] \linkspell{One}
\end{description*}

\begin{description*}
\item \sphere{Bone}
\item \textbf{Special:} Any
	\item[1:] \linkspell{One}
	\item[3:] \linkspell{One}
	\item[5:] \linkspell{One}
	\item[7:] \linkspell{One}
	\item[9:] \linkspell{One}
	\item[11:] \linkspell{One}
	\item[13:] \linkspell{One}
	\item[15:] \linkspell{One}
	\item[17:] \linkspell{One}
	\item[19:] \linkspell{One}
\end{description*}

\begin{description*}
\item \sphere{Bone}
\item \textbf{Special:} Any
	\item[1:] \linkspell{One}
	\item[3:] \linkspell{One}
	\item[5:] \linkspell{One}
	\item[7:] \linkspell{One}
	\item[9:] \linkspell{One}
	\item[11:] \linkspell{One}
	\item[13:] \linkspell{One}
	\item[15:] \linkspell{One}
	\item[17:] \linkspell{One}
	\item[19:] \linkspell{One}
\end{description*}

\begin{description*}
\item \sphere{Bone}
\item \textbf{Special:} Any
	\item[1:] \linkspell{One}
	\item[3:] \linkspell{One}
	\item[5:] \linkspell{One}
	\item[7:] \linkspell{One}
	\item[9:] \linkspell{One}
	\item[11:] \linkspell{One}
	\item[13:] \linkspell{One}
	\item[15:] \linkspell{One}
	\item[17:] \linkspell{One}
	\item[19:] \linkspell{One}
\end{description*}

\begin{description*}
\item \sphere{Bone}
\item \textbf{Special:} Any
	\item[1:] \linkspell{One}
	\item[3:] \linkspell{One}
	\item[5:] \linkspell{One}
	\item[7:] \linkspell{One}
	\item[9:] \linkspell{One}
	\item[11:] \linkspell{One}
	\item[13:] \linkspell{One}
	\item[15:] \linkspell{One}
	\item[17:] \linkspell{One}
	\item[19:] \linkspell{One}
\end{description*}

\begin{description*}
\item \sphere{Bone}
\item \textbf{Special:} Any
	\item[1:] \linkspell{One}
	\item[3:] \linkspell{One}
	\item[5:] \linkspell{One}
	\item[7:] \linkspell{One}
	\item[9:] \linkspell{One}
	\item[11:] \linkspell{One}
	\item[13:] \linkspell{One}
	\item[15:] \linkspell{One}
	\item[17:] \linkspell{One}
	\item[19:] \linkspell{One}
\end{description*}




\end{multicols}




















\sphere{Bone}
\textbf{Special:} Any
\begin{description*}
	\item[1:] \linkspell{One}
	\item[3:] \linkspell{One}
	\item[5:] \linkspell{One}
	\item[7:] \linkspell{One}
	\item[9:] \linkspell{One}
	\item[11:] \linkspell{One}
	\item[13:] \linkspell{One}
	\item[15:] \linkspell{One}
	\item[17:] \linkspell{One}
	\item[19:] \linkspell{One}
\end{description*}

