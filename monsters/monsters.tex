%%%%%%%%%%%%%%%%%%%%%%%%%%%%%%%%%%%%%%%%%%%%%%%%%%
%%%%%%%%%%%%%%%%%%%%%%%%%%%%%%%%%%%%%%%%%%%%%%%%%%
\chapter{Monsters}
%%%%%%%%%%%%%%%%%%%%%%%%%%%%%%%%%%%%%%%%%%%%%%%%%%
%%%%%%%%%%%%%%%%%%%%%%%%%%%%%%%%%%%%%%%%%%%%%%%%%%

\section{Undead}

\textbf{Undead (type)}
Undead are once-living creatures animated by spiritual or supernatural forces.
\begin{itemize*}
\item Darkvision 60ft.
\item Immunity to poison, sleep effects, paralysis, stunning, disease, and death effects.
\item Not subject to critical hits, nonlethal damage, ability drain, or energy drain. Immune to damage to its physical ability scores (Strength, Dexterity, and Constitution), as well as to fatigue and exhaustion effects.
\item Cannot heal damage on its own if it has no Intelligence score, although it can be healed. Negative energy (such as an inflict spell) can heal undead creatures. The fast healing special quality works regardless of the creature’s Intelligence score.
\item Immunity to any effect that requires a Fortitude save (unless the effect also works on objects or is harmless).
\item Not affected by \linkspell{Raise Dead} and \linkspell{Reincarnate} spells or abilities. \linkspell{Resurrection} and \linkspell{True Resurrection} can affect undead creatures. Though they are forever undead unless \linkspell{Wish} or \linkspell{Miracle}.
\item Undead do not breathe, eat, or sleep.
\end{itemize*}

\subsection{Subtypes}

\textbf{Mindless (subtype)}
Undead creatures which doesn't have an intelligences to speak of.
\begin{itemize*}
\item Immune to mind affecting affects.
\item Immune to morale and fear effects.
\item Immune to  Bluff, Diplomacy, or Intimidate attempts.
\item When reduced to 0 hit points or less, it is immediately destroyed.
\item No Constitution or Intelligence score.
\item All Hit Dices become d12
\item Sample creatures with the [Mindless] subtype: Skeletons, and Zombies
\end{itemize*}


\textbf{Dark Minded (subtype)}
Undead creatures with an intelligence score have an intelligence that can be influenced, though they are dead and cannot be influenced by appeals to emotion. A dark minded creature has the following traits:
\begin{itemize*}
\item Immune to morale and fear effects.
\item Any Bluff, Diplomacy, or Intimidate attempts to influence a dark minded creature are made with a -10 penalty.
\item A Dark Minded creature continues to advance in age categories, growing older and wiser over time. It does not accrue any penalties to its attributes for advancing in age categories, and a Dark Minded creature has no maximum age.
\item Sample creatures with the [Dark Minded] subtype: Liches, Nightshades, and Vampires
\end{itemize*}

\textbf{Incorporeal (subtype)}
An incorporeal creature has no physical body.
\begin{itemize*}
\item No Constitution score.
\item All Hit Dices become d12
\item Can pass through solid objects on the Material Plane
\item Can only be harmed by other incorporeal creatures, magic weapons or creatures that strike as magic weapons, and spells, spell-like abilities, or supernatural abilities; It is immune to all nonmagical attack forms.
\item Has no Natural Armor bonus but has a Deflection bonus equal to its Charisma modifier
\item Incorporeal attacks pass through non-magical (ignore) Natural Armor, Armor, and Shields bonus.
\item Cannot take any physical action that would move or manipulate an opponent, its equipment or the world as a whole, nor are they subject to such actions.
\item No Strength score, use Dexterity modifier in it's place
\item Sample creatures with the [Incorporeal] subtype: Ghosts, Shadows, Spectres, and Wraiths.
\end{itemize*}

\textbf{Unliving (subtype)}
A Unliving creature is an undead that mimics many of the capacities of a living creature without truly being alive. An unliving creature has the following game effects:
\begin{itemize*}
\item Unliving creatures require food (often blood or flesh) and sleep, and are vulnerable to magical sleep effects even if they are otherwise immune to mind affecting effects.
\item Unliving creatures have at least one vital organ, and are subject to critical hits from attackers with at least one rank in Knowledge (Undead).
\item Subject to nonlethal damage, ability drain, or energy drain.
\item Sample creatures with the [Unliving] subtype: Ghouls, Necropolitan, Vampires, and Slaymate
\end{itemize*}

\subsection{Becoming Undead}
The basic rules for transforming into Undead were never intended to be playable by player characters. And thus it is unsurprising that the legions of the damned are not only unsatisfying, but actually unplayable when placed in a game. The following are templates that can be added to a character to make them into an Undead without actually changing their Level Adjustment. If a player wants to explore the legendary powers available to some of these creatures, they are encouraged to take Prestige Classes available to undead or to take one or more [Undead] feats that can grant the character these abilities within the normal level progression context. Each undead creature type has access to a special class that characters may take to advance their special abilities.

\textbf{Revenants}
A revenant is the victim of a murder driven to avenge their own death. A game master might allow a character to return from the dead as a revenant if their character died in a particularly unfair fashion or if their character had a lot left to do.
\begin{itemize*}
\item \textbf{Type:} The character's type changes to Undead and the character's former type becomes a subtype with the "augmented" modifier. The character also gains the Dark Minded subtype.
\item \textbf{Hit Dice:} The character's BAB, Saves, and skills are all unaffected. The character must reroll his Hit Points, but every hit die is a d12.
\item \textbf{Ability Scores:} The character loses his Constitution score.
\item \textbf{Alignment:} The character's alignment changes to Lawful.
\item \textbf{Special Qualities:} The character cannot be turned, but may be rebuked. The character heals completely at the setting of the sun, unless he is in a Tomb or hallowed area. This healing can even bring him back from destruction, but if his body is nailed to the ground (or in a Tomb or hallowed area), he can never come back from the dead by any means.
\end{itemize*}

\textbf{Vampires}
A vampire is an unliving mockery of life that lives by cruelly consuming the blood of the innocent. Only characters slain by a vampire's Constitution Drain rise as vampires, and even then only if they have 5 hit dice or more. Characters with less hit dice become monstrous vampire spawn and do not retain their abilities.
\begin{itemize*}
\item \textbf{Type:} The character's type changes to Undead and the character's former type becomes a subtype with the "augmented" modifier. The character also gains the Dark Minded and Unliving subtypes.
\item \textbf{Hit Dice:} The character's Hit Dice, BAB, Saves, and skills are all unaffected.
\item \textbf{Ability Scores:} The character gains a +2 bonus to his Strength and Charisma.
\item \textbf{Alignment:} The character's alignment changes to Evil.
\item \textbf{Special Attacks:} The character can drain blood from a helpless or willing victim, inflicting 2 points of Constitution Drain per round. The character heals 5 points for each point of Constitution drain in this way, and consuming 4 points of Constitution from intelligent creatures is considered enough "food" for one day (and the vampire gains no sustenance from any other food). Humanoids slain by this Constitution Drain may rise as vampires or vampire spawn (though the character has no control over them unless granted by another ability).
\item \textbf{Special Qualities:} The character gains Turn Resistance +2. The character suffers 2d6 damage and is considered staggered every round he is exposed to direct sunlight. This damage cannot be healed by any means until the character is in a place with no light at all (such as a coffin). A vampire character is vulnerable to Light effects.
\end{itemize*}
Vampires may take levels in Vampire Paragon.

\textbf{Ghouls}
Ghoul Fever is a horrifying illness that incites an almost insatiable craving for the flesh of humanoids. Characters with at least 2 class levels brought to zero Constitution by Ghoul Fever find their constitution restored and begin their unlife as Ghouls. Characters with less than 2 class levels simply die and rot.
\begin{itemize*}
\item \textbf{Type:} The character's type changes to Undead and the character's former type becomes a subtype with the "augmented" modifier. The character also gains the Dark Minded and Unliving subtypes.
\item \textbf{Hit Dice:} The character's Hit Dice, BAB, Saves, and skills are all unaffected.
\item \textbf{Ability Scores:} The character's Dexterity increases by +2.
\item \textbf{Alignment:} The character's alignment changes to Evil.
\item \textbf{Special Attacks:} The character gains a bite attack that inflicts an amount of damage appropriate to her size. She also is a carrier of Ghoul Fever.
\item \textbf{Special Qualities:} The character gains Turn Resistance of +2. The character cannot eat anything other than raw meat (vegetables or cooked foods are forcefully vomited up, leaving the character sickened for an hour), and her total dietary requirements are not reduced.
\end{itemize*}
Ghouls may take levels in Ghoul Paragon.

\textbf{Swordwraith}
Mercenaries devoted strongly enough to a life of war that they carry on in death their endless campaign of destruction. A character slain in battle may return as a Swordwraith if his services were hired under false pretenses or if his exploits were particularly impressive before his life finally ended (at the discretion of the DM).

Swordwraiths appear somewhat insubstantial and have faintly glowing eyes, but they are not truly incorporeal and their eyes do not produce enough light to modify vision penalties.
\begin{itemize*}
\item \textbf{Type:} The character's type changes to Undead and the character's former type becomes a subtype with the "augmented" modifier. The character also gains the Dark Minded and Unliving subtypes.
\item \textbf{Hit Dice:} The character's Hit Dice, BAB, Saves, and skills are all unaffected.
\item \textbf{Ability Scores:} The character gains a +2 bonus to his Hide and Move Silently checks.
\item \textbf{Alignment:} The character's alignment is unchanged.
\item \textbf{Special Qualities:} The character gains Turn Resistance +2.
\end{itemize*}
Swordwraiths may take levels in Swordwraith Paragon.