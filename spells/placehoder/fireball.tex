\spellentry{Fireball}

Evocation [Fire]

\textbf{Level:} Sor/Wiz 3

\textbf{Components:} V, S, M

\textbf{Casting Time:} 1 standard action

\textbf{Range:} Long (400 ft. + 40 ft./level)

\textbf{Area:} 20-ft.-radius spread

\textbf{Duration:} Instantaneous

\textbf{Saving Throw:} Reflex half

\textbf{Spell Resistance:} Yes

A \textit{fireball} spell is an explosion of flame that detonates with a low roar 
and deals 1d6 points of fire damage per caster level (maximum 10d6) to every creature 
within the area. Unattended objects also take this damage. The explosion creates 
almost no pressure.

You point your finger and determine the range (distance and height) at which the 
\textit{fireball} is to burst. A glowing, pea-sized bead streaks from the pointing 
digit and, unless it impacts upon a material body or solid barrier prior to attaining 
the prescribed range, blossoms into the \textit{fireball} at that point. (An early 
impact results in an early detonation.) If you attempt to send the bead through 
a narrow passage, such as through an arrow slit, you must "hit" the opening with 
a ranged touch attack, or else the bead strikes the barrier and detonates prematurely.

The \textit{fireball} sets fire to combustibles and damages objects in the area. 
It can melt metals with low melting points, such as lead, gold, copper, silver, 
and bronze. If the damage caused to an interposing barrier shatters or breaks through 
it, the \textit{fireball} may continue beyond the barrier if the area permits; 
otherwise it stops at the barrier just as any other spell effect does.

\textit{Material Component:} A tiny ball of bat guano and sulfur.

