\spellentry{Atonement}

Abjuration

\textbf{Level:} Clr 5, Drd 5

\textbf{Components:} V, S, M, F, DF, XP

\textbf{Casting Time:} 1 hour

\textbf{Range:} Touch

\textbf{Target:} Living creature touched

\textbf{Duration:} Instantaneous

\textbf{Saving Throw:} None

\textbf{Spell Resistance:} Yes

This spell removes the burden of evil acts or misdeeds from the subject. The creature 
seeking atonement must be truly repentant and desirous of setting right its misdeeds. 
If the atoning creature committed the evil act unwittingly or under some form of 
compulsion, Atonement operates normally at no cost to you. However, in 
the case of a creature atoning for deliberate misdeeds and acts of a knowing and 
willful nature, you must intercede with your deity (requiring you to expend 500 
XP) in order to expunge the subject's burden. Many casters first assign a subject 
of this sort a quest (see \linkspell{Geas/Quest}) or similar penance to determine 
whether the creature is truly contrite before casting the Atonement spell 
on its behalf.

Atonement may be cast for one of several purposes, depending on the version 
selected.

\textit{Reverse Magical Alignment Change:} If a creature has had its alignment 
magically changed, \textit{atonement} returns its alignment to its original status 
at no cost in experience points.

\textit{Restore Class:} A paladin who has lost her class features due to committing 
an evil act may have her paladinhood restored to her by this spell.

\textit{Restore Cleric or Druid Spell Powers:} A cleric or druid who has lost the 
ability to cast spells by incurring the anger of his or her deity may regain that 
ability by seeking \textit{atonement} from another cleric of the same deity or 
another druid. If the transgression was intentional, the casting cleric loses 500 
XP for his intercession. If the transgression was unintentional, he does not lose 
XP.

\textit{Redemption or Temptation:} You may cast this spell upon a creature of an 
opposing alignment in order to offer it a chance to change its alignment to match 
yours. The prospective subject must be present for the entire casting process. 
Upon completion of the spell, the subject freely chooses whether it retains its 
original alignment or acquiesces to your offer and changes to your alignment. No 
duress, compulsion, or magical influence can force the subject to take advantage 
of the opportunity offered if it is unwilling to abandon its old alignment. This 
use of the spell does not work on outsiders or any creature incapable of changing 
its alignment naturally.

Though the spell description refers to evil acts, atonement can also be 
used on any creature that has performed acts against its alignment, whether those 
acts are evil, good, chaotic, or lawful.

\textit{Note:} Normally, changing alignment is up to the player. This use of atonement simply offers a believable way for a character to change his or her alignment drastically, suddenly, and definitively.

\textit{Material Component:} Burning incense.

\textit{Focus:} In addition to your holy symbol or normal divine focus, you need 
a set of prayer beads (or other prayer device, such as a prayer wheel or prayer 
book) worth at least 500 gp.

\textit{XP Cost:} When cast for the benefit of a creature whose guilt was the result 
of deliberate acts, the cost to you is 500 XP per casting (see above).

