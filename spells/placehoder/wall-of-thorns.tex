\spellentry{Wall of Thorns}

Conjuration (Creation)

\textbf{Level:} Drd 5, Plant 5

\textbf{Components:} V, S

\textbf{Casting Time:} 1 standard action

\textbf{Range:} Medium (100 ft. + 10 ft./level)

\textbf{Effect:} Wall of thorny brush, up to one 10-ft. cube/level (S)

\textbf{Duration:} 10 min./level (D)

\textbf{Saving Throw:} None

\textbf{Spell Resistance:} No

A \textit{wall of thorns} spell creates a barrier of very tough, pliable, tangled 
brush bearing needle-sharp thorns as long as a human's finger. Any creature forced 
into or attempting to move through a \textit{wall of thorns} takes slashing damage 
per round of movement equal to 25 minus the creature's AC. Dexterity and dodge 
bonuses to AC do not count for this calculation. (Creatures with an Armor Class 
of 25 or higher, without considering Dexterity and dodge bonuses, take no damage 
from contact with the wall.)

You can make the wall as thin as 5 feet thick, which allows you to shape the wall 
as a number of 10-by-10-by-5-foot blocks equal to twice your caster level. This 
has no effect on the damage dealt by the thorns, but any creature attempting to 
break through takes that much less time to force its way through the barrier.

Creatures can force their way slowly through the wall by making a Strength check 
as a full-round action. For every 5 points by which the check exceeds 20, a creature 
moves 5 feet (up to a maximum distance equal to its normal land speed). Of course, 
moving or attempting to move through the thorns incurs damage as described above. 
A creature trapped in the thorns can choose to remain motionless in order to avoid 
taking any more damage.

Any creature within the area of the spell when it is cast takes damage as if it 
had moved into the wall and is caught inside. In order to escape, it must attempt 
to push its way free, or it can wait until the spell ends. Creatures with the ability 
to pass through overgrown areas unhindered can pass through a \textit{wall of thorns 
}at normal speed without taking damage.

A \textit{wall of thorns} can be breached by slow work with edged weapons. Chopping 
away at the wall creates a safe passage 1 foot deep for every 10 minutes of work. 
Normal fire cannot harm the barrier, but magical fire burns it away in 10 minutes.

Despite its appearance, a \textit{wall of thorns} is not actually a living plant, 
and thus is unaffected by spells that affect plants.

