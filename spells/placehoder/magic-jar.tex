\spellentry{Magic Jar}

Necromancy

\textbf{Level:} Sor/Wiz 5

\textbf{Components:} V, S, F

\textbf{Casting Time:} 1 standard action

\textbf{Range:} Medium (100 ft. + 10 ft./level)

\textbf{Target:} One creature

\textbf{Duration:} 1 hour/level or until you return to your body

\textbf{Saving Throw:} Will negates; see text

\textbf{Spell Resistance:} Yes

By casting \textit{magic jar}, you place your soul in a gem or large crystal (known 
as the \textit{magic jar}), leaving your body lifeless. Then you can attempt to 
take control of a nearby body, forcing its soul into the \textit{magic jar}. You 
may move back to the jar (thereby returning the trapped soul to its body) and attempt 
to possess another body. The spell ends when you send your soul back to your own 
body, leaving the receptacle empty.

To cast the spell, the \textit{magic jar} must be within spell range and you must 
know where it is, though you do not need line of sight or line of effect to it. 
When you transfer your soul upon casting, your body is, as near as anyone can tell, 
dead. 

While in the \textit{magic jar}, you can sense and attack any life force within 
10 feet per caster level (and on the same plane of existence). You do need line 
of effect from the jar to the creatures. You cannot determine the exact creature 
types or positions of these creatures. In a group of life forces, you can sense 
a difference of 4 or more Hit Dice between one creature and another and can determine 
whether a life force is powered by positive or negative energy. (Undead creatures 
are powered by negative energy. Only sentient undead creatures have, or are, souls.)

You could choose to take over either a stronger or a weaker creature, but which 
particular stronger or weaker creature you attempt to possess is determined randomly.

Attempting to possess a body is a full-round action. It is blocked by \textit{protection 
from evil} or a similar ward. You possess the body and force the creature's soul 
into the \textit{magic jar} unless the subject succeeds on a Will save. Failure 
to take over the host leaves your life force in the \textit{magic jar}, and the 
target automatically succeeds on further saving throws if you attempt to possess 
its body again.

If you are successful, your life force occupies the host body, and the host's life 
force is imprisoned in the \textit{magic jar}. You keep your Intelligence, Wisdom, 
Charisma, level, class, base attack bonus, base save bonuses, alignment, and mental 
abilities. The body retains its Strength, Dexterity, Constitution, hit points, 
natural abilities, and automatic abilities. A body with extra limbs does not allow 
you to make more attacks (or more advantageous two-weapon attacks) than normal. 
You can't choose to activate the body's extraordinary or supernatural abilities. 
The creature's spells and spell-like abilities do not stay with the body.

As a standard action, you can shift freely from a host to the \textit{magic jar 
}if within range, sending the trapped soul back to its body. The spell ends when 
you shift from the jar to your own body.

If the host body is slain, you return to the \textit{magic jar}, if within range, 
and the life force of the host departs (it is dead). If the host body is slain 
beyond the range of the spell, both you and the host die. Any life force with nowhere 
to go is treated as slain.

If the spell ends while you are in the \textit{magic jar}, you return to your body 
(or die if your body is out of range or destroyed). If the spell ends while you 
are in a host, you return to your body (or die, if it is out of range of your current 
position), and the soul in the \textit{magic jar} returns to its body (or dies 
if it is out of range). Destroying the receptacle ends the spell, and the spell 
can be dispelled at either the \textit{magic jar} or at the host's location.

\textit{Focus:} A gem or crystal worth at least 100 gp.

