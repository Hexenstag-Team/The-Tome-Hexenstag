\spellentry{Antimagic Field}

Abjuration

\textbf{Level:} Clr 8, Magic 6, Protection 6, Sor/Wiz 6

\textbf{Components:} V, S, M/DF

\textbf{Casting Time:} 1 standard action

\textbf{Range:} 10 ft.

\textbf{Area:} 10-ft.-radius emanation, centered on you

\textbf{Duration:} 10 min./level (D)

\textbf{Saving Throw:} None

\textbf{Spell Resistance:} See text

An invisible barrier surrounds you and moves with you. The space within this barrier 
is impervious to most magical effects, including spells, spell-like abilities, 
and supernatural abilities. Likewise, it prevents the functioning of any magic 
items or spells within its confines.

An antimagic field suppresses any spell or magical effect used within, 
brought into, or cast into the area, but does not dispel it. Time spent within 
an antimagic field counts against the suppressed spell's duration.

Summoned creatures of any type and incorporeal undead wink out if they enter an 
antimagic field. They reappear in the same spot once the field goes away. 
Time spent winked out counts normally against the duration of the conjuration that 
is maintaining the creature. If you cast Antimagic Field in an area occupied 
by a summoned creature that has spell resistance, you must make a caster level 
check (1d20 + caster level) against the creature's spell resistance to make it 
wink out. (The effects of instantaneous conjurations are not affected 
by an antimagic field because the conjuration itself is no longer in effect, 
only its result.)

A normal creature can enter the area, as can normal missiles. Furthermore, while 
a magic sword does not function magically within the area, it is still a sword 
(and a masterwork sword at that). The spell has no effect on golems and other constructs 
that are imbued with magic during their creation process and are thereafter self-supporting 
(unless they have been summoned, in which case they are treated like any other 
summoned creatures). Elementals, corporeal undead, and outsiders are likewise unaffected 
unless summoned. These creatures' spell-like or supernatural abilities, however, 
may be temporarily nullified by the field. \linkspell{Dispel Magic} does not remove 
the field.

Two or more antimagic fields sharing any of the same space have no effect 
on each other. Certain spells, such as \linkspell{Wall of Force}, \linkspell{Prismatic Sphere}, 
and \linkspell{Prismatic Wall}, remain unaffected by antimagic field (see 
the individual spell descriptions). Artifacts and deities are unaffected by mortal 
magic such as this. 

Should a creature be larger than the area enclosed by the barrier, any part of 
it that lies outside the barrier is unaffected by the field.

\textit{Arcane Material Component:} A pinch of powdered iron or iron filings.

