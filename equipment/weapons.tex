%%%%%%%%%%%%%%%%%%%%%%%%%%%%%%%%%%%%%%%%%%%%%%%%%%
\section{Weapons}
%%%%%%%%%%%%%%%%%%%%%%%%%%%%%%%%%%%%%%%%%%%%%%%%%%

%%%%%%%%%%%%%%%%%%%%%%%%%
\subsection{Weapon Categories}
%%%%%%%%%%%%%%%%%%%%%%%%%

Weapons are grouped into several interlocking sets of categories.

These categories pertain to what training is needed to become proficient in a weapon's use (simple, martial, or exotic), the weapon's usefulness either in close combat (melee) or at a distance (ranged, which includes both thrown and projectile weapons), its relative encumbrance (light, one-handed, or two-handed), and its size (Small,  Medium, or Large).

\textbf{Simple, Martial, and Exotic Weapons:} A character who uses a weapon with which he or she is not proficient takes a -4 penalty on attack rolls.

\textbf{Melee and Ranged Weapons:} Melee weapons are used for making melee attacks, though some of them can be thrown as well. Ranged weapons are thrown weapons or projectile weapons that are not effective in melee.

\textit{Reach Weapons:} Glaives, guisarmes, lances, longspears, ranseurs, spiked chains, and whips are reach weapons. A reach weapon is a melee weapon that allows its wielder to strike at targets that aren't adjacent to him or her. Most reach weapons double the wielder's natural reach, meaning that a typical Small or Medium wielder of such a weapon can attack a creature 10 feet away, but not a creature in an adjacent square. A typical Large character wielding a reach weapon of the appropriate size can attack a creature 15 or 20 feet away, but not adjacent creatures or creatures up to 10 feet away.

\textit{Double Weapons:} Dire flails, dwarven urgroshes, gnome hooked hammers, orc double axes, quarterstaffs, and two-bladed swords are double weapons. A character can fight with both ends of a double weapon as if fighting with two weapons, but he or she incurs all the normal attack penalties associated with two-weapon combat, just as though the character were wielding a one-handed weapon and a light weapon.

The character can also choose to use a double weapon two handed, attacking with only one end of it. A creature wielding a double weapon in one hand can't use it as a double weapon -- only one end of the weapon can be used in any given round.

\textit{Thrown Weapons:} Daggers, clubs, shortspears, spears, darts, javelins, throwing axes, light hammers, tridents, shuriken, and nets are thrown weapons. The wielder applies his or her Strength modifier to damage dealt by thrown weapons (except for splash weapons). It is possible to throw a weapon that isn't designed to be thrown (that is, a melee weapon that doesn't have a numeric entry in the Range Increment column on Table: Weapons), but a character who does so takes a -4 penalty on the attack roll. Throwing a light or one-handed weapon is a standard action, while throwing a two-handed weapon is a full-round action. Regardless of the type of weapon, such an attack scores a threat only on a natural roll of 20 and deals double damage on a critical hit. Such a weapon has a range increment of 10 feet.

\textit{Projectile Weapons:} Light crossbows, slings, heavy crossbows, shortbows, composite shortbows, longbows, composite longbows, hand crossbows, and repeating crossbows are projectile weapons. Most projectile weapons require two hands to use (see specific weapon descriptions). A character gets no Strength bonus on damage rolls with a projectile weapon unless it's a specially built composite shortbow, specially built composite longbow, or sling. If the character has a penalty for low Strength, apply it to damage rolls when he or she uses a bow or a sling.

\textit{Ammunition:} Projectile weapons use ammunition: arrows (for bows), bolts (for crossbows), or sling bullets (for slings). When using a bow, a character can draw ammunition as a free action; crossbows and slings require an action for reloading. Generally speaking, ammunition that hits its target is destroyed or rendered useless, while normal ammunition that misses has a 50\% chance of being destroyed or lost.

Although they are thrown weapons, shuriken are treated as ammunition for the purposes of drawing them, crafting masterwork or otherwise special versions of them (see Masterwork Weapons), and what happens to them after they are thrown.

\vspace{12pt}
\textbf{Light, One-Handed, and Two-Handed Melee Weapons:} This designation is a measure of how much effort it takes to wield a weapon in combat. It indicates whether a melee weapon, when wielded by a character of the weapon's size category, is considered a light weapon, a one-handed weapon, or a two-handed weapon.

\textit{Light:} A light weapon is easier to use in one's off hand than a one-handed weapon is, and it can be used while grappling. A light weapon is used in one hand. Add the wielder's Strength bonus (if any) to damage rolls for melee attacks with a light weapon if it's used in the primary hand, or one-half the wielder's Strength bonus if it's used in the off hand. Using two hands to wield a light weapon gives no advantage on damage; the Strength bonus applies as though the weapon were held in the wielder's primary hand only.

An unarmed strike is always considered a light weapon.

\textit{One-Handed:} A one-handed weapon can be used in either the primary hand or the off hand. Add the wielder's Strength bonus to damage rolls for melee attacks with a one-handed weapon if it's used in the primary hand, or 1/2 his or her Strength bonus if it's used in the off hand. If a one-handed weapon is wielded with two hands during melee combat, add 1.5 times the character's Strength bonus to damage rolls.

\textit{Two-Handed:} Two hands are required to use a two-handed melee weapon effectively. Apply 1.5 times the character's Strength bonus to damage rolls for melee attacks with such a weapon. 

\vspace{12pt}
\textbf{Weapon Size:} Every weapon has a size category. This designation indicates the size of the creature for which the weapon was designed.

A weapon's size category isn't the same as its size as an object. Instead, a weapon's size category is keyed to the size of the intended wielder. In general, a light weapon is an object two size categories smaller than the wielder, a one-handed weapon is an object one size category smaller than the wielder, and a two-handed weapon is an object of the same size category as the wielder.

\textit{Inappropriately Sized Weapons:} A creature can't make optimum use of a weapon that isn't properly sized for it. A cumulative -2 penalty applies on attack rolls for each size category of difference between the size of its intended wielder and the size of its actual wielder. If the creature isn't proficient with the weapon a -4 nonproficiency penalty also applies.

The measure of how much effort it takes to use a weapon (whether the weapon is designated as a light, one-handed, or two-handed weapon for a particular wielder) is altered by one step for each size category of difference between the wielder's size and the size of the creature for which the weapon was designed. If a weapon's designation would be changed to something other than light, one-handed, or two-handed by this alteration, the creature can't wield the weapon at all.

\vspace{12pt}
\textbf{Improvised Weapons:} Sometimes objects not crafted to be weapons nonetheless see use in combat. Because such objects are not designed for this use, any creature that uses one in combat is considered to be nonproficient with it and takes a -4 penalty on attack rolls made with that object. To determine the size category and appropriate damage for an improvised weapon, compare its relative size and damage potential to the weapon list to find a reasonable match. An improvised weapon scores a threat on a natural roll of 20 and deals double damage on a critical hit. An improvised thrown weapon has a range increment of 10 feet.

%%%%%%%%%%%%%%%%%%%%%%%%%
\subsection{Weapon Qualities}
%%%%%%%%%%%%%%%%%%%%%%%%%

Here is the format for weapon entries (given as column headings on Table: Weapons, below).

\textbf{Cost:} This value is the weapon's cost in gold pieces (gp) or silver pieces (sp). The cost includes miscellaneous gear that goes with the weapon.

This cost is the same for a Small or Medium version of the weapon. A Large version costs twice the listed price.

\textbf{Damage:} The Damage columns give the damage dealt by the weapon on a successful hit. If two damage ranges are given then the weapon is a double weapon. Use the second damage figure given for the double weapon's extra attack.

\vspace{12pt}
\textbf{Critical:} The entry in this column notes how the weapon is used with the rules for critical hits. When your character scores a critical hit, roll the damage two, three, or four times, as indicated by its critical multiplier (using all applicable modifiers on each roll), and add all the results together.

\textit{Exception:} Extra damage over and above a weapon's normal damage is not multiplied when you score a critical hit.

\textit{x2:} The weapon deals double damage on a critical hit.

\textit{x3:} The weapon deals triple damage on a critical hit.

\textit{x3/x4:} One head of this double weapon deals triple damage on a critical hit. The other head deals quadruple damage on a critical hit.

\textit{x4:} The weapon deals quadruple damage on a critical hit.

\textit{19-20/x2:} The weapon scores a threat on a natural roll of 19 or 20 (instead of just 20) and deals double damage on a critical hit. (The weapon has a threat range of 19-20.)

\textit{18-20/x2:} The weapon scores a threat on a natural roll of 18, 19, or 20 (instead of just 20) and deals double damage on a critical hit. (The weapon has a threat range of 18-20.)

\textbf{Range Increment:} Any attack at less than this distance is not penalized for range. However, each full range increment imposes a cumulative -2 penalty on the attack roll. A thrown weapon has a maximum range of five range increments. A projectile weapon can shoot out to ten range increments.

\textbf{Weight:} This column gives the weight of a Medium version of the weapon. Halve this number for Small weapons and double it for Large weapons.

\textbf{Type:} Weapons are classified according to the type of damage they deal: bludgeoning, piercing, or slashing. Some monsters may be resistant or immune to attacks from certain types of weapons.

Some weapons deal damage of multiple types. If a weapon is of two types, the damage it deals is not half one type and half another; all of it is both types. Therefore, a creature would have to be immune to both types of damage to ignore any of the damage from such a weapon.

In other cases, a weapon can deal either of two types of damage. In a situation when the damage type is significant, the wielder can choose which type of damage to deal with such a weapon.

\textbf{Special:} Some weapons have special features. See the weapon descriptions for details.

\end{multicols}

%%%%%%%%%%%%%%%%%%%%%%%%%
\subsection{Weapon Descriptions}
%%%%%%%%%%%%%%%%%%%%%%%%%

\begin{smallbasictable}{Simple Weapons}{l c c c c c c l}
\textbf{Weapon} & \textbf{Cost} & \textbf{Dmg} & \textbf{Critical} & \textbf{Range} & \textbf{Weight} & \textbf{Type\textsuperscript{1}} & \multicolumn{1}{p{1cm}}{\raggedright{}\textbf{Weapon Size\textsuperscript{2}}}\\

\multicolumn{8}{l}{\textbf{Unarmed Attacks}}\\
\hspace{.5cm}Gauntlet & 2gp & 1d3 & x2 &  & 1 lb & Bludgeoning & Diminutive\\
\hspace{.5cm}Unarmed Strike &  &  1d3\textsuperscript{N} & x2 &  &  & Bludgeoning & Diminutive\\

\multicolumn{8}{l}{\textbf{Diminutive Weapons}}\\
\hspace{.5cm}Dagger & 2gp & 1d4 & 19-20/x2 & 10ft & 1 lb & \multicolumn{1}{p{2.1cm}}{\raggedright{}Piercing or Slashing} & Diminutive\\
\hspace{.5cm}Punching Dagger & 1d3 & 1d4 & x3 &  & 1 lb & Piercing & Diminutive\\
\hspace{.5cm}Spiked Gauntlet & 5gp & 1d4 & x2 &  & 1 lb & Piercing & Diminutive\\

\multicolumn{8}{l}{\textbf{Tiny Weapons}}\\
\hspace{.5cm}Light Mace & 5gp & 1d6 & x2 &  & 4 lb & Bludgeoning & Tiny\\
\hspace{.5cm}Sickle & 6gp & 1d6 & x2 &  & 2 lb & Slashing & Tiny\\
\hspace{.5cm}Shortspear & 1gp & 1d6 & x2 & 20ft & 3 lb & Piercing & Tiny\\

\multicolumn{8}{l}{\textbf{Small Weapons}}\\
\hspace{.5cm}Club &  & 1d6 & x2 & 10ft & 3 lb & Bludgeoning & Small\\
\hspace{.5cm}Heavy Mace & 12gp & 1d8 & x2 &  & 8 lb & Bludgeoning & Small\\
\hspace{.5cm}Morningstar & 8gp & 1d8 & x2 &  & 6 lb & \multicolumn{1}{p{2.1cm}}{\raggedright{}Bludgeoning and Piercing} & Small\\
\hspace{.5cm}Spear & 2gp & 1d8 & x3 & 20ft & 6 lb & Piercing & Small\\

\multicolumn{8}{l}{\textbf{Medium Weapons}}\\
\hspace{.5cm}Longspear \textsuperscript{R} & 5gp & 1d8 & x3 &  & 9 lb & Piercing & Medium\\
\hspace{.5cm}Quarterstaff \textsuperscript{D} &  & 1d6/1d6 & x2 &  & 4 lb & Bludgeoning & Medium\\

\multicolumn{8}{p{17cm}}{\cellcolor{white}\textsuperscript{1} When two types are given, the weapon is both types if the entry specifies "and", or either type (player's choice at time of attack) if the entry specifies "or".}\\
\multicolumn{8}{p{17cm}}{\cellcolor{white}\textsuperscript{2} Weapons can be made up or down in size, but it's uncommon within civilization.}\\
\multicolumn{8}{l}{\cellcolor{white}\textsuperscript{N} The weapon deals nonlethal damage rather than lethal damage.}\\
\multicolumn{8}{l}{\cellcolor{white}\textsuperscript{R} Reach weapon.}\\
\multicolumn{8}{l}{\cellcolor{white}\textsuperscript{D} Double weaopn.}\\
\end{smallbasictable}

\begin{smallbasictable}{Martial Weapons}{l c c c c c c l}
\textbf{Weapon} & \textbf{Cost} & \textbf{Dmg} & \textbf{Critical} & \textbf{Range} & \textbf{Weight} & \textbf{Type\textsuperscript{1}} & \multicolumn{1}{p{1cm}}{\raggedright{}\textbf{Weapon Size\textsuperscript{2}}}\\

\multicolumn{8}{l}{\textbf{Tiny Weapons}}\\
\hspace{.5cm}Handaxe & 6gp & 1d6 & x3 &  & 3 lb & Slashing & Tiny\\
\hspace{.5cm}Kukri & 8gp & 1d4 & 18-20/x2 &  & 2 lb & Slashing & Tiny\\
\hspace{.5cm}Light Hammer & 1gp & 1d4 & x2 & 20ft & 2 lb & Bludgeoning & Tiny\\
\hspace{.5cm}Light Pick & 4gp & 1d4 & x4 &  & 3 lb & Piercing & Tiny\\
\hspace{.5cm}Light Shield & special & 1d3 & x2 &  & special & Bludgeoning & Tiny\\
\hspace{.5cm}Light Spiked Shield & special & 1d4 & x2 &  & special & Piercing & Tiny\\
\hspace{.5cm}Sap & 1gp & 1d6\textsuperscript{N} & x2 &  & 2 lb & Bludgeoning & Tiny\\
\hspace{.5cm}Short Sword & 10gp & 1d6 & 19-20/x2 &  & 2 lb & Piercing & Tiny\\
\hspace{.5cm}Spiked Armor & 50gp & 1d6 & x2 &  & special & Piercing & Tiny\\
\hspace{.5cm}Throwing Axe & 8gp & 1d6 & x2 & 10ft & 2 lb & Slashing & Tiny\\

\multicolumn{8}{l}{\textbf{Small Weapons}}\\
\hspace{.5cm}Battleaxe & 10gp & 1d8 & x3 &  & 6 lb & Slashing & Small\\
\hspace{.5cm}Flail & 8gp & 1d8 & x2 &  & 5 lb & Bludgeoning & Small\\
\hspace{.5cm}Heavy Pick & 8gp & 1d6 & x4 &  & 6 lb & Piercing & Small\\
\hspace{.5cm}Heavy Shield & special & 1d4 & x2 &  & special & Bludgeoning & Small\\
\hspace{.5cm}Heavy Spiked Shield & special & 1d6 & x2 &  & special & Piercing & Small\\
\hspace{.5cm}Longsword & 15gp & 1d8 & 19-20/x2 &  & 4 lb & Slashing & Small\\
\hspace{.5cm}Rapier & 20gp & 1d6 & 18-20/x2 &  & 2 lb & Piercing & Small\\
\hspace{.5cm}Scimitar & 15gp & 1d6 & 18-20/x2 &  & 4 lb & Slashing & Small\\
\hspace{.5cm}Trident & 15gp & 1d8 & x2 & 10ft & 4 lb & Piercing & Small\\
\hspace{.5cm}Warhammer & 12gp & 1d8 & x3 &  & 5 lb & Bludgeoning & Small\\

\multicolumn{8}{l}{\textbf{Medium Weapons}}\\
\hspace{.5cm}Kriegsmesser & 75gp & 2d4 & 18-20/x2 &  & 8 lb & Slashing & Medium\\
\hspace{.5cm}Glaive \textsuperscript{R} & 8gp & 1d10 & x3 &  & 10 lb & Slashing & Medium\\
\hspace{.5cm}Greataxe & 20gp & 1d12 & x3 &  & 12 lb & Slashing & Medium\\
\hspace{.5cm}Greatclub & 5gp& 1d10 & x2 &  & 8 lb & Bludgeoning & Medium\\
\hspace{.5cm}Greatsword & 50gp & 2d6 & 19-20/x2 &  & 12 lb & Slashing & Medium\\
\hspace{.5cm}Guisame \textsuperscript{R} & 9gp & 2d4 & x3 &  & 12 lb & Slashing & Medium\\
\hspace{.5cm}Halberd & 10gp & 1d10 & x3 &  & 12 lb & \multicolumn{1}{p{2.1cm}}{\raggedright{}Piercing or Slashing} & Medium\\
\hspace{.5cm}Heavy Flail & 15gp & 1d10 & 19-20/x2 &  & 8 lb & Bludgeoning & Medium\\
\hspace{.5cm}Lance \textsuperscript{R} & 10gp & 1d8 & x3 &  & 10 lb & Slashing & Medium\\
\hspace{.5cm}Ranseur \textsuperscript{R} & 10gp & 2d4 & x3 &  & 12 lb & Slashing & Medium\\
\hspace{.5cm}Scythe & 18gp & 2d4 & x4 &  & 10 lb & \multicolumn{1}{p{2.1cm}}{\raggedright{}Piercing or Slashing} & Medium\\

\multicolumn{8}{p{17cm}}{\cellcolor{white}\textsuperscript{1} When two types are given, the weapon is both types if the entry specifies "and", or either type (player's choice at time of attack) if the entry specifies "or".}\\
\multicolumn{8}{p{17cm}}{\cellcolor{white}\textsuperscript{2} Weapons can be made up or down in size, but it's uncommon within civilization.}\\
\multicolumn{8}{l}{\cellcolor{white}\textsuperscript{N} The weapon deals nonlethal damage rather than lethal damage.}\\
\multicolumn{8}{l}{\cellcolor{white}\textsuperscript{R} Reach weapon.}\\
\end{smallbasictable}

\begin{smallbasictable}{Exotic Melee Weapons}{l c c c c c c l}
\textbf{Weapon} & \textbf{Cost} & \textbf{Dmg} & \textbf{Critical} & \textbf{Range} & \textbf{Weight} & \textbf{Type\textsuperscript{1}} & \multicolumn{1}{p{1cm}}{\raggedright{}\textbf{Weapon Size\textsuperscript{2}}}\\

\multicolumn{8}{l}{\textbf{Tiny Weapons}}\\
\hspace{.5cm}Kama & 2gp & 1d6 & x2 &  & 2 lb & Slashing & Tiny\\
\hspace{.5cm}Nunchaku & 2gp & 1d6 & x2 &  & 2 lb & Bludgeoning & Tiny\\
\hspace{.5cm}Sai & 2gp & 1d4 & x2 & 10ft & 1 lb & Bludgeoning & Tiny\\
\hspace{.5cm}Siangham & 3gp & 1d5 & x2 &  & 1 lb & Piercing & Tiny\\

\multicolumn{8}{l}{\textbf{Small Weapons}}\\
\hspace{.5cm}Dwarven Waraxe & 30gp& 1d10 & x3 &  & 8 lb & Slashing & Small\\
\hspace{.5cm}Gnome Hooked Hammer \textsuperscript{D} & 20gp & 1d6/1d4 & x3/x4 &  & 6 lb & \multicolumn{1}{p{2.1cm}}{\raggedright{}Bludgeoning and Piercing} & Small\\
\hspace{.5cm}Whip \textsuperscript{R} & 1gp & 1d3 & x2 &  & 2 lb & Slashing & Small\\

\multicolumn{8}{l}{\textbf{Medium Weapons}}\\
\hspace{.5cm}Dire Flail \textsuperscript{D} & 90gp & 1d8/1d8 & x2 &  & 10 lb & Bludgeoning & Small\\
\hspace{.5cm}Dwarven Urgosh \textsuperscript{D} & 50gp & 1d8/1d6 & x3 &  & 12 lb & \multicolumn{1}{p{2.1cm}}{\raggedright{}Slashing or Piercing} & Small\\
\hspace{.5cm}Orc Double Axe \textsuperscript{D} & 60gp & 1d8/1d8 & x3 &  & 15 lb & Slashing & Small\\
\hspace{.5cm}Spiked Chain \textsuperscript{R} & 25gp & 2d4 & x2 &  & 10 lb & Piercing & Small\\
\hspace{.5cm}Two-Bladed Sword \textsuperscript{D} & 100gp & 1d8/1d8 & 19-20/x2 &  & 10 lb & Slashing & Small\\

\multicolumn{8}{p{17cm}}{\cellcolor{white}\textsuperscript{1} When two types are given, the weapon is both types if the entry specifies "and", or either type (player's choice at time of attack) if the entry specifies "or".}\\
\multicolumn{8}{p{17cm}}{\cellcolor{white}\textsuperscript{2} Weapons can be made up or down in size, but it's uncommon within civilization.}\\
\multicolumn{8}{l}{\cellcolor{white}\textsuperscript{R} Reach weapon.}\\
\multicolumn{8}{l}{\cellcolor{white}\textsuperscript{D} Double weaopn.}\\
\end{smallbasictable}

\begin{smallbasictable}{Ranged Weapons}{l c c c c c c l}
\textbf{Weapon} & \textbf{Cost} & \textbf{Dmg} & \textbf{Critical} & \textbf{Range} & \textbf{Weight} & \textbf{Type\textsuperscript{1}} & \multicolumn{1}{p{1cm}}{\raggedright{}\textbf{Weapon Size\textsuperscript{2}}}\\

\multicolumn{8}{l}{\textbf{Simple Weapons}}\\
\hspace{.5cm}Dart & 5sp & 1d4 & x2 & 20ft & 1/2 lb & Piercing & Diminutive\\
\hspace{.5cm}Sling &  & 1d4 & x2 & 50ft & 0 lb & Bludgeoning & Diminutive\\
\hspace{1cm}Bullets (10) & 1sp &  &  &  & 5 lb & &\\
\hspace{.5cm}Javelin & 1gp & 1d6 & x2 & 30ft & 2 lb & Piercing & Tiny\\
\hspace{.5cm}Light Crossbow & 35gp & 1d8 & 19-20/x2 & 80ft & 4 lb & Piercing & Small\\
\hspace{.5cm}Heavy Crossbow & 50gp & 1d10 & 19-20/x2 & 120ft & 8 lb & Piercing & Medium\\
\hspace{1cm}Bolts (10) & 1gp &  &  &  & 1 lb &  &\\

\multicolumn{8}{l}{\textbf{Martial Weapons}}\\
\hspace{.5cm}Composite Shortbow & 75gp & 1d6 & x3 & 70ft & 2 lb & Piercing & Small\\
\hspace{.5cm}Shortbow & 30gp & 1d6 & x3 & 60ft & 2 lb & Piercing & Small\\
\hspace{.5cm}Composite Longbow & 100gp & 1d8 & x3 & 110ft & 3 lb & Piercing & Medium\\
\hspace{.5cm}Longbow & 75gp & 1d8 & x3 & 110ft & 3 lb & Piercing & Medium\\
\hspace{1cm}Arrows (20) & 1gp &  &  &  & 1 lb &  &\\

\multicolumn{8}{l}{\textbf{Exotic Weapons}}\\
\hspace{.5cm}Shuriken (5) & 1gp& 1d2 & x2 & 10ft & 1/2 lb & Piercing & Diminutive\\
\hspace{.5cm}Bolas & 5gp & 1d4\textsuperscript{N} & x2 & 10ft & 2 lb & Bludgeoning & Tiny\\
\hspace{.5cm}Hand Crossbow & 100gp & 1d4 & 19-20/x2 & 30ft & 2 lb & Piercing & Tiny\\
\hspace{1cm}Bolts (10) & 1gp &  &  &  & 1lb &  &\\
\hspace{.5cm}Light Repeating Crossbow & 250gp & 1d8 & 19-20/x2 & 80ft & 6 lb & Piercing & Small\\
\hspace{.5cm}Heavy Repeating Crossbow & 400gp & 1d10 & 19-20/x2 & 120ft & 12 lb & Piercing & Medium\\
\hspace{1cm}Bolt Case (5) & 1gp &  &  &  & 1 lb &  &\\
\hspace{.5cm}Net & 20gp &  &  & 10ft & 6 lb &  & Medium\\

\multicolumn{8}{p{17cm}}{\cellcolor{white}\textsuperscript{1} When two types are given, the weapon is both types if the entry specifies "and", or either type (player's choice at time of attack) if the entry specifies "or".}\\
\multicolumn{8}{p{17cm}}{\cellcolor{white}\textsuperscript{2} Weapons can be made up or down in size, but it's uncommon within civilization.}\\
\multicolumn{8}{l}{\cellcolor{white}\textsuperscript{N} The weapon deals nonlethal damage rather than lethal damage.}\\
\end{smallbasictable}

\begin{multicols}{2}

Weapons found on Table: Weapons that have special options for the wielder ("you") are described below. Splash weapons are described under Special Substances and Items.

\textbf{Arrows:} An arrow used as a melee weapon is treated as a light improvised weapon (-4 penalty on attack rolls) and deals damage as a dagger of its size (critical multiplier x2). Arrows come in a leather quiver that holds 20 arrows.

\textbf{Bolas:} You can use this weapon to make a ranged trip attack against an opponent. You can't be tripped during your own trip attempt when using a set of bolas.

\textbf{Bolts:} A crossbow bolt used as a melee weapon is treated as a light improvised weapon (-4 penalty on attack rolls) and deals damage as a dagger of its size (crit x2). Bolts come in a wooden case that holds 10 bolts (or 5, for a repeating crossbow). 

\textbf{Bullets:} Bullets come in a leather pouch that holds 10 bullets.

\textbf{Composite Longbow:} You need at least two hands to use a bow, regardless of its size. You can use a composite longbow while mounted. All composite bows are made with a particular strength rating (that is, each requires a minimum Strength modifier to use with proficiency). If your Strength bonus is less than the strength rating of the composite bow, you can't effectively use it, so you take a -2 penalty on attacks with it. The default composite longbow requires a Strength modifier of +0 or higher to use with proficiency. A composite longbow can be made with a high strength rating to take advantage of an above-average Strength score; this feature allows you to add your Strength bonus to damage, up to the maximum bonus indicated for the bow. Each point of Strength bonus granted by the bow adds 100 gp to its cost.

For purposes of weapon proficiency and similar feats, a composite longbow is treated as if it were a longbow.

\textbf{Composite Shortbow:} You need at least two hands to use a bow, regardless of its size. You can use a composite shortbow while mounted. All composite bows are made with a particular strength rating (that is, each requires a minimum Strength modifier to use with proficiency). If your Strength bonus is lower than the strength rating of the composite bow, you can't effectively use it, so you take a -2 penalty on attacks with it. The default composite shortbow requires a Strength modifier of +0 or higher to use with proficiency. A composite shortbow can be made with a high strength rating to take advantage of an above-average Strength score; this feature allows you to add your Strength bonus to damage, up to the maximum bonus indicated for the bow. Each point of Strength bonus granted by the bow adds 75 gp to its cost. 

For purposes of weapon proficiency and similar feats, a composite shortbow is treated as if it were a shortbow.

\textbf{Dagger:} You get a +2 bonus on \linkskill{Sleight of Hand} checks made to conceal a dagger on your body (see the Sleight of Hand skill).

\textbf{Dire Flail:} A dire flail is a double weapon. You can fight with it as if fighting with two weapons, but if you do, you incur all the normal attack penalties associated with fighting with two weapons, just as if you were using a one-handed weapon and a light weapon. A creature wielding a dire flail in one hand can't use it as a double weapon --  only one end of the weapon can be used in any given round.

When using a dire flail, you get a +2 bonus on opposed attack rolls made to disarm an enemy (including the opposed attack roll to avoid being disarmed if such an attempt fails).

You can also use this weapon to make trip attacks. If you are tripped during your own trip attempt, you can drop the dire flail to avoid being tripped.

\textbf{Dwarven Waraxe:} A dwarven waraxe is too large to use in one hand without special training; thus, it is an exotic weapon. A Medium character can use a dwarven waraxe two-handed as a martial weapon, or a Large creature can use it one-handed in the same way. A dwarf treats a dwarven waraxe as a martial weapon even when using it in one hand.

\textbf{Flail or Heavy Flail:} With a flail, you get a +2 bonus on opposed attack rolls made to disarm an enemy (including the roll to avoid being disarmed if such an attempt fails).

You can also use this weapon to make trip attacks. If you are tripped during your own trip attempt, you can drop the flail to avoid being tripped.

\textbf{Gauntlet:} This metal glove lets you deal lethal damage rather than nonlethal damage with unarmed strikes. A strike with a gauntlet is otherwise considered an unarmed attack. The cost and weight given are for a single gauntlet. Medium and heavy armors (except breastplate) come with gauntlets.

\textbf{Glaive:} A glaive has reach. You can strike opponents 10 feet away with it, but you can't use it against an adjacent foe.

\textbf{Gnome Hooked Hammer:} A gnome hooked hammer is a double weapon. You can fight with it as if fighting with two weapons, but if you do, you incur all the normal attack penalties associated with fighting with two weapons, just as if you were using a one-handed weapon and a light weapon. The hammer's blunt head is a bludgeoning weapon that deals 1d6 points of damage (crit x3). Its hook is a piercing weapon that deals 1d4 points of damage (crit x4). You can use either head as the primary weapon. The other head is the offhand weapon. A creature wielding a gnome hooked hammer in one hand can't use it as a double weapon -- only one end of the weapon can be used in any given round.

You can use a gnome hooked hammer to make trip attacks. If you are tripped during your own trip attempt, you can drop the gnome hooked hammer to avoid being tripped.

\textbf{Guisarme:} A guisarme has reach. You can strike opponents 10 feet away with it, but you can't use it against an adjacent foe.

You can also use it to make trip attacks. If you are tripped during your own trip attempt, you can drop the guisarme to avoid being tripped.

\textbf{Halberd:} If you use a ready action to set a halberd against a charge, you deal double damage on a successful hit against a charging character.

You can use a halberd to make trip attacks. If you are tripped during your own trip attempt, you can drop the halberd to avoid being tripped.

\textbf{Hand Crossbow:} You can draw a hand crossbow back by hand. Loading a hand crossbow is a move action that provokes attacks of opportunity.

You can shoot, but not load, a hand crossbow with one hand at no penalty. You can shoot a hand crossbow with each hand, but you take a penalty on attack rolls as if attacking with two light weapons.

\textbf{Heavy Crossbow:} You draw a heavy crossbow back by turning a small winch. Loading a heavy crossbow is a full-round action that provokes attacks of opportunity.

Normally, operating a heavy crossbow requires two hands. However, you can shoot, but not load, a heavy crossbow with one hand at a -4 penalty on attack rolls. You can shoot a heavy crossbow with each hand, but you take a penalty on attack rolls as if attacking with two one-handed weapons. This penalty is cumulative with the penalty for one-handed firing.

\textbf{Javelin:} Since it is not designed for melee, you are treated as nonproficient with it and take a -4 penalty on attack rolls if you use a javelin as a melee weapon.

\textbf{Kama:} The kama is a special monk weapon. This designation gives a monk wielding a kama special options.

You can use a kama to make trip attacks. If you are tripped during your own trip attempt, you can drop the kama to avoid being tripped.

\textbf{Lance:} A lance deals double damage when used from the back of a charging mount. It has reach, so you can strike opponents 10 feet away with it, but you can't use it against an adjacent foe.

While mounted, you can wield a lance with one hand.

\textbf{Light Crossbow:} You draw a light crossbow back by pulling a lever. Loading a light crossbow is a move action that provokes attacks of opportunity.

Normally, operating a light crossbow requires two hands. However, you can shoot, but not load, a light crossbow with one hand at a -2 penalty on attack rolls. You can shoot a light crossbow with each hand, but you take a penalty on attack rolls as if attacking with two light weapons. This penalty is cumulative with the penalty for one-handed firing.

\textbf{Longbow:} You need at least two hands to use a bow, regardless of its size. A longbow is too unwieldy to use while you are mounted. If you have a penalty for low Strength, apply it to damage rolls when you use a longbow. If you have a bonus for high Strength, you can apply it to damage rolls when you use a composite longbow (see below) but not a regular longbow.

\textbf{Longspear:} A longspear has reach. You can strike opponents 10 feet away with it, but you can't use it against an adjacent foe. If you use a ready action to set a longspear against a charge, you deal double damage on a successful hit against a charging character.

\textbf{Net:} A net is used to entangle enemies. When you throw a net, you make a ranged touch attack against your target. A net's maximum range is 10 feet. If you hit, the target is entangled. An entangled creature takes a -2 penalty on attack rolls and a -4 penalty on Dexterity, can move at only half speed, and cannot charge or run. If you control the trailing rope by succeeding on an opposed Strength check while holding it, the entangled creature can move only within the limits that the rope allows. If the entangled creature attempts to cast a spell, it must make a DC 15 Concentration check or be unable to cast the spell.

An entangled creature can escape with a DC 20 Escape Artist check (a full-round action). The net has 5 hit points and can be burst with a DC 25 Strength check (also a full-round action).

A net is useful only against creatures within one size category of the net.

A net must be folded to be thrown effectively. The first time you throw your net in a fight, you make a normal ranged touch attack roll. After the net is unfolded, you take a -4 penalty on attack rolls with it. It takes 2 rounds for a proficient user to fold a net and twice that long for a nonproficient one to do so.

\textbf{Nunchaku:} The nunchaku is a special monk weapon. This designation gives a monk wielding a nunchaku special options. With a nunchaku, you get a +2 bonus on opposed attack rolls made to disarm an enemy (including the roll to avoid being disarmed if such an attempt fails).

\textbf{Orc Double Axe:} An orc double axe is a double weapon. You can fight with it as if fighting with two weapons, but if you do, you incur all the normal attack penalties associated with fighting with two weapons, just as if you were using a one-handed weapon and a light weapon.

A creature wielding an orc double axe in one hand can't use it as a double weapon -- only one end of the weapon can be used in any given round.

\textbf{Quarterstaff:} A quarterstaff is a double weapon. You can fight with it as if fighting with two weapons, but if you do, you incur all the normal attack penalties associated with fighting with two weapons, just as if you were using a one-handed weapon and a light weapon. A creature wielding a quarterstaff in one hand can't use it as a double weapon -- only one end of the weapon can be used in any given round.

The quarterstaff is a special monk weapon. This designation gives a monk wielding a quarterstaff special options.

\textbf{Ranseur:} A ranseur has reach. You can strike opponents 10 feet away with it, but you can't use it against an adjacent foe.

With a ranseur, you get a +2 bonus on opposed attack rolls made to disarm an opponent (including the roll to avoid being disarmed if such an attempt fails).

\textbf{Rapier:} A Rapier is one size categories lower for light, one, or two handed designation.

\textbf{Repeating Crossbow:} The repeating crossbow (whether heavy or light) holds 5 crossbow bolts. As long as it holds bolts, you can reload it by pulling the reloading lever (a free action). Loading a new case of 5 bolts is a full-round action that provokes attacks of opportunity.

You can fire a repeating crossbow with one hand or fire a repeating crossbow in each hand in the same manner as you would a normal crossbow of the same size. However, you must fire the weapon with two hands in order to use the reloading lever, and you must use two hands to load a new case of bolts.

\textbf{Sai:} With a sai, you get a +4 bonus on opposed attack rolls made to disarm an enemy (including the roll to avoid being disarmed if such an attempt fails).

The sai is a special monk weapon. This designation gives a monk wielding a sai special options.

\textbf{Scythe:} A scythe can be used to make trip attacks. If you are tripped during your own trip attempt, you can drop the scythe to avoid being tripped.

\textbf{Shield, Heavy or Light:} You can bash with a shield instead of using it for defense. See Armor for details.

\textbf{Shortbow:} You need at least two hands to use a bow, regardless of its size. You can use a shortbow while mounted. If you have a penalty for low Strength, apply it to damage rolls when you use a shortbow. If you have a bonus for high Strength, you can apply it to damage rolls when you use a composite shortbow (see below) but not a regular shortbow.

\textbf{Shortspear:} A shortspear is small enough to wield one-handed. It may also be thrown.

\textbf{Shuriken:} A shuriken is a special monk weapon. This designation gives a monk wielding shuriken special options. A shuriken can't be used as a melee weapon.

Although they are thrown weapons, shuriken are treated as ammunition for the purposes of drawing them, crafting masterwork or otherwise special versions of them and what happens to them after they are thrown.

\textbf{Siangham:} The siangham is a special monk weapon. This designation gives a monk wielding a siangham special options.

\textbf{Sickle:} A sickle can be used to make trip attacks. If you are tripped during your own trip attempt, you can drop the sickle to avoid being tripped.

\textbf{Sling:} Your Strength modifier applies to damage rolls when you use a sling, just as it does for thrown weapons. You can fire, but not load, a sling with one hand. Loading a sling is a move action that requires two hands and provokes attacks of opportunity.

You can hurl ordinary stones with a sling, but stones are not as dense or as round as bullets. Thus, such an attack deals damage as if the weapon were designed for a creature one size category smaller than you and you take a -1 penalty on attack rolls.

\textbf{Spear:} A spear can be thrown. If you use a ready action to set a spear against a charge, you deal double damage on a successful hit against a charging character.

\textbf{Spiked Armor:} You can have spikes added to your armor, which allow you to deal extra piercing damage on a successful grapple attack. The spikes count as a martial weapon. If you are not proficient with them, you take a –4 penalty on grapple checks when you try to use them. You can also make a regular melee attack.

\textbf{Spiked Chain:} A spiked chain has reach, so you can strike opponents 10 feet away with it. In addition, unlike most other weapons with reach, it can be used against an adjacent foe.

You can make trip attacks with the chain. If you are tripped during your own trip attempt, you can drop the chain to avoid being tripped.

When using a spiked chain, you get a +2 bonus on opposed attack rolls made to disarm an opponent (including the roll to avoid being disarmed if such an attempt fails).

You can use the Weapon Finesse feat to apply your Dexterity modifier instead of your Strength modifier to attack rolls with a spiked chain sized for you, even though it isn't a light weapon for you.

\textbf{Spiked Gauntlet:} Your opponent cannot use a disarm action to disarm you of spiked gauntlets. The cost and weight given are for a single gauntlet. An attack with a spiked gauntlet is considered an armed attack.

\textbf{Spiked Shield, Heavy or Light:} You can bash with a spiked shield instead of using it for defense. See Armor for details.

\textbf{Trident:} This weapon can be thrown. If you use a ready action to set a trident against a charge, you deal double damage on a successful hit against a charging character.

\textbf{Two-Bladed Sword:} A two-bladed sword is a double weapon. You can fight with it as if fighting with two weapons, but if you do, you incur all the normal attack penalties associated with fighting with two weapons, just as if you were using a one-handed weapon and a light weapon. A creature wielding a two-bladed sword in one hand can't use it as a double weapon -- only one end of the weapon can be used in any given round.

\textbf{Urgrosh, Dwarven:} A dwarven urgrosh is a double weapon. You can fight with it as if fighting with two weapons, but if you do, you incur all the normal attack penalties associated with fighting with two weapons, just as if you were using a one-handed weapon and a light weapon. The urgrosh's axe head is a slashing weapon that deals 1d8 points of damage. Its spear head is a piercing weapon that deals 1d6 points of damage. You can use either head as the primary weapon. The other is the off-hand weapon. A creature wielding a dwarven urgrosh in one hand can't use it as a double weapon -- only one end of the weapon can be used in any given round.

If you use a ready action to set an urgrosh against a charge, you deal double damage if you score a hit against a charging character. If you use an urgrosh against a charging character, the spear head is the part of the weapon that deals damage.

\textbf{Unarmed Strike:} A Medium character deals 1d3 points of nonlethal damage with an unarmed strike. A Small character deals 1d2 points of nonlethal damage. A monk or any character with the Improved Unarmed Strike feat can deal lethal or nonlethal damage with unarmed strikes, at her option. The damage from an unarmed strike is considered weapon damage for the purposes of effects that give you a bonus on weapon damage rolls.

An unarmed strike is always considered a light weapon. Therefore, you can use the Weapon Finesse feat to apply your Dexterity modifier instead of your Strength modifier to attack rolls with an unarmed strike.

\textbf{Whip:} A whip deals nonlethal damage. It deals no damage to any creature with an armor bonus of +1 or higher or a natural armor bonus of +3 or higher. The whip is treated as a melee weapon with 15-foot reach, though you don't threaten the area into which you can make an attack. In addition, unlike most other weapons with reach, you can use it against foes anywhere within your reach (including adjacent foes).

Using a whip provokes an attack of opportunity, just as if you had used a ranged weapon.

You can make trip attacks with a whip. If you are tripped during your own trip attempt, you can drop the whip to avoid being tripped.

When using a whip, you get a +2 bonus on opposed attack rolls made to disarm an opponent (including the roll to keep from being disarmed if the attack fails).

You can use the Weapon Finesse feat to apply your Dexterity modifier instead of your Strength modifier to attack rolls with a whip sized for you, even though it isn't a light weapon for you.

%%%%%%%%%%%%%%%%%%%%%%%%%
\subsection{Masterwork Weapons}
%%%%%%%%%%%%%%%%%%%%%%%%%

A masterwork weapon is a finely crafted version of a normal weapon. Wielding it provides a +1 enhancement bonus on attack rolls.

You can't add the masterwork quality to a weapon after it is created; it must be crafted as a masterwork weapon (see the Craft skill). The masterwork quality adds 300 gp to the cost of a normal weapon (or 6 gp to the cost of a single unit of ammunition). Adding the masterwork quality to a double weapon costs twice the normal increase (+600 gp).

Masterwork ammunition is damaged (effectively destroyed) when used. The enhancement bonus of masterwork ammunition does not stack with any enhancement bonus of the projectile weapon firing it.

All magic weapons are automatically considered to be of masterwork quality. The enhancement bonus granted by the masterwork quality doesn't stack with the enhancement bonus provided by the weapon's magic.

Even though some types of armor and shields can be used as weapons, you can't create a masterwork version of such an item that confers an enhancement bonus on attack rolls. Instead, masterwork armor and shields have lessened armor check penalties.
